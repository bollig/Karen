\chapter{ViennaCL}

\authnote{4 pages}

The ViennaCL project \cite{Rupp2010a, Rupp2010} is a sparse matrix library built on OpenCL that provides templated C++ API to easily and efficiently solve large sparse matrix problems. 

Our decision to utilize ViennaCL stems from its 

The API provides containers for sparse and dense matrices, vectors and views. The sparse formats available are COO, CSR, (...) ELL and HYB

Matrix and Vector Views provide slices and subranges of containers. Subviews were recently added in version 1.2 with full functionality in v1.3. These views are essential to our work on multi-GPUs since we need to operate on views with full vectors containing the ghost values. (Show example of view). 

Algorithms are provided for GMRES, CG, etc. 

To build multi-GPU algorithms for GMRES we discard the ViennaCL arnoldi process in favor of the Givens rotations. Then we add MPI communication. 



\section{ViennaCL Limitations}
ViennaCL is a young library. When we began working with the library, it supported only CSR and COO formats. It has no support for multi-GPU computing. According to the author there were no intentions to extend it multiple GPUs due to the limited return of investment. 

\section{ViennaCL Additions}
Added support for distributed SpMV and SAXPY with MPI communication. 

Extended library to support rectangular systems rather than just square systems. 

Introduced alternative GMRES algorithm based on Givens Rotations

Introduced preconditioner ILU0. 

\section{ViennaCL Matrix Formats}

ViennaCL supports numerous sparse matrix formats. 

Assemble once in CSR format using direct access notation (e.g., mat(row,col) = val), internally convert to more efficient sparse representation. 