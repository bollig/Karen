
\makeatletter
\@ifundefined{standalonetrue}{\newif\ifstandalone}{}
\@ifundefined{section}{\standalonetrue}{\standalonefalse}
\makeatother
\ifstandalone
\documentclass{report}

\usepackage{textcase}
\usepackage[pdftex]{graphicx}
%\usepackage{hyperref}
%\hypersetup{breaklinks=true}


% Added packages
\usepackage[usenames]{color}
\usepackage{amsfonts, amsmath, amssymb, graphics}

% NOTE: bibentry MUST appear before the hyperref or build will fail
\usepackage{bibentry}
\nobibliography*
\usepackage[square,sort,comma,numbers]{natbib}
  
\usepackage{float}
\usepackage[
    hyperindex=true,		% Make numbers of index links as well
   	backref=page, 		% Provide page listing where refs occur in the bibliography
	%breaklinks=true,
    colorlinks,%
    citecolor=green,%
    filecolor=blue,%
    linkcolor=red,%
    urlcolor=red, 
]{hyperref}

\usepackage{dsfont}
%%%% USEPACKAGES for MACROS %%%%%
\usepackage{algpseudocode}
\usepackage[chapter]{algorithm}
\usepackage{caption}
\usepackage{subcaption}
\usepackage{url}

\usepackage{array}
\usepackage{arydshln}
\usepackage{multirow}
\usepackage{multicol}
\usepackage[section]{placeins}

\newcommand{\toprule}[0]{\hline}
\newcommand{\midrule}[0]{\hline\hline}
\newcommand{\bottomrule}[0]{\hline}

\DeclareSymbolFont{AMSb}{U}{msb}{m}{n}
\DeclareMathSymbol{\N}{\mathbin}{AMSb}{"4E}
\DeclareMathSymbol{\Z}{\mathbin}{AMSb}{"5A}
\DeclareMathSymbol{\R}{\mathbin}{AMSb}{"52}
\DeclareMathSymbol{\Q}{\mathbin}{AMSb}{"51}
\DeclareMathSymbol{\PP}{\mathbin}{AMSb}{"50}
\DeclareMathSymbol{\I}{\mathbin}{AMSb}{"49}
%\DeclareMathSymbol{\C}{\mathbin}{AMSb}{"43}

%%%%%% VECTOR NORM: %%%%%%%
\newcommand{\vectornorm}[1]{\left|\left|#1\right|\right|}
\newcommand{\vnorm}[1]{\left|\left|#1\right|\right|}
\newcommand{\by}[0]{\times}
\newcommand{\vect}[1]{\mathbf{#1}}
%\newcommand{\mat}[1]{\mathbf{#1}} 

%\renewcommand{\vec}[1]{ \textbf{#1} }
%%%%%%%%%%%%%%%%%%%%%%

%%%%%%% THM, COR, DEF %%%%%%%
%\newtheorem{theorem}{Theorem}[section]
%\newtheorem{lemma}[theorem]{Lemma}
%\newtheorem{proposition}[theorem]{Proposition}
%\newtheorem{corollary}[theorem]{Corollary}
%\newenvironment{proof}[1][Proof]{\begin{trivlist}
%\item[\hskip \labelsep {\bfseries #1}]}{\end{trivlist}}
%\newenvironment{definition}[1][Definition]{\begin{trivlist}
%\item[\hskip \labelsep {\bfseries #1}]}{\end{trivlist}}
%\newenvironment{example}[1][Example]{\begin{trivlist}
%\item[\hskip \labelsep {\bfseries #1}]}{\end{trivlist}}
%\newenvironment{remark}[1][Remark]{\begin{trivlist}
%\item[\hskip \labelsep {\bfseries #1}]}{\end{trivlist}}
%\newcommand{\qed}{\nobreak \ifvmode \relax \else
%      \ifdim\lastskip<1.5em \hskip-\lastskip
%      \hskip1.5em plus0em minus0.5em \fi \nobreak
%      \vrule height0.75em width0.5em depth0.25em\fi}
%%%%%%%%%%%%%%%%%%%%%%


% \DeclareMathOperator{\Sample}{Sample}
%\let\vaccent=\v % rename builtin command \v{} to \vaccent{}
%\renewcommand{\vec}[1]{\ensuremath{\mathbf{#1}}} % for vectors
\newcommand{\gv}[1]{\ensuremath{\mbox{\boldmath$ #1 $}}} 
% for vectors of Greek letters
\newcommand{\uv}[1]{\ensuremath{\mathbf{\hat{#1}}}} % for unit vector
\newcommand{\abs}[1]{\left| #1 \right|} % for absolute value
\newcommand{\avg}[1]{\left< #1 \right>} % for average
\let\underdot=\d % rename builtin command \d{} to \underdot{}
\renewcommand{\d}[2]{\frac{d #1}{d #2}} % for derivatives
\newcommand{\dd}[2]{\frac{d^2 #1}{d #2^2}} % for double derivatives
\newcommand{\pd}[2]{\frac{\partial #1}{\partial #2}} 
% for partial derivatives
\newcommand{\pdd}[2]{\frac{\partial^2 #1}{\partial #2^2}} 
\newcommand{\pdda}[3]{\frac{\partial^2 #1}{\partial #2 \partial #3}} 
% for double partial derivatives
\newcommand{\pdc}[3]{\left( \frac{\partial #1}{\partial #2}
 \right)_{#3}} % for thermodynamic partial derivatives
\newcommand{\ket}[1]{\left| #1 \right>} % for Dirac bras
\newcommand{\bra}[1]{\left< #1 \right|} % for Dirac kets
\newcommand{\braket}[2]{\left< #1 \vphantom{#2} \right|
 \left. #2 \vphantom{#1} \right>} % for Dirac brackets
\newcommand{\matrixel}[3]{\left< #1 \vphantom{#2#3} \right|
 #2 \left| #3 \vphantom{#1#2} \right>} % for Dirac matrix elements
\newcommand{\grad}[1]{\gv{\nabla} #1} % for gradient
\let\divsymb=\div % rename builtin command \div to \divsymb
\renewcommand{\div}[1]{\gv{\nabla} \cdot #1} % for divergence
\newcommand{\curl}[1]{\gv{\nabla} \times #1} % for curl
\let\baraccent=\= % rename builtin command \= to \baraccent
\renewcommand{\=}[1]{\stackrel{#1}{=}} % for putting numbers above =
\newcommand{\diffop}[1]{\mathcal{L}#1}
\newcommand{\boundop}[1]{\mathcal{B}#1}
\newcommand{\rvec}[0]{{\bf r}}

\newcommand{\Interior}[0]{\Omega}
\newcommand{\domain}[0]{\Omega}
\newcommand{\Boundary}[0]{\partial \Omega}
%\newcommand{\Boundary}[0]{\Gamma}

\newcommand{\on}[1]{\hskip1.5em \textrm{ on } #1}

\newcommand{\gemm}{\texttt{GEMM}}
\newcommand{\trmm}{\texttt{TRMM}}
\newcommand{\gesvd}{\texttt{GESVD}}
\newcommand{\geqrf}{\texttt{GEQRF}}


\newcommand{\minitab}[2][l]{\begin{tabular}{#1}#2\end{tabular}}
\newcommand{\comm}[1]{\textcolor{red}{\textit{#1}}}

\newcommand{\nfrac}[2]{
\nicefrac{#1}{#2}
%\frac{#1}{#2}
}

\usepackage{xparse}


%%%%%%%%%%%%%%%
% Show a Author's Note
% USAGE: 
% \incomplete[Optional footnote message to further clarify note]{The text which is currently not finished}
\DeclareDocumentCommand \incomplete{ o m }
{%
\IfNoValueTF {#1}
{\textcolor{red}{Incomplete: \ul{#2}}} 
{\textcolor{red}{Incomplete: \ul{#2}}\footnote{Comment: #1}}%
}
%%%%%%%%%%%%%%%



%%%%%%%%%%%%%%%
% Show a Author's Note
% USAGE: 
% \authnote[Optional footnote message to further clarify note]{The note to your readers}
\DeclareDocumentCommand \authnote { o m }
{%
\IfNoValueTF {#1}
{\textcolor{blue}{Author's Note: \ul{#2}}} 
{\textcolor{blue}{Author's Note: \ul{#2}}\footnote{Comment: #1}}%
}
%%%%%%%%%%%%%%%



%%%%%%%%%%%%%%%
% Strike out text that doesn't belong in the paper
% USAGE: 
% \strike[Optional footnote to state why it doesn't belong]{Text to strike out}
\DeclareDocumentCommand \strike { o m }
{%
\setstcolor{red}
\IfNoValueTF {#1}
{\textcolor{Gray}{\st{#2}}} 
{\textcolor{Gray}{\st{#2}}\footnote{Comment: #1}}%
}
%%%%%%%%%%%%%%%



%
% colors to show the corrections
\newcommand{\red}[1]{\textbf{\textcolor{red}{#1}}}
\newcommand{\blue}[1]{\textbf{\textcolor{blue}{#1}}}
\newcommand{\cyan}[1]{\textbf{\textcolor{cyan}{#1}}}
\newcommand{\green}[1]{\textbf{\textcolor{green}{#1}}}
\newcommand{\magenta}[1]{\textbf{\textcolor{magenta}{#1}}}
\newcommand{\orange}[1]{\textbf{\textcolor{orange}{#1}}}
%%%%%%%%%% DK DK
% comments between authors
\newcommand{\toall}[1]{\textbf{\green{@@@ All: #1 @@@}}}
\newcommand{\toevan}[1]{\textbf{\red{*** Evan: #1 ***}}}
%\newcommand{\toevan}[1]{}  % USE FOR FINAL VERSION
\newcommand{\toe}[1]{\textbf{\red{*** Evan: #1 ***}}}
\newcommand{\tog}[1]{\textbf{\blue{*** Gordon: #1 ***}}}
%\newcommand{\togordon}[1]{\textbf{\blue{*** Gordon: #1 ***}}}
\renewcommand{\ge}[3]{{\textcolor{blue}{*** \textbf{Gordon:}\strike{#1} #2 ***}}\red{(#3)}}
\renewcommand{\ge}[3]{{\textcolor{blue}{#2}}}
\renewcommand{\ge}[3]{{\textcolor{red}{#2}}}
\newcommand{\eb}[3]{{\textcolor{red}{*** \textbf{Evan:}\strike{#1} #2 ***}}\red{(#3)}}
\renewcommand{\eb}[3]{{{\textcolor{red}{#2}}}}
%\def\ge#1#2#3{}{\textbf{\blue{*** Gordon: #2 ***}}}{(#3)}
\newcommand{\gee}[1]{{\bf{\blue{{\em #1}}}}}
\newcommand{\old}[1]{}
\newcommand{\del}[1]{***#1*** }



\input{macros/misc_mac.tex}
\newcommand{\mathsym}[1]{{}}
\newcommand{\unicode}[1]{{}}
\newcommand{\ep}{\epsilon}
\newcommand{\vx}{\mathbf{x}}


\usepackage{tabularx} 
\newcolumntype{C}{>{\centering\arraybackslash}b{1in}}
\newcolumntype{L}{>{\flushleft\arraybackslash}b{1.5in}}
\newcolumntype{R}{>{\flushright\arraybackslash}b{1.5in}}
\newcolumntype{D}{>{\flushright\arraybackslash}b{2.0in}}
\newcolumntype{E}{>{\flushright\arraybackslash}b{1.0in}}


 


\usepackage{xcolor}
% Sepia
\definecolor{myBGcolor}{HTML}{F6F0D6}
\definecolor{myTextcolor}{HTML}{4F452C}
% Dark
%\definecolor{myBGcolor}{HTML}{3E3535}
%\definecolor{myTextcolor}{HTML}{CFECEC}
%\color{myTextcolor}
\pagecolor{myBGcolor}
 
\usepackage[margin=1.25in]{geometry}
\usepackage{xcolor}

% Sepia
%\definecolor{myBGcolor}{HTML}{F6F0D6}
%\definecolor{myTextcolor}{HTML}{4F452C}

\definecolor{myBGcolor}{HTML}{3E3535}
\definecolor{myTextcolor}{HTML}{CFECEC}
\pagecolor{myBGcolor}
\color{myTextcolor}

\begin{document}
\fi

{ \graphicspath{{rbffd_methods_content/}} 


\subsection{Global RBF Method}
%todo: integrate above...this content is repeated, but better formatted
Consider a PDE expressed in terms of (linear) differential operators, $\diffop$ and $\boundop{}$: 
\begin{eqnarray}
\diffop{u} & = & f \on{\Interior}, \nonumber \\
\boundop{u} &=& g \on{\Boundary} 
\end{eqnarray}
where $\Interior$ is the interior of the physical domain, $\Boundary$ is the boundary of $\Interior$ and $f,g$ are known explicitly. In the case of a non-linear differential operator, a Newton's iteration, or some other method, can be used to linearize the problem (see e.g., \cite{WrightFornberg06}); of course, this increases the complexity of a single time step. 

%We solve PDEs of this form with \emph{collocation}. That is, given a number of points in the domain, we seek the derivative and solution values that best satisfy $f, g$ at node locations. \authnote{Not clear yet. Review Lynch2005 for a better explanation}\cite{lynch2004numerical}merical}

Global RBF collocation methods pose the problem of interpolating a multivariate function $f : \domain
    \rightarrow \R$ where $\domain \subset \R^m$. Given a set of sample values
    $\{f(x_j)\}_{j=1}^{N}$ on a discrete set of nodes $X = \{x_j\}_{j=1}^{N}
    \subset \domain$, an approximation $\hat{f}_N$ can be constructed through
    linear combinations of interpolation functions. Here, we choose univariate,
    radially symmetric functions based on Euclidean distance ($\vnorm{\cdot}$), and use
    translates $\phi(x-x_j)$ of a single continuous real
    valued function $\phi$ defined on $\R$ and centered at $x_j$:
         \begin{equation*} 
         \phi(x) := \varphi(\vnorm{x}).
         \end{equation*} 
    %with a continuous function $\varphi$ on $\R_0^{+}$. 
    Here, $\varphi$ is a
    Radial Basis Function and $\phi$ the
    associated kernel. For simplification $\phi_j(x)$ refers to a kernel centered at $x_j$; i.e., $\varphi(\vectornorm{x-x_j})$.

   The interpolant $\hat{f}_N(x)$ requires a linear combination of translates:  
        \begin{equation*}
      %  \hat{f}_N(x) = \sum_{j=1}^{N} c_j \varphi(\vnorm{x-x_j})
        \hat{f}_N(x) = \sum_{j=1}^{N} c_j \phi_j(x)
        \end{equation*}
    with real coefficients $\{c_j\}_{j=1}^{N}$. Assuming the interpolant passes through
    known values of $f$; i.e., 
        \begin{equation*} 
        \hat{f}_N(x_i) = f(x_i),\ \ \ \ \ 1 \leq i \leq N, 
        \end{equation*}
    allows one to solve for coefficients if the following linear system is uniquely
    solvable: 
        \begin{equation*} 
        %\sum_{j=1}^{N} c_j \varphi(\vnorm{x_i-x_j}) = f(x_i),\ \ \ \ \ 1
        \sum_{j=1}^{N} c_j \phi_j(x_i) = f(x_i),\ \ \ \ \ 1
        \leq i \leq N. 
        \end{equation*} 
    This is true if the $N \by N$ matrix $\phi$ produced by the linear system
%        \begin{align} 
%          \begin{pmatrix}  
%            \varphi(\vnorm{x_1 - x_1}) & \varphi(\vnorm{x_1 - x_2}) & \cdots & \varphi(\vnorm{x_1 - x_N}) \\ 
%            \varphi(\vnorm{x_2 - x_1}) & \varphi(\vnorm{x_2 - x_2}) & \cdots & \varphi(\vnorm{x_2 - x_N}) \\ 
%            \vdots & \ddots & \ddots & \vdots \\
%            \varphi(\vnorm{x_N - x_1}) & \varphi(\vnorm{x_N - x_2}) & \cdots & \varphi(\vnorm{x_N - x_N})
%                \end{pmatrix} 
%                \begin{bmatrix} c_1 \\ c_2 \\ \vdots \\ c_N \end{bmatrix}
%               &=                \begin{bmatrix} f(x_1) \\ f(x_2) \\ \vdots \\ f(x_N) \end{bmatrix} \\
%                         \phi \vec{c} &= \vec{f} 
%        \end{align} 
        \begin{align*} 
          \begin{pmatrix}  
            \phi_1(x_1) & \phi_2(x_1) & \cdots & \phi_N(x_1) \\ 
            \phi_1(x_2) & \phi_2(x_2) & \cdots & \phi_N(x_2) \\ 
            \vdots & \ddots & \ddots & \vdots \\
            \phi_1(x_N) & \phi_2(x_N) & \cdots & \phi_N(x_N)
                \end{pmatrix} 
                \begin{bmatrix} c_1 \\ c_2 \\ \vdots \\ c_N \end{bmatrix}
               &=                \begin{bmatrix} f(x_1) \\ f(x_2) \\ \vdots \\ f(x_N) \end{bmatrix} \\
                         \phi \vec{c} &= \vec{f} 
        \end{align*} 
is nonsingular.

 When choosing an appropriate basis function for interpolation, a
    subset of RBFs have been shown to produce symmetric positive definite
    $\phi$, while others are only conditionally positive definite (for more details see
    \cite{Fasshauer2007}).  With
    the latter set, additional polynomial terms are added to constrain the
    system and enforce positive definiteness, resulting in this expanded system of equations: 
        \begin{align} 
       % \sum_{j=1}^{N} c_j \varphi(\vnorm{x_i-x_j}) + \sum_{k=1}^{t}d_k p_k(x_i) &= f(x_i),\ \ \ \ \ 1
        \sum_{j=1}^{N} c_j \phi_j(x_i) + \sum_{k=1}^{M}d_k p_k(x_i) &= f(x_i),\ \ \ \ \ 1
        \leq i \leq N, \nonumber \\ 
        \sum_{j=1}^{N} c_j p_k(x_j) &=  0,\ \ \ \ \ \ \ \ \ \ 1 \leq k \leq M. \nonumber \\
          \begin{pmatrix} \phi & P \\ P^T & 0 \end{pmatrix} \begin{bmatrix} c \\ d \end{bmatrix} &= \begin{bmatrix} \vec{f} \\ 0 \end{bmatrix}
	\label{eq:additional_constraints}
        \end{align} 
        where $\{p_k\}_{k=1}^{M}$ is a basis for $\Pi_{p}^{m} $ (the set of polynomials in $m$ variables of degree $\leq p$) and 
       \begin{equation*}
       M = {{p+m}\choose{m}}       .
       \end{equation*} 
      %\authnote{Add table? or cut:} Table~\ref{tbl:rbfoptions} provides example RBF functions and notes their (conditionally) positive definiteness. 


Given the coefficients $\vec{c}$, the function value at a test point $x$ is interpolated by
\begin{align}	 
	\hat{f}_N(x) &= \sum_{j=1}^{N}c_j \phi_j(x) + \sum_{l=1}^{M}d_l P_l(x)  \nonumber \\
	&=  \left[\begin{array}{cc}
        \phi & \PP
	\end{array} \right] 
	 \begin{bmatrix}	c \\
					d
		 \end{bmatrix} \nonumber \\
						 &= \vec{\phi}_x^T \vec{c} \nonumber \\
						 &= \vec{\phi}_x^T \phi^{-1}\vec{f}	
						 \label{eq:rbf_interpolate_pt}
\end{align}
where $\vec{c}$ is substituted by the solution to Equation~\ref{eq:additional_constraints}. In Equation~\ref{eq:rbf_interpolate_pt} the term $\phi_{x}^T\phi^{-1}$, dependent only on node positions, can be evaluated prior to knowing $f$. 


%In comparison, calculating the coefficient vectors $c$ and $d$ requires evaluating the basis functions and polynomial part at a set of known collocation points. 





}

\ifstandalone
\bibliographystyle{plain}
\bibliography{merged_references}
\end{document}
\else
\expandafter\endinput
\fi


