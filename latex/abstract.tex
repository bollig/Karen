\begin{abstract}
    Many numerical methods based on Radial Basis Functions (RBFs) are gaining popularity in the geosciences due to their competitive accuracy, functionality on unstructured meshes, and natural extension into higher dimensions. One method in particular, the Radial Basis Function-generated Finite Differences (RBF-FD), has drawn significant attention due to its comparatively low computational complexity versus other RBF methods, high-order accuracy (6th to 10th order is common), and parallel nature. 

    Similar to classical Finite Differences (FD), RBF-FD computes weighted differences of stencil node values to approximate derivatives at stencil centers. The method differs from classical FD in that the test functions used to calculate the differentiation weights are $n$-dimensional RBFs rather than one-dimensional polynomials. This allows for generalization to $n$-dimensional space on completely scattered node layouts.

Although RBF-FD was first proposed nearly a decade ago, it is only now gaining a critical mass to compete against well known competitors in modeling like FD, Finite Volume and Finite Element. To truly contend, RBF-FD must transition from single threaded MATLAB environments to large-scale parallel architectures.

Many HPC systems around the world have made the transition to Graphics Processing Unit (GPU) accelerators as a solution for added parallelism and higher throughput. Some systems offer significantly more GPUs than CPUs. As the problem size, $N$, grows larger, it behooves us to work on parallel architectures, be it CPUs or GPUs. In addition to demonstrating the ability to scale to hundreds or thousands of compute nodes, this work introduces parallelization strategies that span RBF-FD across multi-GPU clusters. The stability and accuracy of the parallel implementations are tested in explicit and implicit modes to verify correctness.
    
This work establishes RBF-FD as a contender in the arena of distributed HPC numerical methods. 
    
\end{abstract}