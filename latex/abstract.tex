\begin{abstract}
    Many numerical methods based on Radial Basis Functions (RBFs) are gaining popularity in the geosciences due to their competitive accuracy, functionality on unstructured meshes, and natural extension into higher dimensions. One method in particular, the Radial Basis Function-generated Finite Differences (RBF-FD), has drawn significant attention due to its comparatively low computational complexity versus other RBF methods, high-order accuracy (6th to 10th order is common), and embarrassingly parallel nature. 

    Similar to classical Finite Differences, RBF-FD computes weighted differences of stencil node values to approximate derivatives at stencil centers. The method differs from classical FD in that the test functions used to calculate the differentiation weights are $n$-dimensional RBFs rather than one-dimensional polynomials. This allows for generalization to $n$-dimensional space on completely scattered node layouts.

    We present our effort to capitalize on the parallelism within RBF-FD for geophysical flow. Many HPC systems around the world are transitioning toward significantly more Graphics Processing Unit (GPU) accelerators than CPUs. As the problem size, $N$, grows larger, it behooves us to work on parallel architectures, be it CPUs or GPUs. In addition to spanning hundreds of compute nodes, this work introduces parallelization strategies to span RBF-FD across GPU clusters. The stability and accuracy of our parallel implementations are presented to verify correctness.
    
\end{abstract}