\chapter{Final Discussion, Conclusions and Future Work}
\label{chap:conclusions}


\section{Future Work}

This work leads to a number of additional investigations:
\begin{itemize}
\item Transitioning away from RBF-FD specific OpenCL kernels and into the higher level matrix algebra library ViennaCL reveals new optimizations for SpMV, possible only when treating the sparse matrix from a higher level. There are a number of alternatives for SpMV that target various matrix features (e.g., jagged diagonal, blocked ELL, etc). New investigation will determine the which, if any, alternatives can benefit RBF-FD. 

\item ViennaCL has support for multiple backends, including the Intel Phi. Initial experiments result in performance $\frac{1}{10}$th as fast as the K20 GPU. Those results are: a) limited by a beta driver for OpenCL on Intel Phi; and b) using OpenCL kernels intended for the GPU architecture. Continued investigations are targeting features (language, vector instructions, etc.) of the Phi. 
\item To improve the accuracy of RBF-FD weights, an experimental RBF-GA implementation was started. The initial exposure to the algorithm leaves the impression that the change of basis on the RHS is a nuisance. In effort to bypass the difficulty in deriving new RHS expressions, we plan to assemble complex DMs using DMs of lower (i.e., first-) order operators.
\end{itemize}

%Unanswered Questions: 
%\begin{itemize} 
%\item If RBF-FD is higher order accurate than FV, why would anyone opt for FV?
%\begin{itemize} 
%\item the conservative form FV is preferred in physical science since it does not allow loss of (what? mass/energy/???)
%\item FV are typically low-order. what is the trade-off in complexity between methods. For a much higher-order RBF-FD if it is not conservative, how many iterations is it accurate consistent with the low-order FV? l
%\end{itemize}
%\end{itemize}

\authnote{BEGIN: reformat this and draw conclusions. repeat what I have concluding each chapter}
Within this document readers should expect to find the following major contributions: 
\begin{enumerate} 
\item The first ever application of a fixed-grid neighbor query algorithm (popularized by distantly related particle methods) to generate stencils for RBF-FD. 
\begin{itemize} 
\item Performance comparison with the RBF community favorite, $k$-D Tree, shows that the lower complexity fixed-grid method is able to achieve up to 2.5x speedup over the former method on randomly distributed nodes. 
\item The fixed-grid method is also shown to accelerate RBF-FD time-step performance by up to 5x due spatial locality of node information in memory and improved cache effects.
\end{itemize} 
\item Integer dilation and bit interleaving are introduced to the RBF community at large to construct a number of space-filling curve variants in 2-D and 3-D. The curves reorder cells of the fixed-grid algorithm on the prospect of additional cache hits for derivative calculations. 
\begin{itemize} 
\item A comparison of the variants to the result of the well-known Reverse Cuthill-McKee (RCM) algorithm concludes that RCM remains the better option for reordering and matrix bandwidth minimization.
\end{itemize} 
\item The design and tuning of the only known implementation of the RBF-FD method to scale across multiple compute nodes of a supercomputing cluster.
\begin{itemize} 
\item Our home-grown implementation divides computation with a Restricted Additive Schwarz (RAS) domain decomposition in a first-of-its-kind application to RBF-FD. % (note: other RBF methods have already encountered RAS decomposition \cite{Yokota2010}). 
\item As part of the tuning process described herein, balanced workloads are achieved through the use of the METIS graph partitioning algorithm \cite{Karypis1999}.  
\item A number of steps are taken to improve communication collectives in the CPU-only implementation. The resulting algorithm splits derivative calculation into two steps and overlaps communication with computation. We observe that up to 80\% of the cost in communication can be hidden in some cases. 
\item Scaling benchmarks up to 1024 processes (divided into 8 processes per node) and a grid resolution of $N=160^3$ vertices (i.e., 4.1 million) prove that the implementation scales well in both a strong and weak sense. 
\end{itemize} 
\item The design and tuning of the only known single- and multi-GPU implementation of RBF-FD. Also, ours is the only known investigation in the broader RBF community to target multiple GPUs (\cite{Schmidt2009a,Cuomo2013} are single-GPU). 
\begin{itemize} 
\item Our first paper (\cite{BolligFlyerErlebacher2012}) introduced a set of custom OpenCL kernels for RBF-FD time-stepping of hyperbolic PDEs with an explicit RK4 scheme (up to 10 GPUs). 
\item Here, the performance per GPU is tuned by choosing an alternative sparse matrix representation. The investigation reveals that the Ellpack (ELL) structure provided by the ViennaCL library \cite{Rupp2010} is a great fit for RBF-FD differentiation matrices, with over 4x faster performance than the original compressed-row storage (CSR). %On one GPU (NVidia M2070) our implementation performs at approximately 8 GFLOP/sec. 
\item The multi-GPU implementation is extended to a novel overlapping algorithm that naturally derives from our distributed CPU optimizations. Non-blocking MPI collectives plus two asynchronous OpenCL queues amortize the cost of data transfer between CPU and GPU, MPI communication, and in some cases a substantial amount of computation. Iterations of the resulting implementation achieve an average of 3x better strong scaling compared to a non-overlapping equivalent. 
\end{itemize} 
\end{enumerate}
Of lesser significance we also: 
\begin{enumerate}
\item Verify our explicit solvers against two well studied hyperbolic PDEs (Vortex Roll-up \cite{NairTransport05, NairJablonowski08} and Cosine Bell Advection \cite{JakobChien1995}).
\begin{itemize} 
\item Convergence studies confirm that hyperviscosity (\cite{Fornberg2011b}) stabilizes solutions.
\item Tuned parameters are provided for selecting the RBF support parameter and scaling hyperviscosity as a function of the problem size
\end{itemize}
\item Implement a distributed multi-GPU preconditioned GMRES within ViennaCL for implicit RBF-FD solutions.
\begin{itemize} 
\item A divergence-free field is constructed as the manufactured solution for a simplified version of Stokes flow constrained to the unit sphere. 
\item A CPU-only preconditioner (Incomplete LU with Zero Fill-in) is implemented to test impact on GMRES convergence rates 
\end{itemize}
\item Provide preliminary benchmarks evaluating the performance of RBF-FD on the Intel Phi Architecture. 
\end{enumerate}
\authnote{END}

\section{On the Future of GPU Computing}

%TODO: Move this to the conclusions chapter
%  addresss this as: 
%   Future of GPU is with CUDA and vendor specific features. 
%   AMD, Intel are trying to compete, but NVidia has the market and new entrants should stick with CUDA. 


%The question, however, is frequently asked: is GPU computing a buzzword that will die out, or is it here to stay? 
%
%Fine tuning of kernels is the way of the past. Anyone fine tuning kernels will be operating at a lower level, attempting to optimize library routines underlying large scale codes. The fine tuning can ultimately be replaced by auto-tuned kernels (e.g., see work in ViennaCL to auto-tune similar to libATLAS). 
%
%Already this is seen in attempts to port applications to the GPU in so-called ``minimally invasive" fashion. 
%
%So the short answer is: yes. And it is from this assumption that we proceed with our efforts to target the GPU with RBF-FD. However, we have made a few predictions on the future of GPUs and these predictions guided our efforts to port RBFs onto GPUs. 
%
%First, we are proponents for OpenCL over CUDA. Although CUDA has a large following due to its early release, we believe in functional portability of code enabled by OpenCL over the performance provided by CUDA. 

New features on NVidia Kepler level GPUs and CUDA v5 allow dynamic parallelism. Dynamic parallelism provides the ability to generate new levels of parallelism on the GPU rather than return to the CPU.  Dynamic parallelism also leads to stream priority features which allow kernels to preemptively execute out of order based on priority and dependencies. Such a feature is powerful for MPI support where the GPU can continue processing kernels while waiting on MPI communication to complete. 


\section{Other Future Work}
In Section~\ref{sec:epsilon_contours} we mention the need for stable methods to solve for weights. We have partially implemented the RBF-GA method from \cite{Fornberg2012} 




