
\newcommand{\rbf}[4]{#1 & #2 & #3 & #4}
\begin{table}[t]
   \centering
   \begin{tabular}{L | C | c |C| } % Column formatting, @{} suppresses leading/trailing space
   \rbf{Name}{Abbrev.}{Formula}{Order ($m$)} \\
   \hline\hline
   \rbf{Multiquadric}{MQ}{$\sqrt{1+(\varepsilon r)^2}$}{1} \\
   \rbf{Inverse Multiquadric}{IMQ}{$\frac{1}{\sqrt{1+(\varepsilon r)^2}}$}{0} \\
   \rbf{Gaussian}{GA}{$e^{-(\varepsilon r)^2}$}{0} \\
   \rbf{Thin Plate Splines}{TPS}{$r^2 ln |r|$}{2} \\
   \rbf{Wendland ($C^2$)}{W2}{$(1-\varepsilon r)^4 (4\varepsilon r + 1)$}{0}
   \end{tabular}
   \caption{Examples of frequently used RBFs based on \cite{Fornberg2008, Fasshauer2007}. $\varepsilon$ is the support parameter. 
   %For details on orders refer to Chapter~\ref{chap:rbf_pde} and \cite{Iske2004}. 
   All RBFs have global support. For compact support, enforce a cut-off radius (see Equation~\ref{eqn:csrbf}).}
   \label{tbl:rbfs}
\end{table}