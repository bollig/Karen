%\makeatletter
%\@ifundefined{standalonetrue}{\newif\ifstandalone}{}
%\@ifundefined{section}{\standalonetrue}{\standalonefalse}
%\makeatother
%\ifstandalone
%\documentclass{report}
%
%\usepackage{textcase}
\usepackage[pdftex]{graphicx}
%\usepackage{hyperref}
%\hypersetup{breaklinks=true}


% Added packages
\usepackage[usenames]{color}
\usepackage{amsfonts, amsmath, amssymb, graphics}

% NOTE: bibentry MUST appear before the hyperref or build will fail
\usepackage{bibentry}
\nobibliography*
\usepackage[square,sort,comma,numbers]{natbib}
  
\usepackage{float}
\usepackage[
    hyperindex=true,		% Make numbers of index links as well
   	backref=page, 		% Provide page listing where refs occur in the bibliography
	%breaklinks=true,
    colorlinks,%
    citecolor=green,%
    filecolor=blue,%
    linkcolor=red,%
    urlcolor=red, 
]{hyperref}

\usepackage{dsfont}
%%%% USEPACKAGES for MACROS %%%%%
\usepackage{algpseudocode}
\usepackage[chapter]{algorithm}
\usepackage{caption}
\usepackage{subcaption}
\usepackage{url}

\usepackage{array}
\usepackage{arydshln}
\usepackage{multirow}
\usepackage{multicol}
\usepackage[section]{placeins}

\newcommand{\toprule}[0]{\hline}
\newcommand{\midrule}[0]{\hline\hline}
\newcommand{\bottomrule}[0]{\hline}

\DeclareSymbolFont{AMSb}{U}{msb}{m}{n}
\DeclareMathSymbol{\N}{\mathbin}{AMSb}{"4E}
\DeclareMathSymbol{\Z}{\mathbin}{AMSb}{"5A}
\DeclareMathSymbol{\R}{\mathbin}{AMSb}{"52}
\DeclareMathSymbol{\Q}{\mathbin}{AMSb}{"51}
\DeclareMathSymbol{\PP}{\mathbin}{AMSb}{"50}
\DeclareMathSymbol{\I}{\mathbin}{AMSb}{"49}
%\DeclareMathSymbol{\C}{\mathbin}{AMSb}{"43}

%%%%%% VECTOR NORM: %%%%%%%
\newcommand{\vectornorm}[1]{\left|\left|#1\right|\right|}
\newcommand{\vnorm}[1]{\left|\left|#1\right|\right|}
\newcommand{\by}[0]{\times}
\newcommand{\vect}[1]{\mathbf{#1}}
%\newcommand{\mat}[1]{\mathbf{#1}} 

%\renewcommand{\vec}[1]{ \textbf{#1} }
%%%%%%%%%%%%%%%%%%%%%%

%%%%%%% THM, COR, DEF %%%%%%%
%\newtheorem{theorem}{Theorem}[section]
%\newtheorem{lemma}[theorem]{Lemma}
%\newtheorem{proposition}[theorem]{Proposition}
%\newtheorem{corollary}[theorem]{Corollary}
%\newenvironment{proof}[1][Proof]{\begin{trivlist}
%\item[\hskip \labelsep {\bfseries #1}]}{\end{trivlist}}
%\newenvironment{definition}[1][Definition]{\begin{trivlist}
%\item[\hskip \labelsep {\bfseries #1}]}{\end{trivlist}}
%\newenvironment{example}[1][Example]{\begin{trivlist}
%\item[\hskip \labelsep {\bfseries #1}]}{\end{trivlist}}
%\newenvironment{remark}[1][Remark]{\begin{trivlist}
%\item[\hskip \labelsep {\bfseries #1}]}{\end{trivlist}}
%\newcommand{\qed}{\nobreak \ifvmode \relax \else
%      \ifdim\lastskip<1.5em \hskip-\lastskip
%      \hskip1.5em plus0em minus0.5em \fi \nobreak
%      \vrule height0.75em width0.5em depth0.25em\fi}
%%%%%%%%%%%%%%%%%%%%%%


% \DeclareMathOperator{\Sample}{Sample}
%\let\vaccent=\v % rename builtin command \v{} to \vaccent{}
%\renewcommand{\vec}[1]{\ensuremath{\mathbf{#1}}} % for vectors
\newcommand{\gv}[1]{\ensuremath{\mbox{\boldmath$ #1 $}}} 
% for vectors of Greek letters
\newcommand{\uv}[1]{\ensuremath{\mathbf{\hat{#1}}}} % for unit vector
\newcommand{\abs}[1]{\left| #1 \right|} % for absolute value
\newcommand{\avg}[1]{\left< #1 \right>} % for average
\let\underdot=\d % rename builtin command \d{} to \underdot{}
\renewcommand{\d}[2]{\frac{d #1}{d #2}} % for derivatives
\newcommand{\dd}[2]{\frac{d^2 #1}{d #2^2}} % for double derivatives
\newcommand{\pd}[2]{\frac{\partial #1}{\partial #2}} 
% for partial derivatives
\newcommand{\pdd}[2]{\frac{\partial^2 #1}{\partial #2^2}} 
\newcommand{\pdda}[3]{\frac{\partial^2 #1}{\partial #2 \partial #3}} 
% for double partial derivatives
\newcommand{\pdc}[3]{\left( \frac{\partial #1}{\partial #2}
 \right)_{#3}} % for thermodynamic partial derivatives
\newcommand{\ket}[1]{\left| #1 \right>} % for Dirac bras
\newcommand{\bra}[1]{\left< #1 \right|} % for Dirac kets
\newcommand{\braket}[2]{\left< #1 \vphantom{#2} \right|
 \left. #2 \vphantom{#1} \right>} % for Dirac brackets
\newcommand{\matrixel}[3]{\left< #1 \vphantom{#2#3} \right|
 #2 \left| #3 \vphantom{#1#2} \right>} % for Dirac matrix elements
\newcommand{\grad}[1]{\gv{\nabla} #1} % for gradient
\let\divsymb=\div % rename builtin command \div to \divsymb
\renewcommand{\div}[1]{\gv{\nabla} \cdot #1} % for divergence
\newcommand{\curl}[1]{\gv{\nabla} \times #1} % for curl
\let\baraccent=\= % rename builtin command \= to \baraccent
\renewcommand{\=}[1]{\stackrel{#1}{=}} % for putting numbers above =
\newcommand{\diffop}[1]{\mathcal{L}#1}
\newcommand{\boundop}[1]{\mathcal{B}#1}
\newcommand{\rvec}[0]{{\bf r}}

\newcommand{\Interior}[0]{\Omega}
\newcommand{\domain}[0]{\Omega}
\newcommand{\Boundary}[0]{\partial \Omega}
%\newcommand{\Boundary}[0]{\Gamma}

\newcommand{\on}[1]{\hskip1.5em \textrm{ on } #1}

\newcommand{\gemm}{\texttt{GEMM}}
\newcommand{\trmm}{\texttt{TRMM}}
\newcommand{\gesvd}{\texttt{GESVD}}
\newcommand{\geqrf}{\texttt{GEQRF}}


\newcommand{\minitab}[2][l]{\begin{tabular}{#1}#2\end{tabular}}
\newcommand{\comm}[1]{\textcolor{red}{\textit{#1}}}

\newcommand{\nfrac}[2]{
\nicefrac{#1}{#2}
%\frac{#1}{#2}
}

\usepackage{xparse}


%%%%%%%%%%%%%%%
% Show a Author's Note
% USAGE: 
% \incomplete[Optional footnote message to further clarify note]{The text which is currently not finished}
\DeclareDocumentCommand \incomplete{ o m }
{%
\IfNoValueTF {#1}
{\textcolor{red}{Incomplete: \ul{#2}}} 
{\textcolor{red}{Incomplete: \ul{#2}}\footnote{Comment: #1}}%
}
%%%%%%%%%%%%%%%



%%%%%%%%%%%%%%%
% Show a Author's Note
% USAGE: 
% \authnote[Optional footnote message to further clarify note]{The note to your readers}
\DeclareDocumentCommand \authnote { o m }
{%
\IfNoValueTF {#1}
{\textcolor{blue}{Author's Note: \ul{#2}}} 
{\textcolor{blue}{Author's Note: \ul{#2}}\footnote{Comment: #1}}%
}
%%%%%%%%%%%%%%%



%%%%%%%%%%%%%%%
% Strike out text that doesn't belong in the paper
% USAGE: 
% \strike[Optional footnote to state why it doesn't belong]{Text to strike out}
\DeclareDocumentCommand \strike { o m }
{%
\setstcolor{red}
\IfNoValueTF {#1}
{\textcolor{Gray}{\st{#2}}} 
{\textcolor{Gray}{\st{#2}}\footnote{Comment: #1}}%
}
%%%%%%%%%%%%%%%



%
% colors to show the corrections
\newcommand{\red}[1]{\textbf{\textcolor{red}{#1}}}
\newcommand{\blue}[1]{\textbf{\textcolor{blue}{#1}}}
\newcommand{\cyan}[1]{\textbf{\textcolor{cyan}{#1}}}
\newcommand{\green}[1]{\textbf{\textcolor{green}{#1}}}
\newcommand{\magenta}[1]{\textbf{\textcolor{magenta}{#1}}}
\newcommand{\orange}[1]{\textbf{\textcolor{orange}{#1}}}
%%%%%%%%%% DK DK
% comments between authors
\newcommand{\toall}[1]{\textbf{\green{@@@ All: #1 @@@}}}
\newcommand{\toevan}[1]{\textbf{\red{*** Evan: #1 ***}}}
%\newcommand{\toevan}[1]{}  % USE FOR FINAL VERSION
\newcommand{\toe}[1]{\textbf{\red{*** Evan: #1 ***}}}
\newcommand{\tog}[1]{\textbf{\blue{*** Gordon: #1 ***}}}
%\newcommand{\togordon}[1]{\textbf{\blue{*** Gordon: #1 ***}}}
\renewcommand{\ge}[3]{{\textcolor{blue}{*** \textbf{Gordon:}\strike{#1} #2 ***}}\red{(#3)}}
\renewcommand{\ge}[3]{{\textcolor{blue}{#2}}}
\renewcommand{\ge}[3]{{\textcolor{red}{#2}}}
\newcommand{\eb}[3]{{\textcolor{red}{*** \textbf{Evan:}\strike{#1} #2 ***}}\red{(#3)}}
\renewcommand{\eb}[3]{{{\textcolor{red}{#2}}}}
%\def\ge#1#2#3{}{\textbf{\blue{*** Gordon: #2 ***}}}{(#3)}
\newcommand{\gee}[1]{{\bf{\blue{{\em #1}}}}}
\newcommand{\old}[1]{}
\newcommand{\del}[1]{***#1*** }



\input{macros/misc_mac.tex}
\newcommand{\mathsym}[1]{{}}
\newcommand{\unicode}[1]{{}}
\newcommand{\ep}{\epsilon}
\newcommand{\vx}{\mathbf{x}}


\usepackage{tabularx} 
\newcolumntype{C}{>{\centering\arraybackslash}b{1in}}
\newcolumntype{L}{>{\flushleft\arraybackslash}b{1.5in}}
\newcolumntype{R}{>{\flushright\arraybackslash}b{1.5in}}
\newcolumntype{D}{>{\flushright\arraybackslash}b{2.0in}}
\newcolumntype{E}{>{\flushright\arraybackslash}b{1.0in}}


 


\usepackage{xcolor}
% Sepia
\definecolor{myBGcolor}{HTML}{F6F0D6}
\definecolor{myTextcolor}{HTML}{4F452C}
% Dark
%\definecolor{myBGcolor}{HTML}{3E3535}
%\definecolor{myTextcolor}{HTML}{CFECEC}
%\color{myTextcolor}
\pagecolor{myBGcolor}
 
%
%\begin{document}
%\fi

\chapter{Avoiding Pole Singularities with RBF-FD}
\label{app:rbffd_remove_singularities}

This content follows \cite{FlyerWright07,FlyerWright09}. 

%Within the test cases of this dissertation, we solve convective PDEs on the unit sphere with the form: 
%$$
%\pd{h}{t} = \vu \cdot \nabla h
%$$
%where $\vu$ is velocity. For example, t
Chapter~\ref{chap:applications} introduces cosine bell advection as
\begin{equation}
\pd{h}{t} = \frac{u}{\cos{\theta}} \pd{h}{\lambda} + v \pd{h}{\theta} \label{eq:cosine_bell_appendix}
\end{equation}
in the spherical coordinate system defined by
\begin{align*}
x & = \cos{\theta}\cos{\lambda} \\
y & = \cos{\theta}\sin{\lambda} \\
z & = \sin{\theta}
\end{align*}
where $\theta \in (-\frac{\pi}{2}, \frac{\pi}{2})$ is the elevation angle and $\lambda \in (-\pi,\pi)$ is the azimuthal angle.
Observe that as $\theta \rightarrow \pm \frac{\pi}{2}$, the $\frac{1}{\cos{\theta}}$ term goes to infinity as a discontinuity. 

One of the many selling points for RBF-FD and other RBF methods is their ability analytically avoid pole singularities, which arise from the choice of coordinate system and not from the methods themselves. Since RBFs are based on Euclidean distance between nodes, and not the geodesic distance, it is said that they do not ``feel'' the effects of the geometry or recognize singularities naturally inherent in the coordinate system \cite{FlyerWright07}. 
Here we demonstrate how pole singularities are analytically avoided with RBF-FD for cosine bell advection.  


Let $r = || \vx - \vx_j ||$ be the Euclidean distance which is invariant of the coordinate system. In Cartesian coordinates, 
$$
r = \sqrt{(x-x_j)^2 + (y-y_j)^2 + (z-z_j)^2}.
$$
In spherical coordinates,
$$
r = \sqrt{2(1-\cos{\theta}\cos{\theta_j}\cos{(\lambda-\lambda_j)} - \sin{\theta}\sin{\theta_j})}.
$$

The RBF-FD operators for $\d{}{\lambda}, \d{}{\theta}$ are discretized with the chain rule: 

\begin{align}
\d{\phi_{j}(r)}{\lambda} = \d{r}{\lambda} \d{\phi_{j}(r)}{r} & = \frac{\cos{\theta}\cos{\theta_j}\sin{(\lambda - \lambda_j)}}{r} \d{\phi_j(r)}{r}, \label{eq:rbfffd_d_dlambda} \\
\d{\phi_{j}(r)}{\theta} = \d{r}{\theta} \d{\phi_{j}(r)}{r} & = \frac{\sin{\theta}\cos{\theta_j}\cos{(\lambda-\lambda_j)} - \cos{\theta}\sin{\theta_j}}{r} \d{\phi}{r}, \label{eq:rbfffd_d_dtheta}
\end{align}
where $\phi_{j}(r)$ is the RBF centered at $\vx_{j}$. 

Plugging \ref{eq:rbfffd_d_dlambda} and \ref{eq:rbfffd_d_dtheta} into \ref{eq:cosine_bell_appendix}, produces the following explicit form: 
\begin{align*}
\d{h}{t} = u(\cos{\theta_j}\sin{(\lambda - \lambda_j)} \frac{1}{r} \d{\phi_j}{r}) + v (\sin{\theta}\cos{\theta_j}\cos{(\lambda-\lambda_j)} - \cos{\theta}\sin{\theta_j} \frac{1}{r} \d{\phi}{r}) 
\end{align*}
where $\cos{\theta}$ from \ref{eq:rbfffd_d_dlambda} analytically cancels with the $\frac{1}{\cos{\theta}}$ in \ref{eq:cosine_bell_appendix}.


Then, formally, one would assemble differentiation matrices containing weights for the following operators: 
\begin{align}
\D_\lambda & = \cos{\theta_j}\sin{(\lambda - \lambda_j)} \frac{1}{r} \d{\phi_j}{r}, \label{eq:dm_lambda} \\
\D_\theta &=  \sin{\theta}\cos{\theta_j}\cos{(\lambda-\lambda_j)} - \cos{\theta}\sin{\theta_j} \frac{1}{r} \d{\phi}{r}, \label{eq:dm_theta}
\end{align} 
and solve the explicit method of lines problem:
$$
\d{h}{t} = u \D_\lambda h + v \D_\theta h
$$
where now the system is completely free of singularities at the poles \cite{FlyerWright09}. 

We note that the expression $\cos{(\frac{\pi}{2})}$ evaluates on some systems to a very small value rather than zero (e.g., $6.1(10^{-17})$ on the Keeneland system with the GNU gcc compiler). The small value in turn causes $\frac{1}{\cos{\theta}}$ to result in a large value (e.g., $1.6(10^{16})$) rather than ``inf" or ``NaN". An ``inf'' or ``NaN'' would corrupt the numerics, but the large value allows terms to cancel in double precision. Rather than avoid placing nodes at the poles, or assume singularities cancel numerically, it is preferred to use operators \ref{eq:dm_lambda} and \ref{eq:dm_theta} to analytically cancel singularities when computing RBF-FD weights. 

We observe that the latter approach causes the conditioning of the system to change slightly. The hyperviscosity parameters provided in Chapter~\ref{chap:applications} function well in both cases. However, their impact on the eigenvalue distributions in the case of analytic cancellation is noticeably stronger and results in a larger spread of eigenvalues in the left half-plane. 
%
%\ifstandalone
%\bibliographystyle{plain}
%\bibliography{merged_references}
%\end{document}
%\else
%\expandafter\endinput
%\fi
