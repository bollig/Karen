\begin{acknowledgments}

I feel compelled to start by thanking my major professor, Dr. Gordon Erlebacher, who shared with me a unique attitude and excitement in regard to research. After years of collaboration, I find myself echoing his own audacious style in questioning the work done by others; never quite satisfied by their responses. Without his supervision and constant prodding this dissertation would not have been possible. %Gordon's skill as a computational scientist is exemplary. Time and again I have watched him fearlessly work in areas outside his own domain of expertise in a way I can only hope to simulate. 

Of no lesser significance is Dr. Natasha Flyer, who is not formally listed as a co-advisor due to restrictions at the university level, but who has in all capacities fulfilled that role. I am sincerely grateful for the time spent at NCAR under her tutelage. In addition to fortifying my understanding of RBF methods and geophysical PDEs, Natasha shared a number of unforgettable life-lessons that reflect in my choice of employment and pursuit of happiness today. 

I also extend my deepest gratitude to the remainder of my committee for their patience and willingness to participate in this venture. 

In no particular order, I am indebted to:
\begin{itemize}
\item Leeia for being my best friend, biggest fan, and for supporting me through every trial and tribulation on the path to complete this work;
\item Drs. David Yuen, Bengt Fornberg, Grady Wright, John Burkardt and Kiran Katta for their support and feedback on numerical modeling;
\item the RBF research group at CU-Boulder for a number of ideas that led to interesting investigations;
\item Dr. Geoff Womeldorff for insightful discussion and the high resolution Spherical CVT meshes tested herein;
\item Steven Henke for a number of discussions regarding our concurrent investigations into space-filling curves, neighbor queries, and GPU computing;
\item Brent Swartz, Yectli Huerta, Andrew Ring, and the rest of the University of Minnesota Supercomputing Institute for playing sounding board to my ideas and providing access to computational resources;
\item the faculty and students in the FSU Department of Scientific Computing (DSC) who built a community of collaboration and interdisciplinary research that I am proud to be part of;
\item the DSC staff (esp. Cecelia Farmer, Maribel Amwake, Dawn Singleton and Risette Posey) for their tireless support and assistance;  
\item and past and present students in Gordon's research group (esp. Ian Johnson, Andrew Young, and Nathan Crock) for brainstorming sessions, weekend hack-a-thons and camaraderie.
\end{itemize}

This research was funded in part by NSF awards DMS-\#0934331 (FSU), DMS-\#0934317 (NCAR) and ATM-\#0602100 (NCAR). It used resources of the University of Minnesota Supercomputing Institute, as well as the Keeneland Computing Facility at the Georgia Institute of Technology, which is supported by the National Science Foundation under Contract OCI-0910735.

I now close with a very special thank you to my loved ones (both friends and family) who have supported me during the course of this work. You have waited patiently as I devoted all time and energy into completing this milestone. I look forward to resuming life with all of you in it. 

%
%\begin{quote}
%This is science, not politics! Don't you know the difference? A politician starts in the middle with ``$ = b$" and draws conclusions based on smoke and mirrors. In science you have no choice but to start from the beginning with something like ``$Ax = b$". (Gordon Erlebacher, c. 2010). 
%\end{quote} 
%

\end{acknowledgments}
