%\makeatletter
%\@ifundefined{standalonetrue}{\newif\ifstandalone}{}
%\@ifundefined{section}{\standalonetrue}{\standalonefalse}
%\makeatother
%\ifstandalone
%\documentclass{report}
%
%\usepackage{textcase}
\usepackage[pdftex]{graphicx}
%\usepackage{hyperref}
%\hypersetup{breaklinks=true}


% Added packages
\usepackage[usenames]{color}
\usepackage{amsfonts, amsmath, amssymb, graphics}

% NOTE: bibentry MUST appear before the hyperref or build will fail
\usepackage{bibentry}
\nobibliography*
\usepackage[square,sort,comma,numbers]{natbib}
  
\usepackage{float}
\usepackage[
    hyperindex=true,		% Make numbers of index links as well
   	backref=page, 		% Provide page listing where refs occur in the bibliography
	%breaklinks=true,
    colorlinks,%
    citecolor=green,%
    filecolor=blue,%
    linkcolor=red,%
    urlcolor=red, 
]{hyperref}

\usepackage{dsfont}
%%%% USEPACKAGES for MACROS %%%%%
\usepackage{algpseudocode}
\usepackage[chapter]{algorithm}
\usepackage{caption}
\usepackage{subcaption}
\usepackage{url}

\usepackage{array}
\usepackage{arydshln}
\usepackage{multirow}
\usepackage{multicol}
\usepackage[section]{placeins}

\newcommand{\toprule}[0]{\hline}
\newcommand{\midrule}[0]{\hline\hline}
\newcommand{\bottomrule}[0]{\hline}

\DeclareSymbolFont{AMSb}{U}{msb}{m}{n}
\DeclareMathSymbol{\N}{\mathbin}{AMSb}{"4E}
\DeclareMathSymbol{\Z}{\mathbin}{AMSb}{"5A}
\DeclareMathSymbol{\R}{\mathbin}{AMSb}{"52}
\DeclareMathSymbol{\Q}{\mathbin}{AMSb}{"51}
\DeclareMathSymbol{\PP}{\mathbin}{AMSb}{"50}
\DeclareMathSymbol{\I}{\mathbin}{AMSb}{"49}
%\DeclareMathSymbol{\C}{\mathbin}{AMSb}{"43}

%%%%%% VECTOR NORM: %%%%%%%
\newcommand{\vectornorm}[1]{\left|\left|#1\right|\right|}
\newcommand{\vnorm}[1]{\left|\left|#1\right|\right|}
\newcommand{\by}[0]{\times}
\newcommand{\vect}[1]{\mathbf{#1}}
%\newcommand{\mat}[1]{\mathbf{#1}} 

%\renewcommand{\vec}[1]{ \textbf{#1} }
%%%%%%%%%%%%%%%%%%%%%%

%%%%%%% THM, COR, DEF %%%%%%%
%\newtheorem{theorem}{Theorem}[section]
%\newtheorem{lemma}[theorem]{Lemma}
%\newtheorem{proposition}[theorem]{Proposition}
%\newtheorem{corollary}[theorem]{Corollary}
%\newenvironment{proof}[1][Proof]{\begin{trivlist}
%\item[\hskip \labelsep {\bfseries #1}]}{\end{trivlist}}
%\newenvironment{definition}[1][Definition]{\begin{trivlist}
%\item[\hskip \labelsep {\bfseries #1}]}{\end{trivlist}}
%\newenvironment{example}[1][Example]{\begin{trivlist}
%\item[\hskip \labelsep {\bfseries #1}]}{\end{trivlist}}
%\newenvironment{remark}[1][Remark]{\begin{trivlist}
%\item[\hskip \labelsep {\bfseries #1}]}{\end{trivlist}}
%\newcommand{\qed}{\nobreak \ifvmode \relax \else
%      \ifdim\lastskip<1.5em \hskip-\lastskip
%      \hskip1.5em plus0em minus0.5em \fi \nobreak
%      \vrule height0.75em width0.5em depth0.25em\fi}
%%%%%%%%%%%%%%%%%%%%%%


% \DeclareMathOperator{\Sample}{Sample}
%\let\vaccent=\v % rename builtin command \v{} to \vaccent{}
%\renewcommand{\vec}[1]{\ensuremath{\mathbf{#1}}} % for vectors
\newcommand{\gv}[1]{\ensuremath{\mbox{\boldmath$ #1 $}}} 
% for vectors of Greek letters
\newcommand{\uv}[1]{\ensuremath{\mathbf{\hat{#1}}}} % for unit vector
\newcommand{\abs}[1]{\left| #1 \right|} % for absolute value
\newcommand{\avg}[1]{\left< #1 \right>} % for average
\let\underdot=\d % rename builtin command \d{} to \underdot{}
\renewcommand{\d}[2]{\frac{d #1}{d #2}} % for derivatives
\newcommand{\dd}[2]{\frac{d^2 #1}{d #2^2}} % for double derivatives
\newcommand{\pd}[2]{\frac{\partial #1}{\partial #2}} 
% for partial derivatives
\newcommand{\pdd}[2]{\frac{\partial^2 #1}{\partial #2^2}} 
\newcommand{\pdda}[3]{\frac{\partial^2 #1}{\partial #2 \partial #3}} 
% for double partial derivatives
\newcommand{\pdc}[3]{\left( \frac{\partial #1}{\partial #2}
 \right)_{#3}} % for thermodynamic partial derivatives
\newcommand{\ket}[1]{\left| #1 \right>} % for Dirac bras
\newcommand{\bra}[1]{\left< #1 \right|} % for Dirac kets
\newcommand{\braket}[2]{\left< #1 \vphantom{#2} \right|
 \left. #2 \vphantom{#1} \right>} % for Dirac brackets
\newcommand{\matrixel}[3]{\left< #1 \vphantom{#2#3} \right|
 #2 \left| #3 \vphantom{#1#2} \right>} % for Dirac matrix elements
\newcommand{\grad}[1]{\gv{\nabla} #1} % for gradient
\let\divsymb=\div % rename builtin command \div to \divsymb
\renewcommand{\div}[1]{\gv{\nabla} \cdot #1} % for divergence
\newcommand{\curl}[1]{\gv{\nabla} \times #1} % for curl
\let\baraccent=\= % rename builtin command \= to \baraccent
\renewcommand{\=}[1]{\stackrel{#1}{=}} % for putting numbers above =
\newcommand{\diffop}[1]{\mathcal{L}#1}
\newcommand{\boundop}[1]{\mathcal{B}#1}
\newcommand{\rvec}[0]{{\bf r}}

\newcommand{\Interior}[0]{\Omega}
\newcommand{\domain}[0]{\Omega}
\newcommand{\Boundary}[0]{\partial \Omega}
%\newcommand{\Boundary}[0]{\Gamma}

\newcommand{\on}[1]{\hskip1.5em \textrm{ on } #1}

\newcommand{\gemm}{\texttt{GEMM}}
\newcommand{\trmm}{\texttt{TRMM}}
\newcommand{\gesvd}{\texttt{GESVD}}
\newcommand{\geqrf}{\texttt{GEQRF}}


\newcommand{\minitab}[2][l]{\begin{tabular}{#1}#2\end{tabular}}
\newcommand{\comm}[1]{\textcolor{red}{\textit{#1}}}

\newcommand{\nfrac}[2]{
\nicefrac{#1}{#2}
%\frac{#1}{#2}
}

\usepackage{xparse}


%%%%%%%%%%%%%%%
% Show a Author's Note
% USAGE: 
% \incomplete[Optional footnote message to further clarify note]{The text which is currently not finished}
\DeclareDocumentCommand \incomplete{ o m }
{%
\IfNoValueTF {#1}
{\textcolor{red}{Incomplete: \ul{#2}}} 
{\textcolor{red}{Incomplete: \ul{#2}}\footnote{Comment: #1}}%
}
%%%%%%%%%%%%%%%



%%%%%%%%%%%%%%%
% Show a Author's Note
% USAGE: 
% \authnote[Optional footnote message to further clarify note]{The note to your readers}
\DeclareDocumentCommand \authnote { o m }
{%
\IfNoValueTF {#1}
{\textcolor{blue}{Author's Note: \ul{#2}}} 
{\textcolor{blue}{Author's Note: \ul{#2}}\footnote{Comment: #1}}%
}
%%%%%%%%%%%%%%%



%%%%%%%%%%%%%%%
% Strike out text that doesn't belong in the paper
% USAGE: 
% \strike[Optional footnote to state why it doesn't belong]{Text to strike out}
\DeclareDocumentCommand \strike { o m }
{%
\setstcolor{red}
\IfNoValueTF {#1}
{\textcolor{Gray}{\st{#2}}} 
{\textcolor{Gray}{\st{#2}}\footnote{Comment: #1}}%
}
%%%%%%%%%%%%%%%



%
% colors to show the corrections
\newcommand{\red}[1]{\textbf{\textcolor{red}{#1}}}
\newcommand{\blue}[1]{\textbf{\textcolor{blue}{#1}}}
\newcommand{\cyan}[1]{\textbf{\textcolor{cyan}{#1}}}
\newcommand{\green}[1]{\textbf{\textcolor{green}{#1}}}
\newcommand{\magenta}[1]{\textbf{\textcolor{magenta}{#1}}}
\newcommand{\orange}[1]{\textbf{\textcolor{orange}{#1}}}
%%%%%%%%%% DK DK
% comments between authors
\newcommand{\toall}[1]{\textbf{\green{@@@ All: #1 @@@}}}
\newcommand{\toevan}[1]{\textbf{\red{*** Evan: #1 ***}}}
%\newcommand{\toevan}[1]{}  % USE FOR FINAL VERSION
\newcommand{\toe}[1]{\textbf{\red{*** Evan: #1 ***}}}
\newcommand{\tog}[1]{\textbf{\blue{*** Gordon: #1 ***}}}
%\newcommand{\togordon}[1]{\textbf{\blue{*** Gordon: #1 ***}}}
\renewcommand{\ge}[3]{{\textcolor{blue}{*** \textbf{Gordon:}\strike{#1} #2 ***}}\red{(#3)}}
\renewcommand{\ge}[3]{{\textcolor{blue}{#2}}}
\renewcommand{\ge}[3]{{\textcolor{red}{#2}}}
\newcommand{\eb}[3]{{\textcolor{red}{*** \textbf{Evan:}\strike{#1} #2 ***}}\red{(#3)}}
\renewcommand{\eb}[3]{{{\textcolor{red}{#2}}}}
%\def\ge#1#2#3{}{\textbf{\blue{*** Gordon: #2 ***}}}{(#3)}
\newcommand{\gee}[1]{{\bf{\blue{{\em #1}}}}}
\newcommand{\old}[1]{}
\newcommand{\del}[1]{***#1*** }



\input{macros/misc_mac.tex}
\newcommand{\mathsym}[1]{{}}
\newcommand{\unicode}[1]{{}}
\newcommand{\ep}{\epsilon}
\newcommand{\vx}{\mathbf{x}}


\usepackage{tabularx} 
\newcolumntype{C}{>{\centering\arraybackslash}b{1in}}
\newcolumntype{L}{>{\flushleft\arraybackslash}b{1.5in}}
\newcolumntype{R}{>{\flushright\arraybackslash}b{1.5in}}
\newcolumntype{D}{>{\flushright\arraybackslash}b{2.0in}}
\newcolumntype{E}{>{\flushright\arraybackslash}b{1.0in}}


 


\usepackage{xcolor}
% Sepia
\definecolor{myBGcolor}{HTML}{F6F0D6}
\definecolor{myTextcolor}{HTML}{4F452C}
% Dark
%\definecolor{myBGcolor}{HTML}{3E3535}
%\definecolor{myTextcolor}{HTML}{CFECEC}
%\color{myTextcolor}
\pagecolor{myBGcolor}
 
%
%\begin{document}
%\fi


\chapter{Projected Weights on the Sphere} 
\label{app:indirect_weights}

Operating on the sphere requires constrained operators described in Section~\ref{sec:projected_grad}:
\begin{align}
\mathbf{P} \cdot \nabla \phi_{k}(r(\vx)) & = \mathbf{P} \cdot \frac{(\vx-\vx_{k})}{r(\vx)} \d{\phi_{k}(r(\vx))}{r(\vx)}  \nonumber \\
& = -\mathbf{P} \cdot \vx_{k}\frac{1}{r(\vx)} \d{\phi_{k}(r(\vx))}{r(\vx)}  \nonumber \\ %\label{eq:xsfc_negative}
& = \begin{pmatrix} x \vx^{T} \vx_{k} - x_{k} \\  y \vx^{T} \vx_{k} - y_{k} \\  z \vx^{T} \vx_{k} - z_{k} \end{pmatrix} \frac{1}{r(\vx)} \d{\phi(r(\vx))}{r} \label{eq:sfc_gradient_operator}.
\end{align}
where 
\begin{align}
P = I - \mathbf{x} \mathbf{x}^T =  \begin{pmatrix} 
(1-x_1^2) & -x_1 x_2 & -x_1 x_3 \\
-x_1 x_2 & (1-x_2^2) & -x_2 x_3 \\ 
-x_1 x_3 & -x_2 x_3 & (1-x_3^2) 
\end{pmatrix} 
\label{eq:sfc_project_gradient}
\end{align}
%The operator $\mathbf{I} - \vx \vx^{T}$ for $\vx = (x,y,z)$ projects a vector onto the plane tangent to the unit sphere at $(x,y,z)$. Therefore, Equation~\ref{eq:sfc_gradient_operator} gives the projection of the gradient operator at $\vx_{k}$ onto the plane tangent to $\vx$. 

Here we investigate the difference between DMs constructed using Equation~\ref{eq:sfc_gradient_operator} on the RHS of the RBF-FD weight system (Equation~\ref{eq:rbffd_weight_system}) versus DMs composed from combinations of the standard Cartesian $\grad$ operator following Equation~\ref{eq:sfc_project_gradient}. We label the former case as \emph{direct weights} and the latter as \emph{indirect weights}. 


\subsubsection{Direct Weights} 

Following \cite{FlyerLehto11}, \ref{eq:sfc_gradient_operator} takes the following form when adapted to RBF-FD:  
\begin{equation}
[ \mathbf{p}_{x} \cdot \nabla{f(\vx)}] |_{\vx = \vx_{c}} = \sum_{k=1}^{n} c_{k} \underbrace{\left[ x_{c} \vx_{c}^{T} \vx_{k} - x_{k} \right] \frac{1}{r} \d{\phi(r(x_{c}))}{r}}_{B_{c,k}^{\mathbf{p}_{x}}}. 
\label{eq:xsfc_operator_flyer_et_al}
\end{equation}
and so forth for the $\mathbf{p}_{y} \cdot \nabla, \mathbf{p}_{z}  \cdot \nabla$ operators, where $\vx_{c}$ is the stencil center and $\vx_{k}$ are stencil nodes. To compute RBF-FD weights for the $\mathbf{p}_{x} \cdot \nabla$ operator, the RHS of Equation~\ref{eq:rbffd_weight_system} is filled with elements $B_{c,k}^{\mathbf{p}_{x}}$. We will refer to this method of obtaining the weights as the \emph{direct} method due to the ability to directly compute RBF-FD weights for the operators $\mathbf{P} \cdot \nabla $, and assemble the differentiation matrices $\D_{\mathbf{p_{x}} \cdot \nabla}, \D_{\mathbf{p_{y}} \cdot \nabla}, \D_{\mathbf{p_{z}} \cdot \nabla}$ without the need to compute and/or store other (unneeded) weights.

\subsubsection{Indirect Weights} 

Alternatively, weights can be computed \emph{indirectly} as a weighted combination of existing RBF-FD DMs for the unprojected $\nabla$ operator. Here we assume that differentiation matrices to compute the components of $\nabla$ are readily available in memory: 
$$
\D_{\nabla} = \begin{pmatrix} \D_{x} \\ \D_{y} \\ \D_{z} \end{pmatrix},
$$
where each matrix contains weights computed with the operators from Section~\ref{sec:rbffd_grad_weights} applied to the RHS of Equation~\ref{eq:rbffd_weight_system}.  

The differentiation matrices for $\mathbf{P} \cdot \nabla$ can then be assembled as a weighted combination of the unprojected gradient matrices: 
\begin{equation}
\D_{\mathbf{P} \cdot \nabla} = \begin{pmatrix} \D_{\mathbf{p_{x} \cdot \nabla}} \\  \D_{\mathbf{p_{y}\cdot \nabla}} \\  \D_{\mathbf{p_{z}\cdot \nabla}} \end{pmatrix} = \begin{pmatrix} 
diag(1-X^{2}) \D_{x} - diag(XY) \D_{y} - diag(XZ) \D_{z} \\
- diag(XY)\D_{x} + diag(1-Y^{2}) \D_{y} - diag(YZ) \D_{z} \\
- diag(XZ)\D_{x} - diag(YZ) \D_{y} + diag(1-Y^{2}) \D_{z} 
\end{pmatrix}
\label{eq:indirect_project}
\end{equation}
 where $X = \{x_{c,i}\}_{i=1}^{N}$, $Y = \{y_{c,i}\}_{i=1}^{N}$, $Z = \{z_{c,i}\}_{i=1}^{N}$ are the individual component values of the stencil centers $\{\vx_{c,i}\}_{i=1}^{N}$ respectively (i.e., $x_c = (x_c, y_c, z_c)$). 

This concept equates to classical Finite Differences where for example, the standard 5-point finite difference formula in 2-D is expressed a weighted combination of coefficients for 1-D differences. 

Indirectly computing weights is of interest for a few reasons: 
\begin{itemize}
\item \emph{The potential for memory conservation.} For example, consider a coupled PDE that requires four operators: $\D_x$, $\D_y$, $\D_z$, $\D_{\Laplacian{}}$. A single DM on $N=10^6$ nodes with stencil size $n=101$ requires roughly $1.6$ GB of memory in double precision. Indirectly computing $\D_{\Laplacian{}}$ based on the three other DMs can save a large chunk of memory. For a GPU or other accelerator (e.g., Intel Phi) with only 6 GB of global memory, the savings can be compelling.
\item \emph{The ability to compose an operator with weights loaded from disk.} Possible use cases include distributing a generated grid online with precomputed RBF-FD weights for the Cartesian gradient. New investigations can reuse the grid and indirectly compose consistent operators for their problem.
\item \emph{Simplification in operator construction.} One often finds it difficult to directly/analytically apply high powered differential operators to RBFs. In those cases, the discretized operators can be easily composed indirectly via one or more matrix additions and/or multiplications. 
\end{itemize}


\section{Comparison of Direct and Indirect Weights} 

%TODO: code in ~/Karen/rbffd_prototypes/matlab/stokes/test_weights

To verify functionality of direct versus indirect approaches, DMs are assembled with each method for the MD-node sets ranging between $N = 121$ and $N=27556$. A manufactured solution is constructed on the sphere as shown in Figure~\ref{fig:direct_vs_indirect_manufactured_solution}. Starting with the function $f(x) = Y_3^2 \sin(20x)$ in Figure~\ref{fig:direct_vs_indirect_manufactured_base}, we seek approximations to the constrained derivative in $x$, $\mathbf{p}_{x} \cdot \nabla f(x)$ (Figure~\ref{fig:direct_vs_indirect_manufactured_xsfc}), and the constrained Laplacian, $\LaplaceBeltrami{f(x)}$ (Figure~\ref{fig:direct_vs_indirect_manufactured_lsfc}). 

The test case, $f(x)$, checks convergence of approximation on a scalar field with oscillations, which are intentionally difficult to capture without sufficient grid resolution. It also exercises RBF-FD's ability to capture derivatives constrained to the sphere while operating in Cartesian coordinates. 

Since each DM results in approximation error, the expectation is that errors will compound when indirectly assembling DMs.  
Indirect weights for $\mathbf{p}_{x} \cdot \nabla$ are assembled following the first row of the RHS of Equation~\ref{eq:indirect_project} with $\D_x$, $\D_y$, and $\D_z$ given as the components of the Cartesian gradient. In the case of the $\LaplaceBeltrami{}$ operator, the indirect weights are constructed as $(\mathbf{P} \cdot \nabla) \cdot (\mathbf{P} \cdot \nabla)$ to consider the worst case when errors multiply. The second order constrained operator applies Equation~\ref{eq:indirect_project} to the components of the Cartesian gradient, squares the resulting components of $\D_{\mathbf{P} \cdot \nabla}$ and adds components to get the final matrix with 9 matrix-vector multiplies, 8 matrix additions and 3 matrix multiplies. We arrive at the same results if Equation~\ref{eq:indirect_project} is applied both to the Cartesian gradient and then again to the resulting DMs for $\D_{\mathbf{P} \cdot \nabla}$, or if intermediate DMs are multiplied against function values and the resulting derivative approximations combined in place of the DMs (i.e., the final DM is not explicitly assembled). 

\begin{figure}
\begin{center}
	\centering
	\begin{subfigure}[t]{0.48\textwidth}
	\centering
	\includegraphics[width=1.0\textwidth]{../figures/appendices/direct_vs_indirect_weights/compare_weight_generation/xsfc_vs_xsfc_alt_on_sph32_times_sine_20x/sph32_times_sin20x-eps-converted-to.pdf}
	\caption{Test function:  \\ $Y_{3}^{2} \sin(20 x) $.  }
		\label{fig:direct_vs_indirect_manufactured_base}
	\end{subfigure}
	
	\begin{subfigure}[t]{0.48\textwidth}
		\centering
	\includegraphics[width=1.0\textwidth]{../figures/appendices/direct_vs_indirect_weights/compare_weight_generation/xsfc_vs_xsfc_alt_on_sph32_times_sine_20x/pdx_sph32_times_sin20x-eps-converted-to.pdf}
	\caption{$\mathbf{p}_{x} \cdot \nabla ( Y_{3}^{2} \sin(20 x))$ }
			\label{fig:direct_vs_indirect_manufactured_xsfc}
	\end{subfigure}
	\begin{subfigure}[t]{0.48\textwidth}
		\centering
	\includegraphics[width=1.0\textwidth]{../figures/appendices/direct_vs_indirect_weights/compare_weight_generation/lsfc_vs_px_grad_dot_px_grad/lsfc_sph32_times_sin20x-eps-converted-to.pdf}
	\caption{Surface Laplacian: \\ $\LaplaceBeltrami ( Y_{3}^{2} \sin(20 x) )$  }
				\label{fig:direct_vs_indirect_manufactured_lsfc}
	\end{subfigure}
	\caption{Test function and its projected derivatives on the surface of the unit sphere. }
				\label{fig:direct_vs_indirect_manufactured_solution}
	\end{center}
\end{figure}

The accuracy of each approximation is measured as the $\ell_2$ relative error,  
$$ \text{relative $\ell_{2}$ error} = \frac{|| f_{approx} - f_{exact} ||_{2} }{ || f_{exact} ||_{2} }, $$ 
where $f_{approx}$ is the approximate derivatives and $f_{exact}$ is the manufactured solution. Figure~\ref{fig:direct_vs_indirect_relative_error} provides the relative $\ell_{2}$ errors for RBF-FD stencils of size $n=17, 31, 50,$ and $101$. The choice of $\epsilon$ in each case follows the parameters provided in Table~\ref{tbl:vortex_hv_params}. RBF-FD weights are calculated using the RBF-Direct method, and no hyperviscosity. 

The relative error for both direct (Figure~\ref{fig:direct_vs_indirect_relative_error_xsfc_direct}) and indirect approximations (Figure~\ref{fig:direct_vs_indirect_relative_error_xsfc_indirect}) of $\mathbf{p}_{x} \cdot \nabla$ are nearly identical. Although the grid is under-resolved and incapable of capturing oscillations for the first few test cases, the results converge well past $N > 400$. In the case of $\LaplaceBeltrami{}$, the relative $\ell_2$ error for direct (Figure~\ref{fig:direct_vs_indirect_relative_error_lsfc_direct}) versus indirect weights (Figure~\ref{fig:direct_vs_indirect_relative_error_lsfc_indirect}) behaves similarly for both cases when $N > 800$ nodes. However, the direct method starts converging earlier than the indirect method and maintains one or two digits higher precision overall. %This leads to the understanding that while the indirect approach can cut corners, it can require two to four times the number of nodes for the same accuracy on $\LaplaceBeltrami{}$. 

\begin{figure}
	\centering
	\begin{subfigure}[t]{0.48\textwidth}
	\includegraphics[width=1.0\textwidth]{../figures/appendices/direct_vs_indirect_weights/compare_weight_generation/xsfc_vs_xsfc_alt_on_sph32_times_sine_20x/direct_rel_l2_error-eps-converted-to.pdf}
	\caption{$\mathbf{p}_{x} \cdot \nabla ( Y_{3}^{2} \sin(20 x))$, direct weights}
			\label{fig:direct_vs_indirect_relative_error_xsfc_direct}
	\end{subfigure}
	\begin{subfigure}[t]{0.48\textwidth}
	\includegraphics[width=1.0\textwidth]{../figures/appendices/direct_vs_indirect_weights/compare_weight_generation/xsfc_vs_xsfc_alt_on_sph32_times_sine_20x/indirect_rel_l2_error-eps-converted-to.pdf}
	\caption{$\mathbf{p}_{x} \cdot \nabla ( Y_{3}^{2} \sin(20 x))$, indirect weights}
		\label{fig:direct_vs_indirect_relative_error_xsfc_indirect}
	\end{subfigure}
	\begin{subfigure}[t]{0.48\textwidth}
	\includegraphics[width=1.0\textwidth]{../figures/appendices/direct_vs_indirect_weights/compare_weight_generation/lsfc_vs_px_grad_dot_px_grad/direct_rel_l2_error-eps-converted-to.pdf}
	\caption{$\LaplaceBeltrami ( Y_{3}^{2} \sin(20 x) )$, direct weights}
			\label{fig:direct_vs_indirect_relative_error_lsfc_direct}
    \end{subfigure}
	\begin{subfigure}[t]{0.48\textwidth}
	\includegraphics[width=1.0\textwidth]{../figures/appendices/direct_vs_indirect_weights/compare_weight_generation/lsfc_vs_px_grad_dot_px_grad/indirect_rel_l2_error-eps-converted-to.pdf}
	\caption{$\LaplaceBeltrami ( Y_{3}^{2} \sin(20 x) )$, indirect weights}
		\label{fig:direct_vs_indirect_relative_error_lsfc_indirect}
    \end{subfigure}
	\caption{Relative $\ell_{2}$ error in differentiation.}
	\label{fig:direct_vs_indirect_relative_error}
\end{figure}

The plateau on convergence in Figure~\ref{fig:direct_vs_indirect_relative_error} is an indicator of limitations in the RBF-Direct method and the choice of $\epsilon$ (see discussion of \emph{saturation error} in \cite{FlyerFornberg11,Fasshauer2007}). Future investigation may consider the use of an alternative weight method like RBF-GA (\cite{Fornberg2012}) to maintain the rate of convergence. RBF-GA utilizes a change of basis on both the LHS and RHS of Equation~\ref{eq:rbffd_weight_system}, which implies the need to apply derivatives to both gaussian RBFs and incomplete gamma functions. With accurate RBF-GA weights, indirectly assembled DMs could be a viable alternative to the pain of deriving the modified RHS. 

Figure~\ref{fig:direct_vs_indirect_unsigned_diff} considers the unsigned difference in relative $\ell_2$ errors: $| \text{($\ell_{2}$)}_{\text{direct}} - \text{($\ell_{2}$)}_{\text{indirect}} |$. The results confirm that indirect weight approximations for the constrained first derivative are as good (or bad) as using direct weights. The absolute value of the differences shows over six digits of agreement in accuracy for the first derivative. In the case of $\LaplaceBeltrami{}$, the indirect approach is significantly less accurate with only one or two digits of agreement in many cases. Fortunately, as $N$ grows the indirect and direct approximations converge. 

In conclusion, the results in Figures~\ref{fig:direct_vs_indirect_relative_error} and \ref{fig:direct_vs_indirect_unsigned_diff} confirm that indirectly assembling RBF-FD weights can function well in terms of convergence, with similar rates as directly computed weights. Test results for the $\LaplaceBeltrami{}$ operator show that higher order operators assembled from simple Cartesian gradient DMs may lose an order of magnitude accuracy in approximation, and require twice as many grid points to recover the solution produced by directly computed RBF-FD weights. 


%\begin{figure}
%	\centering
%	\begin{subfigure}[t]{0.48\textwidth}
%	\includegraphics[width=1.0\textwidth]{../figures/appendices/direct_vs_indirect_weights/compare_weight_generation/xsfc_vs_xsfc_alt_on_sph32_times_sine_20x/diff_of_rel_l2_errors-eps-converted-to.pdf}
%	\caption{$\mathbf{p}_{x} \cdot \nabla ( Y_{3}^{2} \sin(20 x))$}
%	\end{subfigure}
%		\begin{subfigure}[t]{0.48\textwidth}
%	\centering
%	\includegraphics[width=1.0\textwidth]{../figures/appendices/direct_vs_indirect_weights/compare_weight_generation/lsfc_vs_px_grad_dot_px_grad/diff_of_rel_l2_errors-eps-converted-to.pdf}
%	\caption{$\LaplaceBeltrami ( Y_{3}^{2} \sin(20 x) )$ }
%	\end{subfigure}
%	\caption{Signed differences of relative $\ell_{2}$ errors in differentiation between Direct and Indirect weights. The sign indicates on the error indicates the higher of the two weight approaches.}
%    \label{fig:direct_vs_indirect_signed_diff}
%\end{figure}


\begin{figure}
	\centering
		\begin{subfigure}[t]{0.48\textwidth}
		\centering
	\includegraphics[width=1.0\textwidth]{../figures/appendices/direct_vs_indirect_weights/compare_weight_generation/xsfc_vs_xsfc_alt_on_sph32_times_sine_20x/abs_diff_of_rel_l2_errors-eps-converted-to.pdf}
	\caption{$\mathbf{p}_{x} \cdot \nabla ( Y_{3}^{2} \sin(20 x))$}
	\end{subfigure}
	\begin{subfigure}[t]{0.48\textwidth}
	\centering
	\includegraphics[width=1.0\textwidth]{../figures/appendices/direct_vs_indirect_weights/compare_weight_generation/lsfc_vs_px_grad_dot_px_grad/abs_diff_of_rel_l2_errors-eps-converted-to.pdf}
	\caption{$\LaplaceBeltrami ( Y_{3}^{2} \sin(20 x) )$}
	\end{subfigure}
		\caption{Unsigned differences of relative $\ell_{2}$ differentiation errors for Direct and Indirect weights.}
		\label{fig:direct_vs_indirect_unsigned_diff}
\end{figure}



%\ifstandalone
%\bibliographystyle{plain}
%\bibliography{merged_references}
%\end{document}
%\else
%\expandafter\endinput
%\fi
