\chapter{Stokes Revisited}

\begin{align}
\div{[\eta(\grad{\vu} + (\grad{\vu})^T)]} + Ra T \hat{r} & = \grad{p} \label{eq:stokes_momentum_original} \\
\div{\eta(\grad{\vu}} + \div{(\grad{\vu})^T)} + Ra T \hat{r} & = \grad{p} \\
\grad{\eta} \cdot \grad{\vu} + \eta \div{\grad{\vu}} + \div{(\grad{\vu})^T)} + Ra T \hat{r} & = \grad{p} \\
\grad{\eta} \cdot \begin{pmatrix} \pd{u}{x} & \pd{u}{y} & \pd{u}{z} \\ \pd{v}{x} & \pd{v}{y} & \pd{v}{z} \\ \pd{w}{x} & \pd{w}{y} & \pd{w}{z} \end{pmatrix} + \eta \div{\begin{pmatrix} \pd{u}{x} & \pd{u}{y} & \pd{u}{z} \\ \pd{v}{x} & \pd{v}{y} & \pd{v}{z} \\ \pd{w}{x} & \pd{w}{y} & \pd{w}{z} \end{pmatrix}} + \div{\begin{pmatrix} \pd{u}{x} & \pd{v}{y} & \pd{w}{z} \\ \pd{u}{x} & \pd{v}{y} & \pd{w}{z} \\ \pd{u}{x} & \pd{v}{y} & \pd{w}{z} \end{pmatrix}} + Ra T \hat{r} & = \grad{p} \\
\begin{pmatrix} \pd{\eta}{x} \pd{u}{x} & \pd{\eta}{x} \pd{u}{y} & \pd{\eta}{x} \pd{u}{z} \\ \pd{\eta}{y} \pd{v}{x} & \pd{\eta}{y} \pd{v}{y} & \pd{\eta}{y} \pd{v}{z} \\ \pd{\eta}{z} \pd{w}{x} & \pd{\eta}{z} \pd{w}{y} & \pd{\eta}{z} \pd{w}{z} \end{pmatrix} + \eta \begin{pmatrix} \pdd{u}{x} & \pd{u}{y \partial x} & \pd{u}{z \partial x} \\ \pd{v}{x \partial y} & \pdd{v}{y} & \pd{v}{z \partial y} \\ \pd{w}{x \partial z} & \pd{w}{y \partial z} & \pdd{w}{z} \end{pmatrix} + \div{\begin{pmatrix} \div{\vu} \\ \div{\vu} \\ \div{\vu} \end{pmatrix}} + Ra T \hat{r} & = \grad{p} \\
\begin{pmatrix} \pd{\eta}{x} \pd{u}{x} & \pd{\eta}{x} \pd{u}{y} & \pd{\eta}{x} \pd{u}{z} \\ \pd{\eta}{y} \pd{v}{x} & \pd{\eta}{y} \pd{v}{y} & \pd{\eta}{y} \pd{v}{z} \\ \pd{\eta}{z} \pd{w}{x} & \pd{\eta}{z} \pd{w}{y} & \pd{\eta}{z} \pd{w}{z} \end{pmatrix} + \eta \begin{pmatrix} \pdd{u}{x} & \pd{u}{y \partial x} & \pd{u}{z \partial x} \\ \pd{v}{x \partial y} & \pdd{v}{y} & \pd{v}{z \partial y} \\ \pd{w}{x \partial z} & \pd{w}{y \partial z} & \pdd{w}{z} \end{pmatrix} + \div{\begin{pmatrix} 0 \\ 0 \\ 0 \end{pmatrix}} + Ra T \hat{r} & = \grad{p} \\
\begin{pmatrix} \pd{\eta}{x} \pd{u}{x} & \pd{\eta}{x} \pd{u}{y} & \pd{\eta}{x} \pd{u}{z} \\ \pd{\eta}{y} \pd{v}{x} & \pd{\eta}{y} \pd{v}{y} & \pd{\eta}{y} \pd{v}{z} \\ \pd{\eta}{z} \pd{w}{x} & \pd{\eta}{z} \pd{w}{y} & \pd{\eta}{z} \pd{w}{z} \end{pmatrix} + \eta \begin{pmatrix} \pdd{u}{x} & \pd{u}{y \partial x} & \pd{u}{z \partial x} \\ \pd{v}{x \partial y} & \pdd{v}{y} & \pd{v}{z \partial y} \\ \pd{w}{x \partial z} & \pd{w}{y \partial z} & \pdd{w}{z} \end{pmatrix} + Ra T \hat{r} & = \grad{p} \\
\begin{pmatrix} 0 \pd{u}{x} & 0 \pd{u}{y} & 0 \pd{u}{z} \\ 0 \pd{v}{x} & 0 \pd{v}{y} & 0 \pd{v}{z} \\ 0 \pd{w}{x} & 0 \pd{w}{y} & 0 \pd{w}{z} \end{pmatrix} + \eta \begin{pmatrix} \pdd{u}{x} & \pd{u}{y \partial x} & \pd{u}{z \partial x} \\ \pd{v}{x \partial y} & \pdd{v}{y} & \pd{v}{z \partial y} \\ \pd{w}{x \partial z} & \pd{w}{y \partial z} & \pdd{w}{z} \end{pmatrix} + Ra T \hat{r} & = \grad{p} \\
\eta \begin{pmatrix} \pdd{u}{x} & \pd{u}{y \partial x} & \pd{u}{z \partial x} \\ \pd{v}{x \partial y} & \pdd{v}{y} & \pd{v}{z \partial y} \\ \pd{w}{x \partial z} & \pd{w}{y \partial z} & \pdd{w}{z} \end{pmatrix} + Ra T \hat{r} & = \grad{p}
\end{align}
The off diagonal blocks are simple to include in code because it is the product $P\cdot \pd{}{x} * P \cdot \pd{}{y}$ (remember they are symmetric so $xy = yx$). Also in the case of the interleaved values we have extra nonzeros bundled together to benefit from cache effects. When $\eta$ is a non-constant function then we have extra work in the assembly to accumulate blocks. Recall that the use of $Q$ arises due to the fact that $div(curl(F)) = 0$, so $Q$ essentially gets the curl of an input vector field and we use the resulting $\curl{F}$ as the manufactured divergence free field. 
