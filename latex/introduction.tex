\makeatletter
\@ifundefined{standalonetrue}{\newif\ifstandalone}{}
\@ifundefined{section}{\standalonetrue}{\standalonefalse}
\makeatother
\ifstandalone
\documentclass[11pt]{report}
\usepackage{textcase}
\usepackage[pdftex]{graphicx}
%\usepackage{hyperref}
%\hypersetup{breaklinks=true}


% Added packages
\usepackage[usenames]{color}
\usepackage{amsfonts, amsmath, amssymb, graphics}

% NOTE: bibentry MUST appear before the hyperref or build will fail
\usepackage{bibentry}
\nobibliography*
\usepackage[square,sort,comma,numbers]{natbib}
  
\usepackage{float}
\usepackage[
    hyperindex=true,		% Make numbers of index links as well
   	backref=page, 		% Provide page listing where refs occur in the bibliography
	%breaklinks=true,
    colorlinks,%
    citecolor=green,%
    filecolor=blue,%
    linkcolor=red,%
    urlcolor=red, 
]{hyperref}

\usepackage{dsfont}
%%%% USEPACKAGES for MACROS %%%%%
\usepackage{algpseudocode}
\usepackage[chapter]{algorithm}
\usepackage{caption}
\usepackage{subcaption}
\usepackage{url}

\usepackage{array}
\usepackage{arydshln}
\usepackage{multirow}
\usepackage{multicol}
\usepackage[section]{placeins}

\newcommand{\toprule}[0]{\hline}
\newcommand{\midrule}[0]{\hline\hline}
\newcommand{\bottomrule}[0]{\hline}

\DeclareSymbolFont{AMSb}{U}{msb}{m}{n}
\DeclareMathSymbol{\N}{\mathbin}{AMSb}{"4E}
\DeclareMathSymbol{\Z}{\mathbin}{AMSb}{"5A}
\DeclareMathSymbol{\R}{\mathbin}{AMSb}{"52}
\DeclareMathSymbol{\Q}{\mathbin}{AMSb}{"51}
\DeclareMathSymbol{\PP}{\mathbin}{AMSb}{"50}
\DeclareMathSymbol{\I}{\mathbin}{AMSb}{"49}
%\DeclareMathSymbol{\C}{\mathbin}{AMSb}{"43}

%%%%%% VECTOR NORM: %%%%%%%
\newcommand{\vectornorm}[1]{\left|\left|#1\right|\right|}
\newcommand{\vnorm}[1]{\left|\left|#1\right|\right|}
\newcommand{\by}[0]{\times}
\newcommand{\vect}[1]{\mathbf{#1}}
%\newcommand{\mat}[1]{\mathbf{#1}} 

%\renewcommand{\vec}[1]{ \textbf{#1} }
%%%%%%%%%%%%%%%%%%%%%%

%%%%%%% THM, COR, DEF %%%%%%%
%\newtheorem{theorem}{Theorem}[section]
%\newtheorem{lemma}[theorem]{Lemma}
%\newtheorem{proposition}[theorem]{Proposition}
%\newtheorem{corollary}[theorem]{Corollary}
%\newenvironment{proof}[1][Proof]{\begin{trivlist}
%\item[\hskip \labelsep {\bfseries #1}]}{\end{trivlist}}
%\newenvironment{definition}[1][Definition]{\begin{trivlist}
%\item[\hskip \labelsep {\bfseries #1}]}{\end{trivlist}}
%\newenvironment{example}[1][Example]{\begin{trivlist}
%\item[\hskip \labelsep {\bfseries #1}]}{\end{trivlist}}
%\newenvironment{remark}[1][Remark]{\begin{trivlist}
%\item[\hskip \labelsep {\bfseries #1}]}{\end{trivlist}}
%\newcommand{\qed}{\nobreak \ifvmode \relax \else
%      \ifdim\lastskip<1.5em \hskip-\lastskip
%      \hskip1.5em plus0em minus0.5em \fi \nobreak
%      \vrule height0.75em width0.5em depth0.25em\fi}
%%%%%%%%%%%%%%%%%%%%%%


% \DeclareMathOperator{\Sample}{Sample}
%\let\vaccent=\v % rename builtin command \v{} to \vaccent{}
%\renewcommand{\vec}[1]{\ensuremath{\mathbf{#1}}} % for vectors
\newcommand{\gv}[1]{\ensuremath{\mbox{\boldmath$ #1 $}}} 
% for vectors of Greek letters
\newcommand{\uv}[1]{\ensuremath{\mathbf{\hat{#1}}}} % for unit vector
\newcommand{\abs}[1]{\left| #1 \right|} % for absolute value
\newcommand{\avg}[1]{\left< #1 \right>} % for average
\let\underdot=\d % rename builtin command \d{} to \underdot{}
\renewcommand{\d}[2]{\frac{d #1}{d #2}} % for derivatives
\newcommand{\dd}[2]{\frac{d^2 #1}{d #2^2}} % for double derivatives
\newcommand{\pd}[2]{\frac{\partial #1}{\partial #2}} 
% for partial derivatives
\newcommand{\pdd}[2]{\frac{\partial^2 #1}{\partial #2^2}} 
\newcommand{\pdda}[3]{\frac{\partial^2 #1}{\partial #2 \partial #3}} 
% for double partial derivatives
\newcommand{\pdc}[3]{\left( \frac{\partial #1}{\partial #2}
 \right)_{#3}} % for thermodynamic partial derivatives
\newcommand{\ket}[1]{\left| #1 \right>} % for Dirac bras
\newcommand{\bra}[1]{\left< #1 \right|} % for Dirac kets
\newcommand{\braket}[2]{\left< #1 \vphantom{#2} \right|
 \left. #2 \vphantom{#1} \right>} % for Dirac brackets
\newcommand{\matrixel}[3]{\left< #1 \vphantom{#2#3} \right|
 #2 \left| #3 \vphantom{#1#2} \right>} % for Dirac matrix elements
\newcommand{\grad}[1]{\gv{\nabla} #1} % for gradient
\let\divsymb=\div % rename builtin command \div to \divsymb
\renewcommand{\div}[1]{\gv{\nabla} \cdot #1} % for divergence
\newcommand{\curl}[1]{\gv{\nabla} \times #1} % for curl
\let\baraccent=\= % rename builtin command \= to \baraccent
\renewcommand{\=}[1]{\stackrel{#1}{=}} % for putting numbers above =
\newcommand{\diffop}[1]{\mathcal{L}#1}
\newcommand{\boundop}[1]{\mathcal{B}#1}
\newcommand{\rvec}[0]{{\bf r}}

\newcommand{\Interior}[0]{\Omega}
\newcommand{\domain}[0]{\Omega}
\newcommand{\Boundary}[0]{\partial \Omega}
%\newcommand{\Boundary}[0]{\Gamma}

\newcommand{\on}[1]{\hskip1.5em \textrm{ on } #1}

\newcommand{\gemm}{\texttt{GEMM}}
\newcommand{\trmm}{\texttt{TRMM}}
\newcommand{\gesvd}{\texttt{GESVD}}
\newcommand{\geqrf}{\texttt{GEQRF}}


\newcommand{\minitab}[2][l]{\begin{tabular}{#1}#2\end{tabular}}
\newcommand{\comm}[1]{\textcolor{red}{\textit{#1}}}

\newcommand{\nfrac}[2]{
\nicefrac{#1}{#2}
%\frac{#1}{#2}
}

\usepackage{xparse}


%%%%%%%%%%%%%%%
% Show a Author's Note
% USAGE: 
% \incomplete[Optional footnote message to further clarify note]{The text which is currently not finished}
\DeclareDocumentCommand \incomplete{ o m }
{%
\IfNoValueTF {#1}
{\textcolor{red}{Incomplete: \ul{#2}}} 
{\textcolor{red}{Incomplete: \ul{#2}}\footnote{Comment: #1}}%
}
%%%%%%%%%%%%%%%



%%%%%%%%%%%%%%%
% Show a Author's Note
% USAGE: 
% \authnote[Optional footnote message to further clarify note]{The note to your readers}
\DeclareDocumentCommand \authnote { o m }
{%
\IfNoValueTF {#1}
{\textcolor{blue}{Author's Note: \ul{#2}}} 
{\textcolor{blue}{Author's Note: \ul{#2}}\footnote{Comment: #1}}%
}
%%%%%%%%%%%%%%%



%%%%%%%%%%%%%%%
% Strike out text that doesn't belong in the paper
% USAGE: 
% \strike[Optional footnote to state why it doesn't belong]{Text to strike out}
\DeclareDocumentCommand \strike { o m }
{%
\setstcolor{red}
\IfNoValueTF {#1}
{\textcolor{Gray}{\st{#2}}} 
{\textcolor{Gray}{\st{#2}}\footnote{Comment: #1}}%
}
%%%%%%%%%%%%%%%



%
% colors to show the corrections
\newcommand{\red}[1]{\textbf{\textcolor{red}{#1}}}
\newcommand{\blue}[1]{\textbf{\textcolor{blue}{#1}}}
\newcommand{\cyan}[1]{\textbf{\textcolor{cyan}{#1}}}
\newcommand{\green}[1]{\textbf{\textcolor{green}{#1}}}
\newcommand{\magenta}[1]{\textbf{\textcolor{magenta}{#1}}}
\newcommand{\orange}[1]{\textbf{\textcolor{orange}{#1}}}
%%%%%%%%%% DK DK
% comments between authors
\newcommand{\toall}[1]{\textbf{\green{@@@ All: #1 @@@}}}
\newcommand{\toevan}[1]{\textbf{\red{*** Evan: #1 ***}}}
%\newcommand{\toevan}[1]{}  % USE FOR FINAL VERSION
\newcommand{\toe}[1]{\textbf{\red{*** Evan: #1 ***}}}
\newcommand{\tog}[1]{\textbf{\blue{*** Gordon: #1 ***}}}
%\newcommand{\togordon}[1]{\textbf{\blue{*** Gordon: #1 ***}}}
\renewcommand{\ge}[3]{{\textcolor{blue}{*** \textbf{Gordon:}\strike{#1} #2 ***}}\red{(#3)}}
\renewcommand{\ge}[3]{{\textcolor{blue}{#2}}}
\renewcommand{\ge}[3]{{\textcolor{red}{#2}}}
\newcommand{\eb}[3]{{\textcolor{red}{*** \textbf{Evan:}\strike{#1} #2 ***}}\red{(#3)}}
\renewcommand{\eb}[3]{{{\textcolor{red}{#2}}}}
%\def\ge#1#2#3{}{\textbf{\blue{*** Gordon: #2 ***}}}{(#3)}
\newcommand{\gee}[1]{{\bf{\blue{{\em #1}}}}}
\newcommand{\old}[1]{}
\newcommand{\del}[1]{***#1*** }



\input{macros/misc_mac.tex}
\newcommand{\mathsym}[1]{{}}
\newcommand{\unicode}[1]{{}}
\newcommand{\ep}{\epsilon}
\newcommand{\vx}{\mathbf{x}}


\usepackage{tabularx} 
\newcolumntype{C}{>{\centering\arraybackslash}b{1in}}
\newcolumntype{L}{>{\flushleft\arraybackslash}b{1.5in}}
\newcolumntype{R}{>{\flushright\arraybackslash}b{1.5in}}
\newcolumntype{D}{>{\flushright\arraybackslash}b{2.0in}}
\newcolumntype{E}{>{\flushright\arraybackslash}b{1.0in}}


 


\usepackage{xcolor}
% Sepia
\definecolor{myBGcolor}{HTML}{F6F0D6}
\definecolor{myTextcolor}{HTML}{4F452C}
% Dark
%\definecolor{myBGcolor}{HTML}{3E3535}
%\definecolor{myTextcolor}{HTML}{CFECEC}
%\color{myTextcolor}
\pagecolor{myBGcolor}


\begin{document}
\fi



\chapter{Introduction}
\label{chap:introduction}

A plethora of scientific problems can be expressed as a collection of partial differential equations defined for some domain. In order to solve these 
problems, computational numerical methods are employed on a discretized version of the domain. Traditionally, numerical methods have been \emph{meshed methods}, in that they rely on some underlying grid/lattice to connect discretized points in a well-defined manner. More recently a new class of methods have surfaced, called \emph{meshfree methods}, which discard the requirement for connectivity and operate only on point clouds. Each class of methods offers numerous and, in many ways, complementary benefits. This work focuses on \emph{Radial Basis Function-generated Finite Differences (RBF-FD)}; a method that draws inspiration from both meshed and meshfree methods.

%\authnote{this paragraph does not fit well}
%The first task in traditional meshed methods is to generate an
%underlying grid/mesh. Node placement can be done in a variety of
%ways including uniform, non-uniform and random (monte carlo)
%sampling, or through iterative methods like Lloyd's algorithm
%that generate a regularized sampling of the domain (see e.g.,
%\cite{Du1999}). In addition to choosing nodes, meshed methods
%require connectivity/adjacency lists to form stencils (e.g., Finite Differences) or
%elements (e.g., Finite Element Method). 
%Well balanced meshes play a significant role in isolating nodes and allowing methods to be partitioned for distributed computing. 

%this implies an added challenge to cover
%the domain closure with a chosen element type. While these tasks
%may be straightforward in one- or two-dimensions, the extension
%into higher dimensions becomes increasingly more cumbersome
%\cite{Li2007}. Also, it is often the case that methods have limited support for irregular node distributions (e.g., FD)

For decades meshed methods like Finite Difference, Finite Element and Finite Volume have been the powerhouses in computational modeling. The well-studied Eulerian schemes, with structured and stationary nodes, come backed by a vast literature on topics related to solvers, preconditioners, parallelization, etc. Unfortunately, those methods often come with many restrictions, be it limitations on node placement or other concerns like difficulties in scaling to higher dimensions. In the ideal case we seek a method defined on arbitrary geometries, that behaves regularly in any dimension, and avoids the cost of 
mesh generation. The ability to locally refine areas of interest in a practical fashion is also desirable. Fortunately, meshfree 
methods provide all of these properties: based wholly on a set of independent points in $n$-dimensional space, 
there is minimal cost for mesh generation, and refinement is as simple as adding new points where 
they are needed. 

%Since their adoption by the mathematics community in the 1980s (\cite{Fasshauer2007}), a vast number of meshfree methods have 
%arisen for the solution of partial differential equations (PDEs). Smoothed Particle Hydrodynamics, Partition of Unity method, and element-free Galerkin are all examples applied to fluid flow \cite{Chandhini2007}, but they are not even a scratch on the surface . A good survey of meshfree methods can be found in \cite{Li2007}.
%

A subset of meshfree methods of particular interest to the
numerical modeling community today revolves around Radial Basis Functions (RBFs).
RBFs are a class of radially symmetric functions (i.e.,
symmetric about a point, $x_j$, called the \emph{center}) of the
form: 
	\begin{equation} 
		\phi_j(\vx) = \phi(r(\vx))
	\end{equation} 
where the value of the univariate function $\phi$ is a
function of the Euclidean distance from the center point $\vx_j$ given by
$r(\vx) = ||\vx-\vx_j||_2 = \sqrt{(x-x_j)^2 + (y-y_j)^2 + (z-z_j)^2}$. Examples of
commonly used RBFs are shown in Figure~\ref{fig:rbfs} (for corresponding equations refer to Table~\ref{tbl:rbfs}). RBF
methods are based on a superposition of translates of these
radially symmetric functions, providing a linearly independent
but non-orthogonal basis used to interpolate between nodes in
$n$-dimensional space. 

\begin{figure}[ht]
\centering
	\begin{subfigure}[b]{0.25\textwidth}
    	\centering
		\includegraphics[width=1.0\textwidth]{../figures/paper1/figures/ga_rbf2d-eps-converted-to.pdf} 
		\caption{Gaussian}
	\end{subfigure}
	\begin{subfigure}[b]{0.25\textwidth}
	\centering
		\includegraphics[width=1.0\textwidth]{../figures/paper1/figures/imq_rbf2d-eps-converted-to.pdf}
		\caption{Inverse Multiquadric}
	\end{subfigure}
	\begin{subfigure}[b]{0.25\textwidth}
	\centering
		\includegraphics[width=1.0\textwidth]{../figures/paper1/figures/mq_rbf2d-eps-converted-to.pdf} 
		\caption{Multiquadric}
	\end{subfigure} 
    \caption{Commonly Used Radial Basis Functions (RBFs).}
    \label{fig:rbfs}
\end{figure}

At the core of all RBF-based PDE methods lies the fundamental problem of approximation/interpolation. Some methods (e.g., global- and compact-RBF methods) apply RBFs to approximate derivatives directly. Others (e.g., RBF-FD) leverage the basis functions to compute stencil weights, which in turn are used in generalized finite-difference approximations of derivatives. 

As ``meshless" methods, RBF methods excel at solving problems that require geometric flexibility with scattered node layouts in $d$-dimensional space. They naturally extend into higher dimensions with minimal increase in programming complexity \cite{FlyerWright07,WrightFlyerYuen10}. In addition to competitive accuracy and convergence compared with other state-of-the-art methods \cite{FlyerWright07, FlyerWright09, FlyerLehto10, WrightFlyerYuen10, FlyerFornberg11}, RBF methods boast stability for large time steps. Not all RBF methods are truly meshless. Most of them operate the same given an input set of nodes, whether or not the input includes an underlying mesh. However, some methods connect all nodes to all others (to be meshless), while methods like RBF-FD rely on stencils to connect points within small neighborhoods of one another. The generated stencils limit the scope of influence for nodes and complexity of the method, but also simulate a mesh. 

%Most literature surrounding RBF solutions for PDEs apply the concept of collocation (see Chapter~\ref{chap:background}). Recent trends in the community appear to have shifted, and RBF-FD seems to be top of the list. 


%RBF-FD is a hybrid of RBF scattered data interpolation and classical Finite Difference (FD). It shares many of the benefits from other RBF methods, like functionality on scattered node layouts in any dimension and high-order accurate solutions. Likewise, benefits akin to classical FD such as compact stencils for derivative approximation, and low computational complexity.

%Classical FD derivatives are expressed at a single node (center) as a weighted combination/difference of solution values from a small neighborhood (i.e., a stencil) around the center. Common approximations such as upwind differencing, center differencing, and other higher order approximations are of this form. 
%In similar fashion, RBF-FD combines solutions values based on stencils, but it does so in a more generalized sense than standard FD. For example, classical FD is typically restricted to regular meshes and often symmetric stencils in practice with the same set of weights for each stencil. Weights can be derived from polynomial expansion and obtained in 1D by solving a Vandermonde interpolation matrix \cite{FornbergLehto11}. Higher dimension FD stencils are composed from combinations of 1D formulas applied to each dimension. This implies restrictions on the shape/layout of stencils. In contrast to this, RBF-FD is designed for stencils with irregular node placement and can easily provide a unique set of weights for each stencil with no restrictions on stencil shape. 

%TODO:  \cite{Wright2004, Wright2003, WrightFornberg06, Chandhini2007}. 
To track the history of RBF methods, one must rewind to 1971 and R.L. Hardy's seminal research on interpolation with multi-quadric basis functions \cite{Hardy1971}. The transition from interpolation to RBF PDE methods dates back to 1990 \cite{Kansa1990a,Kansa1990b}. 
RBF-FD was first introduced in concept in 2000 \cite{Tolstykh2000}, but it took another few years for the method to really get a start (\cite{Shu2003}, \cite{Tolstykh2003a}, \cite{Wright2003} and \cite{Cecil2004}). Now a little over a decade old, the method is finally showing signs that it has the critical-mass following necessary for use in large-scale scientific models. At the onset of this work, most of the literature considered RBF-FD for problem sizes up to a few thousand nodes; at most into tens of thousands of nodes. Similar to most RBF methods, RBF-FD is predominantly implemented within small-scale, serial computing environments. The RBF community at large continues to depend on MATLAB for investigation and extension development.

The goal of this dissertation is to scale RBF-FD solutions on high resolution meshes across high performance clusters, and to lead the way for its adoption within HPC and supercomputing circles. As part of this push to HPC, leveraging \emph{Graphics Processing Units (GPUs)} for computation is considered critical. GPUs, introduced in Chapter~\ref{chap:gpu_rbffd}, are many-core accelerators capable of general purpose, embarrassingly parallel computations. Accelerators represent the latest trend in HPC, where compute nodes are commonly supplemented by one or more accessory boards for offloading compute intensive tasks. Our effort leads the way for RBF-FD applicationss in an age when compute nodes with attached accelerator boards are considered key to breaching the exa-scale computing barrier \cite{GPUandExascale2011}. 

Parallelization of RBF-FD is achieved at two levels. On the first level, the
physical domain of the problem is partitioned
into overlapping subdomains, each handled by different MPI processes. All processes
operate independently to compute/load RBF-FD stencil weights, run diagnostic
tests and perform other initialization tasks. A process only computes weights
corresponding to stencils centered in the interior of its partition. After
initialization, processes work together to concurrently solve the PDE. Communication
barriers ensure that processes execute in lockstep and maintain consistent
solution values across the domain.  

The second level of
parallelization offloads tasks to the GPU at each iteration of the PDE solver. The bulk of computation in RBF-FD relies on a \emph{Sparse Matrix-Vector multiply (SpMV)} to evaluate derivatives. When mapped to the GPU, the SpMV operation is broken into two parts in order to overlap communication and computation, and amortize the cost of data transfer across multiple levels of hardware (see Chapter~\ref{chap:multigpu_rbffd}). 
% Although the stencil weight calculation is also data-parallel, we assume that in this context that the weights are precomputed and loaded once from disk during the initialization phase. 

%Additional key challenges lie in the choice of grid, the choice of stencil, and stability of scalable solutions. 
%To demonstrate Chapter~\ref{chap:applied_rbffd} verifies 


%
%Even today, RBFs are still up-and-coming in the scientific world with many avenues of research left to consider. Global formulations are understood to have spectral convergence properties, high accuracy and other benefits like adaptivity and ease of implementation over meshed methods \cite{Fasshauer:2007}. However, little is known about the behavior of local and RBF-FD methods. Open questions include (but are in no way limited to): a) ideal node placement to eliminate singularities; b) data-structures for stencil storage and evaluation; c) problem sizes larger than a few thousand nodes; and d) parallel implementations across new heterogeneous multi- and many-core architectures. In response to this, our group, in collaboration with researchers assembled from a national lab and four universities (see Chapter~\ref{chap:funding}), has been granted funds by the National Science Foundation to collaboratively:
%
%\begin{quote}
%\emph{``Bring RBFs to the forefront of multi-scale geophysical modeling by developing fast, efficient, and parallelizable RBF algorithms in arbitrary geometries, with performance enhanced by hardware accelerators, such as graphic processing units (GPUs)."} \cite{RBF_proposal:2009}
%\end{quote}
%
%



The layout of this document is as follows. This chapter continues with a survey of work related to parallelizing RBF-FD, targeting the GPU, and spanning a multi-GPU cluster. Chapter~\ref{chap:background} provides a historical survey of RBF methods as a backdrop to present RBF-FD in Chapter~\ref{chap:rbffd_method}. Chapter~\ref{chap:stencils} introduces a new algorithm for generating RBF-FD stencils as a faster alternative to the RBF community favorite, $k$-D Tree. In Chapter~\ref{chap:distributed_rbffd}, the first scalable implementation of RBF-FD to span one thousand processors is described in detail. Chapter~\ref{chap:gpu_rbffd} continues with the challenge of offloading computation to GPUs, and Chapter~\ref{chap:multigpu_rbffd} expands the discussion to RBF-FD on a multi-GPU cluster. In Chapter~\ref{chap:applications} the parallel RBF-FD implementation is applied to solve both explicit and implicit geophysical PDEs. Finally, Chapter~\ref{chap:conclusion} concludes with a summary of novel contributions, results and a discussion on future directions.

\section{On Parallel/Distributed RBF-FD} 

%\authnote{Related work for start of Parallel/GPU chapter}
Parallel implementations of RBF methods rely on domain decomposition. Depending on the implementation, domain decomposition not only accelerates solution procedures, but can decrease the ill-conditioning that plague all global RBF methods \cite{Divo2007}. The ill-conditioning is reduced if each domain is treated as a separate RBF domain, and the boundary update is treated separately. Domain decomposition methods for RBFs were introduced by Beatson et al. \cite{Beatson2000} in the year 2000 as a way to increase problem sizes into the millions of nodes.

This work leverages a domain decomposition, but not for the purpose of conditioning. Instead the focus is on decomposing the domain in order to scale RBF-FD across more than a thousand CPU cores of an HPC cluster. Add to this the twist of incorporating a novel implementation on the GPU with overlapping communication and computation. This combination is unmatched in related work. However, RBF methods do have a bit of history of parallel implementations. 

In 2007, Divo and Kassab \cite{Divo2007} used a domain decomposition method with artificial 
subdomain boundaries for their implementation of a local collocation method \cite{Divo2007}. 
The subdomains are processed independently, with derivative values 
at artificial boundary points averaged to maintain global consistency of physical values. Their implementation 
was designed for a 36 node cluster, but benchmarks and scalability tests are not provided.

% Divo: it seems almost unnecessary to use domain decomposition if they have the local method.
% I suppose the domain decomposition is necessary for averaging physical values more than RBF collocation.
However, research on the parallelization of RBF algorithms to solve PDEs on multiple CPU/GPU architectures is essentially non-existent. We have found three studies that have addressed this topic, none of which implement RBF-FD but rather take the avenue of domain decomposition for global RBFs (similar to a spectral element approach). In \cite{Divo2007}, Divo and Kassab introduce subdomains with artificial boundaries that are processed independently. Their implementation was designed for a 36 node cluster, but benchmarks and scalability tests are not provided. Kosec and \v{S}arler \cite{Kosec2008} parallelize coupled heat transfer and fluid flow models using OpenMP on a single workstation with one dual-core processor. They achieved a speedup factor of 1.85x over serial execution, although there were
no results from scaling tests. Yokota, Barba and Knepley \cite{Yokota2010} apply a restrictive additive Schwarz domain decomposition to parallelize global RBF interpolation of more then 50 million nodes on 1024 CPU processors.

Kosec and \v{S}arler \cite{Kosec2008} have the only known (to our knowledge) OpenMP implementation for RBFs. The authors parallelize coupled heat transfer 
and fluid flow problems on a single workstation. 
The application involves the local RBF collocation method, explicit time-stepping and Neumann boundary conditions. A speedup 
factor of 1.85x over serial execution was achieved by executing on two CPU cores; no further 
results from scaling tests were provided. 

Stevens et al. \cite{Stevens2009a} mention a parallel implementation under development, but no document is available at this time. 

Perhaps the most competitive parallel implementation of RBFs is the PetRBF branch of PETSc \cite{Yokota2010}. The authors of PetRBF have implemented a highly scalable, efficient RBF interpolation method based on compact RBFs (i.e., they operate on sparse matrices). The authors demonstrate efficient weak scaling of PetRBF across 1024 processes on a Blue Gene/L, and strong scaling up to 128 processes on the same hardware. On the Blue Gene/L, PetRBF is demonstrated to achieve an impressive 74\% parallel weak scaling efficiency on 1024 processes (operating on over 50 million points), and 84\% strong scaling efficiency for 128 processes. Strong scaling was also tested on a Cray XT4, where strong scaling tops out at 36\% for 128 processes, a respectable number---and similar to observed results for our own code on \authnote{review: } 128 processes.  


%Within the scope of this paper we detail our method for spanning RBF-FD across multiple CPU/GPU processors and emphasize numerical validation of the implementation rather than optimization strategies. We will consider optimization in future work. The calculations are performed on Keeneland, a high performance computing installation supported by the National Science Foundation and located at Oak Ridge National Lab. Keeneland currently has 240 CPUs accompanied by 360 NVidia Fermi class GPUs with at least double that number expected by the end of 2012 \cite{Vetter2011}.



\section{On GPU RBF Methods}

With regard to the latter, there is some research on leveraging RBFs on GPUs in the fields of visualization \cite{Cuntz2007,Weiler2005},  surface reconstruction \cite{Corrigan2005,Carr2003}, and neural networks \cite{Brandstetter2008}.

  Only Schmidt et al. \cite{Schmidt2009b} have accelerated a global RBF method for PDEs on the GPU. Their MATLAB implementation applies global RBFs to solve the linearized shallow water equations utilizing the AccelerEyes Jacket \cite{JacketGuide2009} library to target a single GPU.


GPUs were introduced in 80's \hl{see master's thesis}.

Originally GPUs were designed as parallel rasterizing units. They had limited logic control in contrast to the serial CPUs and their advanced branching and looping logic. 

Gradually new and complex logic was added to the GPU to produce the shader languages that allowed developers to customize specific parts of the rendering pipeline. This allowed scientific problems such as the diffusion equation \hl{cite Lore and others} to be solved in process of rendering. In other words, the GPU was tricked into computing.

The year 2006 brought the modern age of GPU computing with the introduction of CUDA from NVidia. The high level language allowed scientists to leverage the GPU as a parallel accelerator without all of the overhead of setting up graphics contexts and tricking the hardware into computing. Memory management is still the developer's responsibility, but compiler transforms generic C/Fortran code to GPU instruction set. 

Scientific Computing has seen a widespread adoption of GPGPU because of the goal to get to ``exa-scale'' computing, which may only be possible in the near future with the help of GPU accelerators \cite{GPUandExascale2011}. 

NVidia is not the only company involved in many core parallel accelerators. Other groups like AMD and Intel have been increasing the number of cores as well. The end effect is a hybridization where CPUs look similar to GPUs and vice-versa. 

Until 2009, the hardware distinction required that developers target parallelism on CPUs and GPUs using different languages. Then the OpenCL standard was drafted and implemented. OpenCL is a parallel language that strives to provide functional portability rather than performance. 

We focus on the OpenCL language within this dissertation with confidence that hardware will change frequently. In fact, every 18 months \hl{cite} shows a new release of GPU hardware, manycore CPU hardware and extensions to parallel languages. But if hardware is constantly changing, then we need to focus on a high level implementation that allows portability. We need a language like OpenCL to carry our implementations into the future regardless of what hardware and which company survive. 


Related work on RBFs and GPUs is sparse. In 2009, Schmidt et al. \cite{Schmidt2009a, Schmidt2009b} implemented a global RBF method for Tsunami simulation on the GPU using the AccelerEyes Jacket \cite{JacketGuide2009} add-on for MATLAB. Jacket provides a MATLAB interface to data structures and routines that internally call to the NVidia CUDA API. Their model was based on a single large dense matrix solve, and with the help of Jacket the authors were able to achieve approximately 7x speedup over the standard MATLAB solution on the then current generation of the MacBook Pro laptop. The authors compared the laptop CPU (processor details not specified) to the built-in NVidia GeForce 8600M GT GPU. Schmidt et al.'s implementation was the first contribution to the RBF community to leverage accelerators. The results were significant and promising, but no further contributions were made on the topic. 

While both Schmidt et al.'s method and the method presented here are based on RBFs, the two problems are only distantly related when it comes to implementation on the GPU. Dense matrix operations have a high computational complexity, are considered ideal (or near to) by linear algebra libraries like BLAS \cite{BLAS} and LAPACK \cite{Lapack1999}, and were demonstrated to fit well on GPUs from the onset of General Purpose GPU (GPGPU) Computing. In fact, NVidia included CUBLAS \cite{CUBLAS} (a GPU based BLAS library for their hardware) with their initial public release of the game-changing CUDA development kit in 2006. In stark contrast to this, sparse matrix operations have minimal computational complexity and are less than ideal for the GPU.


Earlier this year (2013), Cuomo et al. \cite{Cuomo2013} implemented RBF-interpolation on the GPU for surface reconstruction. Their implementation utilizes PetRBF \cite{Yokota2010}, and new built-in extensions that allow GPU access within PETSc. PETSc internally wraps the CUSP project \cite{CUSP} for sparse matrix algebra on the GPU. With the help of these libraries, Cuomo et al. solve and apply sparse interpolation systems on the GPU for up to three million nodes on an NVidia Fermi C1060 GPU (4GB). They compare results to a single core CPU implementation on an Intel i7-940 CPU and demonstrate that the GPU accelerate their solutions between 6x and 25x. Unfortunately, the authors do not show evidence of scaling the interpolation across multiple GPUs; so while evidence exists that PetRBF now has full GPU support, it remains to be seen how well the code can scale in GPU mode. 
 
\section{On Multi-GPU Methods}
 
Multi-GPU Jacobi iteration for Navier stokes flow in cavity \url{http://scholarworks.boisestate.edu/cgi/viewcontent.cgi?article=1003&context=mecheng_facpubs}

Thibault et al. have multiple works on Multi-GPU and overlapping comm and comp. 



\ifstandalone
%Bei direkter Übersetzung sollte gleich noch das Literaturverzeichnis rein.
\bibliographystyle{plain}
\bibliography{merged_references}
\end{document}
\else
\expandafter\endinput
\fi
