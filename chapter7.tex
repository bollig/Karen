%!TEX root = karen.tex

\chapter{Neighbor Queries and Node Ordering}


\section{Neighbor Queries} 

As part of the preprocessing stage for RBF-FD, the scattered point cloud must be analyzed to generate stencils. To generate a stencil, any collection of nodes can be selected. However, by choosing nodes close to the stencil center and well balanced around it, we stand to get the best possible approximations to derivatives. 

Why? well, the approximations are based on differences. Similar to classic FD, draw a secant connecting two nodes of a stencil. The slope of the secant determines the gradient at either point or a point in between. In the limit as the points are moved closer to the same spot, the approximation to the derivative at that point becomes exact. 

\section{Node Ordering}

Locality sensitive hashing also allows us to reorder the nodes

\cite{Saad2003} mentions the impact of ordering on conditioning.

Algorithms like Reverse Cuthill McKee and Approximate Minimum Degree ordering allow general restructuring of matrices. 

\authnote{NEed to compare conditioning of LSH and other algorithms in Matlab}

Q: what is an ideal ordering?
Q: what is the best conditioning from ordering?
Q: what is the relative cost of ordering?


