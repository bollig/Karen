%!TEX root = karen.tex

These figures represent optimizations of the Cosine Bell and Vortex Roll-up test cases. Essentially, the optimizations here are general for multi-GPU SpMV. Improving these test cases improves all explicit schemes for RBF-FD (i.e., hyperbolic and parabolic equations, and various time stepping schemes like euler, RK4, Adams-Bashforth etc.). 

\section{MPI\_Alltoallv}
Change send/recv to alltoallv. Track wait time. Show limitation on scalability with GPU (sublinear) vs CPU (linear). How high can we get linear on CPU? 


Communication between processors requires each processor to iterate through their neighboring processors and share information. This can be seen as a simple for loop allowing every processor to touch its neighbors in round-robin fashion. The benchmarks seen in \authnote{figures from paper1} show the strong scaling of our implementation with a for loop and send/recv. 

In Figs~\ref{figs:alltoall_scaling} we show the strong scaling of the cosine bell after 

Alternatively

an all to all collective. That is, all processors share some information with potentially every other processor. 

\section{Asynchronous OpenCL}

What are the limitations if using just async and not the queues?

\section{Multi-Queue OpenCL}
How does performance improve if we use two queues (one for Q and one for R)? 


\section{Work-Group Size and Number of Stencils}
What if a work-group is larger than a warp? What if the group was occupied by multiple stencils. What improvements to speedup do we see?

How many stencils can each group handle (assuming values stay in shared memory? 
Shared memory bank conflicts? How do we sort the values? 
What is the occupancy of the GPU?

\section{Parallel Reduction in Shared Memory}
What significant gain do we see from adding a segmented scan to the shared memory? 

What other improvements can we think of? 

\section{Comparison: custom SpMV for explicit schemes vs ViennaCL}