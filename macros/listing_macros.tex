\usepackage{listings}

% Setup an environment similar to verbatim but which will highlight any bash commands we have
\lstnewenvironment{unixcmds}[0]
{
%\lstset{language=bash,frame=shadowbox,rulesepcolor=\color{blue}}
\lstset{ %
language=sh,		% Language
basicstyle=\ttfamily,
backgroundcolor=\color{light-gray}, 
rulecolor=\color{blue},
%frame=tb, 
columns=fullflexible,
%framexrightmargin=-.2\textwidth,
linewidth=0.8\textwidth,
breaklines=true,
%prebreak=/, 
  prebreak = \raisebox{0ex}[0ex][0ex]{\ensuremath{\hookleftarrow}},
%basicstyle=\footnotesize,       % the size of the fonts that are used for the code
%numbers=left,                   % where to put the line-numbers
%numberstyle=\footnotesize,      % the size of the fonts that are used for the line-numbers
%stepnumber=2,                   % the step between two line-numbers. If it's 1 each line 
                                % will be numbered
%numbersep=5pt,                  % how far the line-numbers are from the code
showspaces=false,               % show spaces adding particular underscores
showstringspaces=false,         % underline spaces within strings
showtabs=false,                 % show tabs within strings adding particular underscores
frame=single,	                % adds a frame around the code
tabsize=2,	                % sets default tabsize to 2 spaces
captionpos=b,                   % sets the caption-position to bottom
breakatwhitespace=false,        % sets if automatic breaks should only happen at whitespace
}
} { }

% Setup an environment similar to verbatim but which will highlight any bash commands we have
\lstnewenvironment{cppcode}[1]
{
%\lstset{language=bash,frame=shadowbox,rulesepcolor=\color{blue}}
\lstset{ %
	backgroundcolor=\color{light-gray}, 
	rulecolor=\color[rgb]{0.133,0.545,0.133},
	tabsize=4,
	language=[GNU]C++,
%	basicstyle=\ttfamily,
        basicstyle=\scriptsize,
        upquote=true,
        aboveskip={1.5\baselineskip},
        columns=fullflexible,
        %framexrightmargin=-.1\textwidth,
       %framexleftmargin=6mm,
        showstringspaces=false,
        extendedchars=true,
        breaklines=true,
        prebreak = \raisebox{0ex}[0ex][0ex]{\ensuremath{\hookleftarrow}},
        frame=single,
        showtabs=false,
        showspaces=false,
        showstringspaces=false,
        numbers=left,                   % where to put the line-numbers
	numberstyle=\footnotesize,      % the size of the fonts that are used for the line-numbers
	stepnumber=4,                   % the step between two line-numbers. If it's 1 each line 
                                % will be numbered
	firstnumber=#1,
         numbersep=5pt,                  % how far the line-numbers are from the code
        identifierstyle=\ttfamily,
        keywordstyle=\color[rgb]{0,0,1},
        commentstyle=\color[rgb]{0.133,0.545,0.133},
        stringstyle=\color[rgb]{0.627,0.126,0.941},
}
} { }

% Setup an environment similar to verbatim but which will highlight any bash commands we have
\lstnewenvironment{mcode}[1]
{
\lstset{ %
	backgroundcolor=\color{light-gray}, 
	rulecolor=\color[rgb]{0.133,0.545,0.133},
	tabsize=4,
	language=Matlab,
%	basicstyle=\ttfamily,
        basicstyle=\scriptsize,
        upquote=true,
        aboveskip={1.5\baselineskip},
        columns=fullflexible,
        %framexrightmargin=-.1\textwidth,
       %framexleftmargin=6mm,
        showstringspaces=false,
        extendedchars=true,
        breaklines=true,
        prebreak = \raisebox{0ex}[0ex][0ex]{\ensuremath{\hookleftarrow}},
        frame=single,
        showtabs=false,
        showspaces=false,
        showstringspaces=false,
        numbers=left,                   % where to put the line-numbers
	numberstyle=\footnotesize,      % the size of the fonts that are used for the line-numbers
	stepnumber=4,                   % the step between two line-numbers. If it's 1 each line 
                                % will be numbered
	firstnumber=#1,
         numbersep=5pt,                  % how far the line-numbers are from the code
        identifierstyle=\ttfamily,
        keywordstyle=\color[rgb]{0,0,1},
        commentstyle=\color[rgb]{0.133,0.545,0.133},
        stringstyle=\color[rgb]{0.627,0.126,0.941},
}
} { }

\newcommand{\inputmcode}[1]{%
\lstset{ %
	backgroundcolor=\color{light-gray},  %
	rulecolor=\color[rgb]{0.133,0.545,0.133}, %
	tabsize=4, %
	language=Matlab, %
%	basicstyle=\ttfamily,
        basicstyle=\scriptsize, %
        %        upquote=true,
        aboveskip={1.5\baselineskip}, %
        columns=fullflexible, %
        %framexrightmargin=-.1\textwidth,
       %framexleftmargin=6mm,
        showstringspaces=false, %
        extendedchars=true, %
        breaklines=true, %
        prebreak = \raisebox{0ex}[0ex][0ex]{\ensuremath{\hookleftarrow}}, %
        frame=single, %
        showtabs=false, %
        showspaces=false, %
        showstringspaces=false,%
        numbers=left,                   % where to put the line-numbers
	numberstyle=\footnotesize,      % the size of the fonts that are used for the line-numbers
	stepnumber=4,                   % the step between two line-numbers. If it's 1 each line 
                                % will be numbered
         numbersep=5pt,                  % how far the line-numbers are from the code
        identifierstyle=\ttfamily, %
        keywordstyle=\color[rgb]{0,0,1}, %
        commentstyle=\color[rgb]{0.133,0.545,0.133}, %
        stringstyle=\color[rgb]{0.627,0.126,0.941} %
}
\lstinputlisting{#1}%
}

%\lstset{ %
%	backgroundcolor=\color{light-gray}, 
%	rulecolor=\color[rgb]{0.133,0.545,0.133},
%	tabsize=4,
%	language=Matlab,
%%	basicstyle=\ttfamily,
%        basicstyle=\scriptsize,
%        upquote=true,
%        aboveskip={1.5\baselineskip},
%        columns=fullflexible,
%        %framexrightmargin=-.1\textwidth,
%       %framexleftmargin=6mm,
%        showstringspaces=false,
%        extendedchars=true,
%        breaklines=true,
%        prebreak = \raisebox{0ex}[0ex][0ex]{\ensuremath{\hookleftarrow}},
%        frame=single,
%        showtabs=false,
%        showspaces=false,
%        showstringspaces=false,
%        numbers=left,                   % where to put the line-numbers
%	numberstyle=\footnotesize,      % the size of the fonts that are used for the line-numbers
%	stepnumber=4,                   % the step between two line-numbers. If it's 1 each line 
%                                % will be numbered
%	firstnumber=#1,
%         numbersep=5pt,                  % how far the line-numbers are from the code
%        identifierstyle=\ttfamily,
%        keywordstyle=\color[rgb]{0,0,1},
%        commentstyle=\color[rgb]{0.133,0.545,0.133},
%        stringstyle=\color[rgb]{0.627,0.126,0.941},
%}


