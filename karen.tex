\documentclass[10pt]{article}
%\documentclass[letter,10pt]{article}

\usepackage{graphicx}
\usepackage{graphics}
\usepackage{epstopdf}

\usepackage{ulem}
\usepackage{verbatim}
%\linespread{1.6}  % double spaces lines
\usepackage[left=1in,top=1in,right=1in,bottom=1in,nohead]{geometry}

% Temporarily put all figures and tables at the end of the document
%\usepackage[nolists,]{endfloat}
%\renewcommand{\efloatseparator}{\mbox{}}

\usepackage{array}
\usepackage{arydshln}
\usepackage{amsfonts, amsmath, amssymb, graphics}
\usepackage{multirow}
\usepackage{multicol}
\usepackage{hyperref}
% make ref numbers function as hyperlinks in the text. 
\hypersetup{
    colorlinks,%
    citecolor=black,%
    filecolor=black,%
    linkcolor=black,%
    urlcolor=black
}
% Make URLS the same font style as the rest of the text.
%\urlstyle{same}

\usepackage{algpseudocode}
\usepackage{algorithm}
\usepackage{hypcap}
	
%\usepackage{mathpazo}
\usepackage{flexisym}
\usepackage{breqn}
\usepackage[numbers,sort&compress]{natbib}
\usepackage{hypernat}
\usepackage{multirow}
\usepackage[center]{subfigure}


\newcommand{\mathsym}[1]{{}}
\newcommand{\unicode}[1]{{}}

\newcommand{\vv}{\mathbf{v}}
\newcommand{\vu}{\mathbf{u}}
\newcommand{\vx}{\mathbf{x}}

%\newcommand{\grad}[1]{\nabla{#1}}
%\newcommand{\div}[1]{\nabla \cdot {#1}}
%\newcommand{\curl}[1]{\nabla \times {#1}}

\usepackage[usenames,dvipsnames]{color}
%\usepackage[english]{babel}
\usepackage{tabularx}
\usepackage{soul}
\usepackage{xparse}
\usepackage{listings}
%\usepackage[normalem]{ulem}



%%%%%%%%%%%%%%%
% Show a list of items "todo" or "done" 
% USAGE: 
% \begin{todolist} 
% 	\todo Something not finished
% 	\done Something finished
% \end{todolist} 
\newenvironment{todolist}{%
  \begin{list}{}{}% whatever you want the list to be
  \let\olditem\item
  \renewcommand\item{\olditem \textcolor{red}{(TODO)}: }
  \newcommand\todo{\olditem \textcolor{red}{(TODO)}: }
   \newcommand\done{\olditem \textcolor{ForestGreen}{(DONE)}: }
}{%
  \end{list}
} 
%%%%%%%%%%%%%%%

%%%%%%%%%%%%%%%
% Show a Author's Note
% USAGE: 
% \incomplete[Optional footnote message to further clarify note]{The text which is currently not finished}
\DeclareDocumentCommand \incomplete{ o m }
{%
\IfNoValueTF {#1}
{\textcolor{red}{Incomplete: \ul{#2}}} 
{\textcolor{red}{Incomplete: \ul{#2}}\footnote{Comment: #1}}%
}
%%%%%%%%%%%%%%%



%%%%%%%%%%%%%%%
% Show a Author's Note
% USAGE: 
% \authnote[Optional footnote message to further clarify note]{The note to your readers}
\DeclareDocumentCommand \authnote { o m }
{%
\IfNoValueTF {#1}
{\textcolor{blue}{Author's Note: \ul{#2}}} 
{\textcolor{blue}{Author's Note: \ul{#2}}\footnote{Comment: #1}}%
}
%%%%%%%%%%%%%%%



%%%%%%%%%%%%%%%
% Strike out text that doesn't belong in the paper
% USAGE: 
% \strike[Optional footnote to state why it doesn't belong]{Text to strike out}
\DeclareDocumentCommand \strike { o m }
{%
\setstcolor{Red}
\IfNoValueTF {#1}
{\textcolor{Gray}{\st{#2}}} 
{\textcolor{Gray}{\st{#2}}\footnote{Comment: #1}}%
}
%%%%%%%%%%%%%%%

\definecolor{light-gray}{gray}{0.95}

\newcommand{\cbox}[3]{
\ \\
\fcolorbox{#1}{#2}{
\parbox{\textwidth}{
#3
}
}
}

% Setup an environment similar to verbatim but which will highlight any bash commands we have
\lstnewenvironment{unixcmds}[0]
{
%\lstset{language=bash,frame=shadowbox,rulesepcolor=\color{blue}}
\lstset{ %
language=sh,		% Language
basicstyle=\ttfamily,
backgroundcolor=\color{light-gray}, 
rulecolor=\color{blue},
%frame=tb, 
columns=fullflexible,
%framexrightmargin=-.2\textwidth,
linewidth=0.8\textwidth,
breaklines=true,
%prebreak=/, 
  prebreak = \raisebox{0ex}[0ex][0ex]{\ensuremath{\hookleftarrow}},
%basicstyle=\footnotesize,       % the size of the fonts that are used for the code
%numbers=left,                   % where to put the line-numbers
%numberstyle=\footnotesize,      % the size of the fonts that are used for the line-numbers
%stepnumber=2,                   % the step between two line-numbers. If it's 1 each line 
                                % will be numbered
%numbersep=5pt,                  % how far the line-numbers are from the code
showspaces=false,               % show spaces adding particular underscores
showstringspaces=false,         % underline spaces within strings
showtabs=false,                 % show tabs within strings adding particular underscores
frame=single,	                % adds a frame around the code
tabsize=2,	                % sets default tabsize to 2 spaces
captionpos=b,                   % sets the caption-position to bottom
breakatwhitespace=false,        % sets if automatic breaks should only happen at whitespace
}
} { }

% Setup an environment similar to verbatim but which will highlight any bash commands we have
\lstnewenvironment{cppcode}[1]
{
%\lstset{language=bash,frame=shadowbox,rulesepcolor=\color{blue}}
\lstset{ %
	backgroundcolor=\color{light-gray}, 
	rulecolor=\color[rgb]{0.133,0.545,0.133},
	tabsize=4,
	language=[GNU]C++,
%	basicstyle=\ttfamily,
        basicstyle=\scriptsize,
        upquote=true,
        aboveskip={1.5\baselineskip},
        columns=fullflexible,
        %framexrightmargin=-.1\textwidth,
       %framexleftmargin=6mm,
        showstringspaces=false,
        extendedchars=true,
        breaklines=true,
        prebreak = \raisebox{0ex}[0ex][0ex]{\ensuremath{\hookleftarrow}},
        frame=single,
        showtabs=false,
        showspaces=false,
        showstringspaces=false,
        numbers=left,                   % where to put the line-numbers
	numberstyle=\footnotesize,      % the size of the fonts that are used for the line-numbers
	stepnumber=4,                   % the step between two line-numbers. If it's 1 each line 
                                % will be numbered
	firstnumber=#1,
         numbersep=5pt,                  % how far the line-numbers are from the code
        identifierstyle=\ttfamily,
        keywordstyle=\color[rgb]{0,0,1},
        commentstyle=\color[rgb]{0.133,0.545,0.133},
        stringstyle=\color[rgb]{0.627,0.126,0.941},
}
} { }

% Setup an environment similar to verbatim but which will highlight any bash commands we have
\lstnewenvironment{mcode}[1]
{
\lstset{ %
	backgroundcolor=\color{light-gray}, 
	rulecolor=\color[rgb]{0.133,0.545,0.133},
	tabsize=4,
	language=Matlab,
%	basicstyle=\ttfamily,
        basicstyle=\scriptsize,
        upquote=true,
        aboveskip={1.5\baselineskip},
        columns=fullflexible,
        %framexrightmargin=-.1\textwidth,
       %framexleftmargin=6mm,
        showstringspaces=false,
        extendedchars=true,
        breaklines=true,
        prebreak = \raisebox{0ex}[0ex][0ex]{\ensuremath{\hookleftarrow}},
        frame=single,
        showtabs=false,
        showspaces=false,
        showstringspaces=false,
        numbers=left,                   % where to put the line-numbers
	numberstyle=\footnotesize,      % the size of the fonts that are used for the line-numbers
	stepnumber=4,                   % the step between two line-numbers. If it's 1 each line 
                                % will be numbered
	firstnumber=#1,
         numbersep=5pt,                  % how far the line-numbers are from the code
        identifierstyle=\ttfamily,
        keywordstyle=\color[rgb]{0,0,1},
        commentstyle=\color[rgb]{0.133,0.545,0.133},
        stringstyle=\color[rgb]{0.627,0.126,0.941},
}
} { }

\newcommand{\inputmcode}[1]{%
\lstset{ %
	backgroundcolor=\color{light-gray},  %
	rulecolor=\color[rgb]{0.133,0.545,0.133}, %
	tabsize=4, %
	language=Matlab, %
%	basicstyle=\ttfamily,
        basicstyle=\scriptsize, %
        %        upquote=true,
        aboveskip={1.5\baselineskip}, %
        columns=fullflexible, %
        %framexrightmargin=-.1\textwidth,
       %framexleftmargin=6mm,
        showstringspaces=false, %
        extendedchars=true, %
        breaklines=true, %
        prebreak = \raisebox{0ex}[0ex][0ex]{\ensuremath{\hookleftarrow}}, %
        frame=single, %
        showtabs=false, %
        showspaces=false, %
        showstringspaces=false,%
        numbers=left,                   % where to put the line-numbers
	numberstyle=\footnotesize,      % the size of the fonts that are used for the line-numbers
	stepnumber=4,                   % the step between two line-numbers. If it's 1 each line 
                                % will be numbered
         numbersep=5pt,                  % how far the line-numbers are from the code
        identifierstyle=\ttfamily, %
        keywordstyle=\color[rgb]{0,0,1}, %
        commentstyle=\color[rgb]{0.133,0.545,0.133}, %
        stringstyle=\color[rgb]{0.627,0.126,0.941} %
}
\lstinputlisting{#1}%
}

%\lstset{ %
%	backgroundcolor=\color{light-gray}, 
%	rulecolor=\color[rgb]{0.133,0.545,0.133},
%	tabsize=4,
%	language=Matlab,
%%	basicstyle=\ttfamily,
%        basicstyle=\scriptsize,
%        upquote=true,
%        aboveskip={1.5\baselineskip},
%        columns=fullflexible,
%        %framexrightmargin=-.1\textwidth,
%       %framexleftmargin=6mm,
%        showstringspaces=false,
%        extendedchars=true,
%        breaklines=true,
%        prebreak = \raisebox{0ex}[0ex][0ex]{\ensuremath{\hookleftarrow}},
%        frame=single,
%        showtabs=false,
%        showspaces=false,
%        showstringspaces=false,
%        numbers=left,                   % where to put the line-numbers
%	numberstyle=\footnotesize,      % the size of the fonts that are used for the line-numbers
%	stepnumber=4,                   % the step between two line-numbers. If it's 1 each line 
%                                % will be numbered
%	firstnumber=#1,
%         numbersep=5pt,                  % how far the line-numbers are from the code
%        identifierstyle=\ttfamily,
%        keywordstyle=\color[rgb]{0,0,1},
%        commentstyle=\color[rgb]{0.133,0.545,0.133},
%        stringstyle=\color[rgb]{0.627,0.126,0.941},
%}


\newcommand{\Laplacian}[1]{\nabla^2 #1}

% set of all nodes received and contained on GPU
\newcommand{\setAllNodes}[0]{\mathcal{G}}
% set of stencil centers on GPU
\newcommand{\setCenters}[0]{\mathcal{Q}}
% set of stencil centers with nodes in \setDepend
\newcommand{\setBoundary}[0]{\mathcal{B}}
% set of nodes received by other GPUs
\newcommand{\setDepend}[0]{\mathcal{R}}
% set of nodes sent to other GPUs
\newcommand{\setProvide}[0]{\mathcal{O}}


\newcommand{\toprule}[0]{\hline}
\newcommand{\midrule}[0]{\hline\hline}
\newcommand{\bottomrule}[0]{\hline}

\newcolumntype{C}{>{\centering\arraybackslash}b{1in}}
\newcolumntype{L}{>{\flushleft\arraybackslash}b{1.5in}}
\newcolumntype{R}{>{\flushright\arraybackslash}b{1.5in}}
\newcolumntype{D}{>{\flushright\arraybackslash}b{2.0in}}
\newcolumntype{E}{>{\flushright\arraybackslash}b{1.0in}}

\DeclareSymbolFont{AMSb}{U}{msb}{m}{n}
\DeclareMathSymbol{\N}{\mathbin}{AMSb}{"4E}
\DeclareMathSymbol{\Z}{\mathbin}{AMSb}{"5A}
\DeclareMathSymbol{\R}{\mathbin}{AMSb}{"52}
\DeclareMathSymbol{\Q}{\mathbin}{AMSb}{"51}
\DeclareMathSymbol{\PP}{\mathbin}{AMSb}{"50}
\DeclareMathSymbol{\I}{\mathbin}{AMSb}{"49}
%\DeclareMathSymbol{\C}{\mathbin}{AMSb}{"43}

%%%%%% VECTOR NORM: %%%%%%%
\newcommand{\vectornorm}[1]{\left|\left|#1\right|\right|}
\newcommand{\vnorm}[1]{\left|\left|#1\right|\right|}
\newcommand{\by}[0]{\times}
\newcommand{\vect}[1]{\mathbf{#1}}
%\newcommand{\mat}[1]{\mathbf{#1}} 

%\renewcommand{\vec}[1]{ \textbf{#1} }
%%%%%%%%%%%%%%%%%%%%%%

%%%%%%% THM, COR, DEF %%%%%%%
%\newtheorem{theorem}{Theorem}[section]
%\newtheorem{lemma}[theorem]{Lemma}
%\newtheorem{proposition}[theorem]{Proposition}
%\newtheorem{corollary}[theorem]{Corollary}
%\newenvironment{proof}[1][Proof]{\begin{trivlist}
%\item[\hskip \labelsep {\bfseries #1}]}{\end{trivlist}}
%\newenvironment{definition}[1][Definition]{\begin{trivlist}
%\item[\hskip \labelsep {\bfseries #1}]}{\end{trivlist}}
%\newenvironment{example}[1][Example]{\begin{trivlist}
%\item[\hskip \labelsep {\bfseries #1}]}{\end{trivlist}}
%\newenvironment{remark}[1][Remark]{\begin{trivlist}
%\item[\hskip \labelsep {\bfseries #1}]}{\end{trivlist}}
%\newcommand{\qed}{\nobreak \ifvmode \relax \else
%      \ifdim\lastskip<1.5em \hskip-\lastskip
%      \hskip1.5em plus0em minus0.5em \fi \nobreak
%      \vrule height0.75em width0.5em depth0.25em\fi}
%%%%%%%%%%%%%%%%%%%%%%

%
%\usepackage[algochapter]{algorithm2e}
%\usepackage[usenames]{color}
% colors to show the corrections
\newcommand{\red}[1]{\textbf{\textcolor{red}{#1}}}
\newcommand{\blue}[1]{\textbf{\textcolor{blue}{#1}}}
\newcommand{\cyan}[1]{\textbf{\textcolor{cyan}{#1}}}
\newcommand{\green}[1]{\textbf{\textcolor{green}{#1}}}
\newcommand{\magenta}[1]{\textbf{\textcolor{magenta}{#1}}}
\newcommand{\orange}[1]{\textbf{\textcolor{orange}{#1}}}
%%%%%%%%%% DK DK
% comments between authors
\newcommand{\toall}[1]{\textbf{\green{@@@ All: #1 @@@}}}
\newcommand{\toevan}[1]{\textbf{\red{*** Evan: #1 ***}}}
%\newcommand{\toevan}[1]{}  % USE FOR FINAL VERSION
\newcommand{\toe}[1]{\textbf{\red{*** Evan: #1 ***}}}
\newcommand{\tog}[1]{\textbf{\blue{*** Gordon: #1 ***}}}
%\newcommand{\togordon}[1]{\textbf{\blue{*** Gordon: #1 ***}}}
\renewcommand{\ge}[3]{{\textcolor{blue}{*** \textbf{Gordon:}\strike{#1} #2 ***}}\red{(#3)}}
\renewcommand{\ge}[3]{{\textcolor{blue}{#2}}}
\renewcommand{\ge}[3]{{\textcolor{Red}{#2}}}
\newcommand{\eb}[3]{{\textcolor{Red}{*** \textbf{Evan:}\strike{#1} #2 ***}}\red{(#3)}}
\renewcommand{\eb}[3]{{{\textcolor{Red}{#2}}}}
%\def\ge#1#2#3{}{\textbf{\blue{*** Gordon: #2 ***}}}{(#3)}
\newcommand{\gee}[1]{{\bf{\blue{{\em #1}}}}}
\newcommand{\old}[1]{}
\newcommand{\del}[1]{***#1*** }



% \DeclareMathOperator{\Sample}{Sample}
%\let\vaccent=\v % rename builtin command \v{} to \vaccent{}
%\renewcommand{\vec}[1]{\ensuremath{\mathbf{#1}}} % for vectors
\newcommand{\gv}[1]{\ensuremath{\mbox{\boldmath$ #1 $}}} 
% for vectors of Greek letters
\newcommand{\uv}[1]{\ensuremath{\mathbf{\hat{#1}}}} % for unit vector
\newcommand{\abs}[1]{\left| #1 \right|} % for absolute value
\newcommand{\avg}[1]{\left< #1 \right>} % for average
\let\underdot=\d % rename builtin command \d{} to \underdot{}
\renewcommand{\d}[2]{\frac{d #1}{d #2}} % for derivatives
\newcommand{\dd}[2]{\frac{d^2 #1}{d #2^2}} % for double derivatives
\newcommand{\pd}[2]{\frac{\partial #1}{\partial #2}} 
% for partial derivatives
\newcommand{\pdd}[2]{\frac{\partial^2 #1}{\partial #2^2}} 
\newcommand{\pdda}[3]{\frac{\partial^2 #1}{\partial #2 \partial #3}} 
% for double partial derivatives
\newcommand{\pdc}[3]{\left( \frac{\partial #1}{\partial #2}
 \right)_{#3}} % for thermodynamic partial derivatives
\newcommand{\ket}[1]{\left| #1 \right>} % for Dirac bras
\newcommand{\bra}[1]{\left< #1 \right|} % for Dirac kets
\newcommand{\braket}[2]{\left< #1 \vphantom{#2} \right|
 \left. #2 \vphantom{#1} \right>} % for Dirac brackets
\newcommand{\matrixel}[3]{\left< #1 \vphantom{#2#3} \right|
 #2 \left| #3 \vphantom{#1#2} \right>} % for Dirac matrix elements
\newcommand{\grad}[1]{\gv{\nabla} #1} % for gradient
\let\divsymb=\div % rename builtin command \div to \divsymb
\renewcommand{\div}[1]{\gv{\nabla} \cdot #1} % for divergence
\newcommand{\curl}[1]{\gv{\nabla} \times #1} % for curl
\let\baraccent=\= % rename builtin command \= to \baraccent
\renewcommand{\=}[1]{\stackrel{#1}{=}} % for putting numbers above =
\newcommand{\diffop}[1]{\mathcal{L}#1}
\newcommand{\boundop}[1]{\mathcal{B}#1}
\newcommand{\rvec}[0]{{\bf r}}

\newcommand{\Interior}[0]{\Omega}
\newcommand{\domain}[0]{\Omega}
%\newcommand{\Boundary}[0]{\partial \Omega}
\newcommand{\Boundary}[0]{\Gamma}

\newcommand{\on}[1]{\hskip1.5em \textrm{ on } #1}

\newcommand{\gemm}{\texttt{GEMM}}
\newcommand{\trmm}{\texttt{TRMM}}
\newcommand{\gesvd}{\texttt{GESVD}}
\newcommand{\geqrf}{\texttt{GEQRF}}


\newcommand{\minitab}[2][l]{\begin{tabular}{#1}#2\end{tabular}}
\newcommand{\comm}[1]{\textcolor{red}{\textit{#1}}}

\newcommand{\nfrac}[2]{
\nicefrac{#1}{#2}
%\frac{#1}{#2}
}

% Rename  this file          misc_mac.tex
%----------------------------------------------------------------------
%%%%%%%%%%%%%%%%%%%%%%%%%%%%%%%%%%%%%%%%%%%%%%%%%%%%%%%%%%%%%%%%%%%%%%%%%%%%%%%
%
%	Math Symbols   Math Symbols   Math Symbols   Math Symbols   
%
%%%%%%%%%%%%%%%%%%%%%%%%%%%%%%%%%%%%%%%%%%%%%%%%%%%%%%%%%%%%%%%%%%%%%%%%%%%%%%%
\def\pmb#1{\setbox0=\hbox{$#1$}%
	\kern-.025em\copy0\kern-\wd0
	\kern.05em\copy0\kern-\wd0
	\kern-.025em\raise.0433em\box0}
\def\pmbf#1{\pmb#1}
\def\bfg#1{\pmb#1}

% BETTER VALUES FOR AUTOMATIC FIGURE PLACEMENT THAN THOSE PROVIDED BY 
% LATEX DEFAULTS.

\renewcommand{\textfloatsep}{1ex}
\renewcommand{\floatpagefraction}{0.9}
\renewcommand{\intextsep}{1ex}
\renewcommand{\topfraction}{.9}
\renewcommand{\bottomfraction}{.9}
\renewcommand{\textfraction}{.1}

% #1  position of floating figure (h|t|b|p)
% #1  EPS postscript file
% #2  size
% #3  caption

%usage of newfig:
%  \newfig{file.ps}{3in}{Fig1: this is a figure}

%\input{epsf}
%\def\newfig#1#2#3{
%  \begin{figure}[htbp]
%  \vspace{1ex}
%  \setlength{\epsfxsize}{#2}
%  \centerline{\epsfbox{#1}}
%  \vspace{-.1in}\caption{\small #3}\break\vspace{.2in}
%  \label{#1}
%  \end{figure}
%}

%usage of newfigtwo: 2 figures, vertically stacked
% \newfig
%	{file1.ps}
%	{file2.ps}
%	{width}
%	{vertical space}
%	{Caption}

\def\newfigtwo#1#2#3#4#5{
  \begin{figure}[htbp]
  \vspace{1ex}
  \setlength{\epsfxsize}{#3}
  \centerline{\epsfbox{#1}}
  \vspace{#4}
  \setlength{\epsfxsize}{#3}
  \centerline{\epsfbox{#2}}
  \vspace{-.1in}\caption{\small #5}\break\vspace{.2in}
  \label{#1}
  \end{figure}
}

\def\newfigh#1#2#3#4{  % add height specification
  \begin{figure}[htbp]
  \vspace{1ex}
  \setlength{\epsfxsize}{#2}
  \setlength{\epsfysize}{#4}
  \centerline{\epsfbox{#1}}
  \vspace{-.1in}\caption{\small #3}\break\vspace{.2in}
  \label{#1}
  \end{figure}
}

\def\herefig#1#2#3{
  \begin{figure}[h]
  \setlength{\epsfxsize}{#2}
  \centerline{\epsfbox{#1}}
  \caption{\small #3}
  \label{#1}
  \end{figure}
}

\def\etal{{{\em et~al.\,\,}}}
\def\note#1{\\ =====#1===== \\}
\def\FBOX#1{\ \\ \fbox{\begin{minipage}{5in}#1\end{minipage}}\\ }
\newcount\sectionno     \sectionno=0
\newcount\eqnum         \eqnum=0
\def\addeqno{\global\advance \eqnum by  1 }
\def\subeqno{\global\advance \eqnum by -1 }
%\def\eqn{\addeqno \eqno \hbox{(\number\sectionno.\number\eqnum)} }

\def\tildetilde#1{\tilde{\tilde{#1}}}
\def\barbar#1{\overbar{\overbar{#1}}}

\def\vsp#1{\vspace{#1 ex}}
\def\fpar{\hspace{\parindent}}
%
%  \pf : 2 arguments: numerator and denominator of partial derivative
%
\def\pf#1#2{{\frac{\partial{#1}}{\partial{#2}}}}
\def\pfs#1#2{{\partial_{#2}{#1}}}
\def\pftwo#1#2{{\frac{\partial^2{#1}}{\partial{#2}^2}}}
\def\pfxx#1#2{{\frac{\partial^2{#1}}{\partial{#2}^2}}}
%\def\pfxy#1#2{{\frac{\partial^2{#1}}{\partial{#2}\partial{#3}}}}
\def\pfn#1#2#3{{\frac{\partial^{#1}{#2}}{\partial{#3}^{#1}}}}
\def\df#1#2{{\frac{d{#1}}{d{#2}}}}
\def\dfn#1#2#3{{\frac{d^{#1}{#2}}{d{#3}^{#1}}}}
\def\Dt#1#2{\frac{D#1}{D#2}}
\def\dt#1#2{\frac{d#1}{d#2}}
\def\bld#1{{\bf #1}}
\def\pfp#1#2#3{\pf{}{#3}{\left(\frac{#1}{#2}\right)}}

\def\norm#1{\|#1\|}

%
% Graphic characters  (\dot already defined by TeX/LateX)
%
\def\dash{\rule[1.5pt]{2mm}{.3mm}\HS{.9mm}}
\def\dott{\rule[1.5pt]{.7mm}{.3mm}\HS{.7mm}}
\def\dashline{\dash\dash\dash}
\def\dotline{\dott\dott\dott\dott\dott\dott}
\def\dashdotline{\dash$\cdot$\HS{.9mm}\dash}
\def\solidline{\rule[2pt]{7mm}{.3mm}}
% 
% overcircle
%
\def\ovcircle#1{\buildrel{\circ}\over{#1}}
%\def\below#1#2{\buildrel{#2}\under{#1}}
%\def\above#1#2{\buildrel{#2}\over{#1}}
%
%  big parenthesis and brackets
%
\def\bigpar#1#2{{\left(\frac{#1}{#2}\right)}}
\def\bigbra#1#2{{\left\[\frac{#1}{#2}\right\]}}

\def\Lp{\left(}
\def\Rp{\right)}
\def\Lb{\left[}
\def\Rb{\right]}
\def\Ln{\left\langle}
\def\Rn{\right\rangle}
\def\Ld{\left.}
\def\Rd{\right.}
\def\Lv{\left|}
\def\Rv{\right|}
\def\Lbr{\left|}
\def\Rbr{\right|}
\def\lng{\langle}
\def\rng{\rangle}
\def\Lc{\left\{}
\def\Rc{\right\}}
%%% %

% Cannot be handled by Lyx
%\def\[{{[}}
%\def\]{{]}}

%
\def\eol{\nonumber \\}
\def\eolnonb{\nonumber\\}
\def\eolnb{\\}
\def\nonb{\nonumber}
\def\be{\begin{equation}}
\def\ee{\end{equation}}
\def\BEQNA{\begin{eqnarray}}
\def\EEQNA{\end{eqnarray}}
\def\eqa{&=&}
\def\beqna{\begin{eqnarray}}
\def\eeqna{\end{eqnarray}}
\def\bverb{\begin{verbatim}}
\def\everb{\end{verbatim}}
\def\VERB#1{\bverb #1 \everb}
\def\btbl{\begin{tabular}}
\def\etbl{\end{tabular}}
\def\bmini{\begin{minipage}[t]{5.5in}}
\def\emini{\end{minipage}}
\def\parray#1#2{\left(\begin{array}{#1}#2\end{array}\right)}
\def\barray#1#2{\left[\begin{array}{#1}#2\end{array}\right]}
\def\carray#1#2{\left\{\begin{array}{#1}#2\end{array}\right.}
\def\darray#1#2{\left|\begin{array}{#1}#2\end{array}\right|}

\def\BEGTABLE#1{\begin{table}[hbt]\vspace{2ex}\begin{center}\bmini\centering\btbl{#1}}
\def\ENDTABLE#1#2{\etbl\caption[#1]{#2}\EMINI\end{center}\vspace{2ex}\end{table}}

\def\bfltbl#1{\begin{table}[hbt]\vspace{2ex}\begin{center}\bmini\centering\btbl{#1}}
\def\efltbl#1#2{\etbl\caption[#1]{#2}\emini\end{center}\vspace{2ex}\end{table}}
\def\mcol{\multicolumn}
%
%  label equations with (#)
%
\def\reff#1{(\ref{#1})}
%
%  macros borrowed from viewgraph package
%

%\newenvironment{LETTRS}[3]{\begin{letter}{#1}
%\input{origin}\opening{Dear #2:}\input{#3}\closing{Sincerely yours,}\end{letter}}{\clearpage}

\newenvironment{VIEW}[1]{{\BC\Huge\bf #1 \EC}\LARGE\VS{.05in}}{\clearpage}

\def\RM#1{\rm{#1\ }}
\def\BV{\begin{VIEW}}
\def\EV{\end{VIEW}}

\def\NI{\noindent}

\def\VS{\vspace*}
\def\HS{\hspace*}
\def\IT{\item}

\def\BARR{\begin{array}}
\def\EARR{\end{array}}

\def\BPARR{\left(\begin{array}}
\def\EPARR{\end{array}\right)}

\def\BDET{\left|\begin{array}}
\def\EDET{\end{array}\right|}

\def\BDF{\begin{definition}}
\def\EDF{\end{definition}}

\def\BSU{\begin{block}{Summary}}
\def\ESU{\end{block}}

\def\BEX{\begin{example}}
\def\EEX{\end{example}}

\def\BTH{\begin{theorem}}
\def\ETH{\end{theorem}}

\def\BCO{\begin{corollary}}
\def\ECO{\end{corollary}}

\def\BPROOF{\begin{proof}}
\def\EPROOF{\end{proof}}

\def\BLM{\begin{lemma}}
\def\ELM{\end{lemma}}

\def\BEQ{\begin{equation}}
\def\EEQ{\end{equation}}

\def\BEQNNB{$$}
\def\EEQNNB{$$}

\def\BE{\begin{enumerate}}
\def\EE{\end{enumerate}}

\def\BD{\begin{description}}
\def\ED{\end{description}}

\def\BI{\begin{itemize}}
\def\EI{\end{itemize}}

\def\BC{\begin{center}}
\def\EC{\end{center}}

\def\BFIG{\begin{figure}}
\def\EFIG{\end{figure}}

\def\BTABB{\begin{tabbing}}
\def\ETABB{\end{tabbing}}

\def\BMINI{\begin{minipage}}
\def\EMINI{\end{minipage}}

\def\BTABLE{\begin{table}}
\def\ETABLE{\end{table}}

\def\BTABUL{\begin{tabular}}
\def\ETABUL{\end{tabular}}

\def\MCOL{\multicolumn}
\def\UL{\underline}
\def\ULL#1{\UL{\UL{#1}}}

\def\BDOC{\begin{document}}
\def\EDOC{\end{document}}

\def\EM#1{{\em #1\/}}
\def\FN{\footnote}

% Courtesy of Ugo Piomelli

\def\latexfig #1 #2 #3 #4 #5 {\ \vfill
\hfill\hbox to 0.05in{\vbox to #3truein{
         \special{psfile="#1" angle=270 hscale=100 
                  hoffset=#4 voffset=#5 vscale=100} }\hfill}
\hfill\vspace{-0.1in}        }

% #1 is the .ps filename
% #2 is not used in the present version
% #3 is the size of the white space left above the caption (in inches)
% #4 is the horizontal offset from some unknown reference point.
%    It is in 1/72 of an inch and is positive to the right.
% #5 is the vertical offset from some unknown reference point.
%    It is in 1/72 of an inch and is positive upwards.


	
% Speaker's name, Affiliation, Email address, Title of talk, Co-authors, 
% Abstract (preferably in text/plain TeX form, up to 150 words)

\title{Parallelizing the GMRES Algorithm on a Multi-CPU/GPU Cluster}

\author{Evan F. Bollig \\ bollig@scs.fsu.edu \\ Florida State University 
%\and Natasha Flyer \\ flyer@ucar.edu \\ NCAR 
%\and Gordon Erlebacher \\ gerlebacher@fsu.edu \\ Florida State University 
} 
%\and Joseph Lohmeier \\ lohmeier@ucar.edu \\ Boise State University}
%Florida State University, Dept. of Scientific Computation, Dirac Science 
%Library Tallahassee, FL 32304}
\begin{document}

\maketitle
%\tableofcontents
%\listoffigures
%\listoftables 

\section{Introduction}
This handout follows the implementation of my Parallel (Multi-GPU) GMRES algorithm. 

\section{TOREAD} 

http://www.irisa.fr/sage/recherche/sparsesolver.html

(consider Deflated GMRES for gpu. Doesnt appear to have been tested. Li and Saad point out that Deflated PCG was tested with success.)

\section{Related Work: Li and Saad 2011} 

Li and Saad demonstrated that for unstructured matrices SpMV kernels can be up to 10x faster than the CPU (Tesla C1060 vs Intel Xeon E5504). 
Their Incomplete Cholesky (special case of ILU) and Conjugate Gradient method on the GPU is up to 3x faster. 
An ILU preconditioned GMRES method can be up to 4x faster. 

\authnote{ THIS IS MY GOAL: at least 4x speedup for my ILU0 and GMRES for RBF-FD unstructured matrices} 

\section{Related Work: Bahi et al. 2011}
A multi-GPU implementation of GMRES was introduced in 2011 by Bahi, et al. \cite{Bahi2011}. In Algorithm~\ref{alg:gmres}, I have included the original pseudocode presented in \cite{Bahi2011} and {\color{blue}comments} for a discussion of what the algorithm does. 

\begin{algorithm}                      % enter the algorithm environment
\caption{Left-preconditioned GMRES with restarts}          % give the algorithm a caption
\label{alg:gmres}                           % and a label for \ref{} commands later in the document
\begin{algorithmic}[1]                    % enter the algorithmic environment
    \State $\varepsilon$ (tolerance for the residual norm $r$), $x_0$ (initial guess), and set $convergence = false$
    \While{ $convergence == false$}
    \State $r_0 = M^{-1} (b-Ax_0)$ \Comment{{\color{blue} They left off details of restarts, is the rest of their algorithm accurate?}}
    \State $\beta = ||r_0||_2$
    \State $v_1 = r_0 / \beta$ \Comment{ {\color{blue}This should really read: $v_1 = \frac{1}{\beta} r_0$)}}
	\For {$j=1$ to $m$} \label{alg:gmres_rotation_loop_start} \Comment{ {\color{blue}$m$ is the $restart$ input parameter ($\#$ of iters between restarts) } }
			\State $w_j = M^{-1} A v_j$
			\For {$i = 1$ to $j$}\label{alg:gmres_inner_loop_start}\Comment{{\color{blue}Key: we do $j$ iterations so the amount of work always grows }}  
				\State $h_{i,j} = (w_j, v_i)$ \Comment{{\color{blue}$(\cdot,\cdot)$ is an inner/dot product (usually $< \cdot, \cdot >$) }}
				\State $w_j = w_j - h_{i,j} v_i$
			\EndFor \label{alg:gmres_inner_loop_stop}
			\State $h_{j+1, j}  = ||w_j||_2$			\Comment{ {\color{blue}This $(m+1)\times m$ upper Hessenberg matrix must be stored }}
			\State $v_{j+1} = w_j / h_{j+1,j}$		\Comment{{\color{blue}This forms an orthonormal basis $V_m$ and must be stored ($N \times m$ matrix)}}
	\EndFor\label{alg:gmres_rotation_loop_stop} 
	\State Set $V_m = [v_1, \cdots, v_m]$ and $\bar{H}_m = (h_{i,j})$ an upper Hesssenberg matrix of order $(m+1)\times m$
	\State \label{alg:gmres_least_squares} Solve a least-square problem of size $m$: $\min_{y \in \R^m} ||\beta e_1 - \bar{H}_m y||_2$	\Comment{{\color{blue}Ambiguous. Will clarify.}}
	\State $x_m = x_0 + V_m y_m$ \label{alg:gmres_residual_norm}
	\If { $||M^{-1}(b-Ax_m)||_2  < \varepsilon$ }
		\State $convergence = true$
	\EndIf
	\State $x_0 = x_m$
    \EndWhile
\end{algorithmic}
\end{algorithm}

\subsection{Communication Points} 

In \cite{Bahi2011} the authors point out that communication happens at two types of points in the algorithm: \begin{enumerate} \item Before SpMV operations, and \item After vector operations \end{enumerate}. I have added markup to their pseudo-code in Algorithm~\ref{alg:gmres_comm} to illustrate where they perform communication. 

\begin{algorithm}                      % enter the algorithm environment
\caption{Left-preconditioned GMRES with restarts and {\color{red} communication poins} }          % give the algorithm a caption
\label{alg:gmres_comm}                           % and a label for \ref{} commands later in the document
\begin{algorithmic}[1]                   % enter the algorithmic environment
    \State $\varepsilon$ (tolerance for the residual norm $r$), $x_0$ (initial guess), and set $convergence = false$
    \While{ $convergence == false$}
    \State {{\color{red} MPI\_AlltoAllv() }}
    \State $r_0 = M^{-1} (b-Ax_0)$ 	
    \State $\beta = ||r_0||_2$		
	\State{{\color{red} MPI\_Allreduce() }}	 	
    \State $v_1 = r_0 / \beta$
	\For {$j=1$ to $m$} 
			\State {{\color{red} MPI\_AlltoAllv() }}
			\State $w_j = M^{-1} A v_j$			 	
			\For {$i = 1$ to $j$}
				\State $h_{i,j} = (w_j, v_i)$ 				
				\State{{\color{red} MPI\_Allreduce() }}
				\State $w_j = w_j - h_{i,j} v_i$
			\EndFor 
			\State $h_{j+1, j}  = ||w_j||_2$			
			\State{{\color{red} MPI\_Allreduce() }}
			\State $v_{j+1} = w_j / h_{j+1,j}$	
	\EndFor
	\State Set $V_m = [v_1, \cdots, v_m]$ and $\bar{H}_m = (h_{i,j})$ an upper Hesssenberg matrix of order $(m+1)\times m$
	\State Solve a least-square problem of size $m$: $\min_{y \in \R^m} ||\beta e_1 - \bar{H}_m y||_2$
	\State $x_m = x_0 + V_m y_m$
	\If { $||M^{-1}(b-Ax_m)||_2 < \varepsilon$ }
		\State $convergence = true$
	\EndIf
	\State{{\color{red} MPI\_Allreduce() }}
	\State $x_0 = x_m$
    \EndWhile
\end{algorithmic}
\end{algorithm}

\subsection{Practical Implementations}

In Algorithm~\ref{alg:gmres}, line~\ref{alg:gmres_least_squares} ambiguously states we need to solve a least-squares problem. 

%This ambiguity is also present in Saad's book (see e.g., page 165, \cite{Saad2003}) although the authors return on page 169 to discuss Givens rotations and other useful details for ``Practical Implementation". This section summarizes a few of those details. 

We could take the algorithm as is and express the least-squares problem as 
\begin{equation} 
\bar{H}_m^T \beta e_1 - \bar{H}_m^T \bar{H}_m y = 0 \label{eq:least_squares}
\end{equation}
for a cost of O($m^3$) operations on line~\ref{alg:gmres_least_squares}. If $m$ is small this is not a concern. However, it is common practice instead to find the implicit QR decomposition of $\bar{H}_m$ to reduce the complexity to O($m^2$). 

Saad and Schultz \cite{Saad1986} detail an efficient method of computing plane rotations at the end of each inner-loop iteration (i.e., just before line~\ref{alg:gmres_rotation_loop_stop} of Algorithm~\ref{alg:gmres}). Their method computes one Givens rotation at a cost of O($m$) per iteration, such that at the end of the loop, the QR factorization is implicitly formed with an overhead of O($m^2$). With the QR factor, the minimization problem transforms to 
$$\min_{y \in \R^m} ||Q^T (\beta e_1 - \bar{H}_m y) ||_2 = \min_{y \in \R^m} ||g_m - R_m y||_2$$
where $R_m$ is the upper triangular $m\times m$ factor of $\bar{H}_m$ from the QR process, and $g_m = Q^T \beta e_1$ has been updated at each iteration. Now, using the knowledge that the 2-norm of a vector is invariant under orthogonal transformations (i.e., the matrix $Q^T$), our minimization problem reduces to solving $$ R_m y = g_m $$ which can be done with a simple upper triangular back solve. 

Saad and Schultz \cite{Saad1986} also demonstrate that by using the implicit QR factorization, the residual norm of the approximate solution $x_m$ is available at every iteration without explicitly forming $x_m$ on Line~\ref{alg:gmres_residual_norm}. Instead, due to the Givens rotations of system, the residual norm is always available in the $(k+1)$-st component of $g_k$. The explicit formation of $x_m$ involves the product of $N \times m$ matrix $V_m$ and $N$ length vector $x_m$. Therefore, using the implicit residual norm and postponing the formation of our solution saves a significant O($Nm$) operations per iteration. 

As the first multi-GPU GMRES paper, Bahi et al. \cite{Bahi2011} do not specify if they use Givens rotations for efficiency. The description of their algorithm states one thread on the GPU takes responsibility in Algorithm~\ref{alg:gmres} at line~\ref{alg:gmres_least_squares} and solves the size-$m$ least squares problem. This leads me to believe they solve Equation~\ref{eq:least_squares}. They also mention a kernel that forms a local view of $x_m$ on line~\ref{alg:gmres_residual_norm}, which further argues against soliciting the help of the implicit QR. \authnote{THIS COULD BE A NEW CONTRIBUTION. Do the plane rotations require additional synchronization? Do the plane rotations impact convergence? Is there a reason to avoid plane rotations stated in other parallel GMRES algorithms publications?} 


\section{Related Work: Dekker 2000}
In 2000, Dekker \cite{Dekker2000} published a report describing the Parallel GMRES method. His approach was to apply an additive schwarz decomposition to the standard GMRES (with Givens rotations) and then add a few tweaks to increase scalability. The punch line of his method is that it increases computational overhead per processor in order to reduce the overall communication time. This is ideal for GPUs where the cost of communication includes both CPU-CPU and two way CPU-GPU transfers. I suspect this is the approach we want to take. However, in Dekker's subsequent papers (see \cite{Dekker2001}, \cite{Dekker2005}) he mentions a requirement for his convergence studies to have red-black ordering on the nodes. This is not possible with RBF-FD as far as I know. Red-black ordering would require us to partition every other node into red and the rest in black. This can be extended to more than two colors, but I'm not sure our connectivity would allow for the separation in matrix blocks that he leverages in the latter papers. \cite{Dekker2000} does not appear to have this requirement but I need more exposure with the method to be certain.  


\section{CUSP Implementation} 

To implement my own multi-GPU Parallel GMRES, I chose to leverage the existing code-base of libraries like ViennaCL and CUSP. Starting with our existing physical domain partitioning, we can drop in our calls to ``sendrecvUpdates(...)" for the MPI\_AlltoAll calls. Then we just need to add an MPI\_Allreduce replacement, and we should be nearly finished. 

The one aspect of this parallel implementation that I'm concerned with is the least-squares solution. In CUSP and ViennaCL, the implicit rotations are used to accelerate the least squares step. If this cannot span multiple GPUs---so far the discussion is that its high in communication cost, but nothing saying its impossible---then I will have to re-work more of their algorithms. 

I'll start by 

\section{Conclusion}


\bibliographystyle{acm}
\bibliography{references}

\end{document}
