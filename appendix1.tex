%!TEX root = karen.tex
\chapter{Avoiding Pole Singularities with RBF-FD}
This content follows \cite{FlyerWright07}. 

Within the test cases of this dissertation, we solve convective PDEs on the unit sphere with the form: 
$$
\pd{h}{t} = \vu \cdot \nabla h
$$
where $\vu$ is velocity. For example, the cosine bell advection has this particular form:
\begin{equation}
\pd{h}{t} = \frac{u}{\cos{\theta}} \pd{h}{\lambda} + v \pd{h}{\theta} \label{eq:cosine_bell_appendix}
\end{equation}
in the spherical coordinate system defined by
\begin{align*}
x & = \cos{\theta}\cos{\lambda} \\
y & = \cos{\theta}\sin{\lambda} \\
z & = \sin{\theta}
\end{align*}
where $\theta \in (-\frac{\pi}{2}, \frac{\pi}{2})$ is the elevation angle and $\lambda \in (-\pi,\pi)$ is the azimuthal angle.
Observe that as $\theta \rightarrow \pm \frac{\pi}{2}$, the $\frac{1}{\cos{\theta}}$ term goes to infinity and causes a discontinuity. 

One of the many selling points for RBF-FD and other RBF methods is their ability analytically avoid pole singularities, which arise from the choice of coordinate system and not the choice of numerical method. 
Here we demonstrate how to analytically avoid the pole singularities with RBF-FD.  


Let $r = || \vx - \vx_j ||$ be the Euclidean distance which is invariant of the coordinate system. In Cartesian coordinates, we have
$$
r = \sqrt{(x-x_j)^2 + (y-y_j)^2 + (z-z_j)^2}.
$$
In spherical coordinates we have:
$$
r = \sqrt{2(1-\cos{\theta}\cos{\theta_j}\cos{(\lambda-\lambda_j)} - \sin{\theta}\sin{\theta_j})}.
$$

The RBF-FD operators for $\d{}{\lambda}, \d{}{\theta}$ can be discretized with the chain rule: 

\begin{align}
\d{\phi_{j}(r)}{\lambda} = \d{r}{\lambda} \d{\phi_{j}(r)}{r} & = \frac{\cos{\theta}\cos{\theta_j}\sin{(\lambda - \lambda_j)}}{r} \d{\phi_j(r)}{r}, \label{eq:rbfffd_d_dlambda} \\
\d{\phi_{j}(r)}{\theta} = \d{r}{\theta} \d{\phi_{j}(r)}{r} & = \frac{\sin{\theta}\cos{\theta_j}\cos{(\lambda-\lambda_j)} - \cos{\theta}\sin{\theta_j}}{r} \d{\phi}{r}, \label{eq:rbfffd_d_dtheta}
\end{align}
where $\phi_{j}(r)$ is the RBF centered at $\vx_{j}$. 

Plugging \ref{eq:rbfffd_d_dlambda} and \ref{eq:rbfffd_d_dtheta} into \ref{eq:cosine_bell_appendix}, produces the following explicit form: 
\begin{align*}
\d{h}{t} = u(\cos{\theta_j}\sin{(\lambda - \lambda_j)} \frac{1}{r} \d{\phi_j}{r}) + v (\sin{\theta}\cos{\theta_j}\cos{(\lambda-\lambda_j)} - \cos{\theta}\sin{\theta_j} \frac{1}{r} \d{\phi}{r}) 
\end{align*}
where $\cos{\theta}$ from \ref{eq:rbfffd_d_dlambda} analytically cancels with the $\frac{1}{\cos{\theta}}$ in \ref{eq:cosine_bell_appendix}.


Then, formally, within the code one would compute weights and assemble the differentiation matrices: 
\begin{align}
\D_\lambda & = \cos{\theta_j}\sin{(\lambda - \lambda_j)} \frac{1}{r} \d{\phi_j}{r}, \label{eq:dm_lambda} \\
\D_\theta &=  \sin{\theta}\cos{\theta_j}\cos{(\lambda-\lambda_j)} - \cos{\theta}\sin{\theta_j} \frac{1}{r} \d{\phi}{r}, \label{eq:dm_theta}
\end{align} 
and solve the explicit method of lines problem:
$$
\d{h}{t} = u \D_\lambda h + v \D_\theta h
$$
where now the system is completely free of singularities at the poles. 

We note that the expression $\cos{(\frac{\pi}{2})}$ evaluates on some systems to a very small number rather than zero (e.g., $6.1(10^{-17}$) on the Keeneland system and GNU gcc toolchain). The small value in turn allows $\frac{1}{\cos{\theta}}$ to evaluate to a large value (e.g., $1.6(10^{16})$) rather than ``inf" or ``NaN". The large value allows the cosine terms to cancel, whereas an ``inf'' or ``NaN'' would corrupt the solution. One could write general code to compute weights for \ref{eq:rbfffd_d_dlambda} and solve the PDE if the computation is not corrupted, but it is better practice to use the specific operators \ref{eq:dm_lambda}, \ref{eq:dm_theta}.