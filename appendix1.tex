\chapter{Avoiding Pole Singularities with RBF-FD}

This content is a summary of \cite{FlyerWright07}. 

Let $r = || x - x_j ||$ be the Euclidean distance which is invariant of the coordinate system. In Cartesian coordinates, we have
$$
r = \sqrt{(x-x_j)^2 + (y-y_j)^2 + (z-z_j)^2}.
$$
For the spherical coordinate system defined by
\begin{align}
x = \cos{\theta}\cos{\lambda}
y = \cos{\theta}\sin{\lambda} 
z = \sin{\theta}
\end{align}
where $\theta \in (-\frac{\pi}{2}, \frac{\pi}{2})$ is the elevation angle and $\lambda \in (-\pi,\pi)$ the azimuthal angle, we have
$$
r = \sqrt{2(1-\cos{\theta}\cos{\theta_j}\cos{\lambda-\lambda_j} - \sin{\theta}\sin{\theta_j})}
$$

\authnote{Finish: 1) $\grad$ operator in spherical. 2) the chain rule. 3) then finish with 

Note that in the code if one computes the \emph{true weights} to approximate $\d{}{\lambda}$, then they have an explicit system that takes the form: 

$$
\d{h}{t} = \frac{u}{\cos{\theta}} ( \cos{\theta}\cos{\theta_j}\sin{\lambda - \lambda_j} \frac{1}{r} \d{\phi_j}{r}) + v (\sin{\theta}\cos{\theta_j}\cos{\lambda-\lambda_j} - \cos{\theta}\sin{\theta_j} \frac{1}{r} \d{\phi}{r})
$$

\authnote{In the computer the expression $\frac{1}{\cos{\theta}$ evaluates to a very large number ($1*10^32$) but is not ``inf" or ``NaN". The large value allows the $\cos{\frac{\pi}{2}}$ in the operator to null out the term rather than appear as a singularity. The latter two values (``inf" and ``NaN" would not null out, but instead corrupt the solution. Therefore, while it functions to write code generally following Eq\ref{need_ref_here}, it is better practice to analytically remove the singularity and evaluate the system as: }

The proper way to solve this in code, however, is to analytically cancel the $\frac{1}{\cos{\theta}}$ in order to avoid singularities at the pole, so we are left with the following weight operators: 

\begin{align}
D_\lambda & = \cos{\theta_j}\sin{\lambda - \lambda_j} \frac{1}{r} \d{\phi_j}{r} \\
D_\theta &=  \sin{\theta}\cos{\theta_j}\cos{\lambda-\lambda_j} - \cos{\theta}\sin{\theta_j} \frac{1}{r} \d{\phi}{r}
\end{align} 

and then write the explicit solver to evaluate: 

$$
\d{h}{t} = u D_\lambda h + v D_\theta h
$$

where now the system is completely free of singularities at the poles. 
