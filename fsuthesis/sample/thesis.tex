%%% This is a sample thesis showing how to use the FSU Thesis Class.
%%% This file is the main LaTeX source file.  It contains all the
%%% front matter and back matter of the document.  The main text
%%% (chapters and appendices) are '\include'd from this file.  Lines
%%% that begin with a '%' are "comments", and they are completely
%%% ignored by LaTeX.

%%% The document begins by declaring the document class and additional
%%% features that we'll need.

\documentclass[11pt]{fsuthesis}

%%% I'll be using some of the definitions from the AMS math package,
%%% so I include a line for that here.  LaTeX has many optional
%%% packages which may be included here to add useful features to your
%%% document.  "The LaTeX Companion" is a great resource for delving
%%% into some advanced formatting and functionality among LaTeX's many
%%% packages.

\usepackage{amsmath}
\usepackage{mflogo}         % (not needed for general use)
\usepackage[overload]{textcase}
%\usepackage[round]{natbib}  % an alternative bibliography package

%%% The following few lines are a little different: if I want to
%%% typeset my thesis using pdflatex, I need to use the 'pdftex'
%%% driver to insert figures, and I'll also turn on hyperlinks in my
%%% document.  If instead I'll be producing a DVI or PostScript
%%% document (which I can convert to PDF later), I will want to use
%%% the 'dvips' driver for the graphicx package, and I won't turn on
%%% hyperlinks.  I can accomplish this by testing whether or not I'm
%%% running in PDF mode using the '\ifpdf' test below.  If I'm in PDF
%%% mode, then all my figures will need to be either PDF files, PNG
%%% files, or JPEG files.  If I'm using the 'dvips' driver, then any
%%% figures I include should be EPS (Encapsulated PostScript).  (This
%%% sample document provides figures in both formats.)

\ifpdf   % We will execute this part if we are in PDF mode
  \usepackage[pdftex]{graphicx}
  \usepackage[colorlinks=true,bookmarks=true,%
     pdfborder={0 0 0},linkcolor=blue,urlcolor=red,%
     breaklinks=true,bookmarksnumbered=true]{hyperref}
\else    % otherwise we'll run this part if we are not in PDF mode
  \usepackage[dvips]{graphicx}
\fi

%%% The following lines increase LaTeX's default penalties for club
%%% lines and widow lines, which will discourage creating clubs and
%%% widows in all but the most extreme cases.  (If the last line on a
%%% page is the first line of a new paragraph, it's called club line.
%%% If the first line of a page is the last line of a paragraph, it
%%% is called a widow line.  Either of these conditions look a little
%%% ugly, typographically speaking.)

\widowpenalty=9999
\clubpenalty=9999

%%% Normally, the fsuthesis class will try to stretch the page content
%%% to fill the whole page, unless it's the end of a chapter.
%%% Occasionally you may find that pages with a lot of figures,
%%% equations, or tables may cause ugly gaps.  Should this happen, you
%%% could try setting \raggedbottom.  (It's commented out here.)

%\raggedbottom

%%% Next, we set up some of the document description elements for use
%%% on the title and committee pages.  If the college or school does
%%% not sub-divide into departments, then you should delete the
%%% '\department{}' macro below.

\title{This Is My Thesis Title:\protect\\Broken over Two Lines}
                                      %% Manuscript title
\author{V\`\i ct\"or \'E. I\c smyne}  %% Testing accented characters in name
\college{College of Arty Science}     %% School or College
\department{Department of Furtive Studies}  %% Delete if no department
\manuscripttype{Thesis}               %% [Thesis, Dissertation, Treatise]
\degree{Master of Science}            %% Name of the degree
\semester{Summer}                     %% Graduation Semester:
                                      %%   [Fall, Spring, Summer]
\degreeyear{2011}                     %% Graduation Year
\defensedate{June 15, 2011}           %% Date of Defense

%%% The following options insert additional metadata into a PDF file if
%%% you are using pdflatex to process your document.  This information
%%% is not printed in the paper, but the information will be available
%%% in the document properties tab of Acrobat Reader (for example).
%%% The document title and the author name are already inserted
%%% automatically.  Metadata can help search engines locate your
%%% document in searches.  The "subject" should be a concise
%%% description of the paper's topic. The "keywords" should be
%%% relevant terms that would help other researchers in your field
%%% locate your paper.  Keywords should be separated by semicolons.

\subject{Dissertation Formatting}     %% PDF document metadata subject
\keywords{latex; fsuthesis; etd; tables; figures; bibtex; document formatting}
				      %% PDF document metadata search
                                      %%   keywords, separated by semicolons

%%% Assuming I have four professors on my committee, I will need four
%%% lines here, one for each committee member.  The professors should
%%% be specified in the order in which they should appear on the
%%% committee page.  The first argument of the \committeeperson macro
%%% is the professor's name (without title), and the second argument
%%% is the professor's committee membership position.

\committeeperson{Faux Causson Yorverk}{Professor Directing Thesis}
\committeeperson{Verda Boizaar}{University Representative}
\committeeperson{Beauxeau D'Claune}{Committee Member}
\committeeperson{Arlip Zarseeld}{Committee Member}
%\committeeperson{Tumen Enoats}{Committee Member}
\committeeperson{Deb O'Nair}{Committee Member}
%\committeeperson{Auss M'Pauers}{Committee Member}

%%% The initial document setup is done.  Now we tell LaTeX that our
%%% text will begin.

\begin{document}

\frontmatter          %% Establish small roman numeral numbering
\maketitle            %% Create the title page
\makecommitteepage    %% Create the committee page

%%% If I wish to have a dedication, then I will use the dedication
%%% environment.  This is optional, so if you don't need a dedication
%%% page, you may delete everything between (and including)
%%% the \begin{dedication} and \end{dedication} text.  The format of
%%% the page is entirely up to you.

\begin{dedication}
\centering
To my parents, who always suspected I'd end up here
\end{dedication}

%%% Likewise, I use the acknowledgments environment to create an
%%% acknowledgments page.  Delete this section if there will be no
%%% acknowledgments page.

\begin{acknowledgments}
Many thanks are due to many people.  My major professor didn't
know what she was getting herself into when she took me on as
a student, and I will always be grateful for her support and
guidance.  The other members of my committee deserve hazard pay,
and this paper would not be the same without their diligence:
many thanks.
\end{acknowledgments}

%%% I will need all the contents here except the list of musical
%%% examples.  (These pages will be generated automatically by LaTeX
%%% as it encounters headings and table and figure environments later
%%% in the document.)  I only need to include a figure or table list
%%% if I have more than one figure or table.

\tableofcontents
\listoftables
\listoffigures
%\listofmusex

%%% I would like to present a short list of symbols.  By using the
%%% listofsymbols environment, the proper entry will be made in the
%%% Table of Contents.  Other than the heading, the rest of the page
%%% is entirely up to me.  I will use a LaTeX tabular environment to
%%% list the data.

\begin{listofsymbols}
The following short list of symbols are used throughout the document.
The symbols represent quantities that I tried to use consistently.
\begin{center}
\begin{tabular}{ll}
$\pi$&$3.1415926\ldots$\\
$E$&$mc^2$\\
$F$&$ma$\\
$R_e$&Mean Radius of the Earth${}\approx 6367.65\,\textup{km}$\\
$e$&Base of Natural Logarithms${}\approx 2.71828\ldots$\\
$P$&The principal borrowed\\
$N$&The number of payments\\
$i$&The fractional (periodic) interest rate\\
$P_j$&The principal part of payment $j$\\
$I_j$&The interest part of payment $j$\\
$B$&A final balloon payment\\
$x$&The regular payment\\
$R$&The principal remaining after $r$ payments \\
$r$&Some number of payments such that $0 < r < N$ \\
$R_j$&The principal remaining after $j$ payments \\
$A_j$&The total interest paid out after $j$ payments
\end{tabular}
\end{center}
\end{listofsymbols}

%%% You may also create a list of abbreviations if it may be of use to
%%% your readers.  Otherwise, this section may be deleted or remain
%%% commented.  Also, if you need another frontmatter environment for
%%% some other use (besides a list of abbreviations or symbols), you
%%% could rename the heading, though the environment name remains the
%%% same:

%\renewcommand*{\listabbrevname}{Index of Scary Movies}
%\begin{listofabbrevs}
%  [... insert scary movie index material here ...]
%\end{listofabbrevs} 

%%% Now I use the abstract environment to create an abstract page.
%%% Except for the heading, the rest of the page contents are up to
%%% me.

\begin{abstract}
The FSU Thesis Class is a \LaTeX{} document class useful for writing
Theses, Dissertations, and Treatises.  It has several custom macros
and environments which are intended to ease the burden of formatting
for writers of these documents so that they may focus more on the
research and presentation rather than on the page layout. This sample
document is intended to provide a few examples of how most of the
class features may be used.

The main source file for this document is \texttt{thesis.tex}, and
this is where you should start reading.  The document's source is
spread over several files.  Many of the files contain helpful \LaTeX{}
comments which are not printed out here.  It may be instructive to
look at the source files as you read this ``output'' to see how the
document was created.
\end{abstract}

%%% The abstract is the last element of the so-called "front matter"
%%% of the document.  We now move on to the "main matter", the
%%% chapters (and optional appendices).  By calling the \mainmatter
%%% macro, we reset page numbering to arabic numerals beginning with
%%% '1'.

\mainmatter

%%% Here I am defining some additional LaTeX macros for use throughout
%%% my document.  These are intended to simplify some typing and to
%%% encourage uniformity, since I won't always have to remember, for
%%% example, that acronyms should be in a smaller ALLCAPS font: I can
%%% just type '\acro{text}'.  Note that though I am defining these
%%% macros here, they will be accessible to any other source text
%%% which follows, even if the source text is in a completely
%%% different file that I'll be \include-ing later. You are free to
%%% define your own macros in a similar fashion.

\newcommand*{\acro}[1]{{\small\textsc{#1}}}
\newcommand*{\lit}[1]{\texttt{#1}}
\newcommand*{\pkg}[1]{\textsf{#1}}

%%% I have chosen to create one new LaTeX file for each chapter of my
%%% document.  This decision is arbitrary.  I may break the document
%%% into as many or as few pieces as I like.  I may name the chapter
%%% and appendix files anything I like (though I might have some
%%% trouble if I use spaces or odd characters in my file names).  The
%%% only requirement is that the file name ends with the extension
%%% '.tex'.  For the \input lines below, I can leave out the '.tex'
%%% extension.  If instead I prefer to create one long file containing
%%% my entire thesis, I may just continue on here by starting with the
%%% first \chapter command of my document.  (In this case, I would not
%%% need the following '\input' commands at all.)

\input chapter1
\input chapter2
\input chapter3
\input chapter4

%%% Having finished the main text of my document, I will move on to
%%% the appendices (or my one appendix, in this case).  An appendix
%%% will look exactly like a chapter, even including a \chapter macro
%%% call.  By first calling the \appendix macro, I will cause all
%%% subsequent chapter headings to be labeled "Appendix" and to be
%%% enumerated by letters rather than numbers.

\appendix
\input appendix1

%%% At this point, we are moving into the document's "back matter".
%%% The first element is the references/bibliography section.
%%% According to the FSU "Guidlines & Requirements for Thesis,
%%% Treatise, and Dissertation Writers", this section should be called
%%% "References" if it contains only source material cited in the
%%% document, and it should be called "Bibliography" if it contains a
%%% broader scope of material than that actually cited.  Because this
%%% document only lists cited materials, I will need to be sure the
%%% heading is labeled "References".  The FSU Thesis Class "User
%%% Guide" tells me to use \renewcommand to accomplish this:

\renewcommand*{\bibname}{References}

%%% This document assumes that I will be using BiBTeX to format my
%%% references.  I'll be using the 'plain' format style, and the
%%% bibliographic data are kept in the file 'myrefs.bib'.  (I leave
%%% off the file extension in the \bibliography command below.)  There
%%% are many different bibliography citation and formatting styles
%%% available, though any particular LaTeX installation may not have
%%% all of them resident.  A quick web search with 'bibtex' and your
%%% own discipline as keywords may help you locate both references and
%%% the appropriate style files for you to download.

\bibliographystyle{plain}
%\bibliographystyle{plainnat} % If using 'natbib', use this line instead
\bibliography{myrefs}

%%% The last element of the document is the biographical sketch.  The
%%% heading and table of contents entry are automatically created by
%%% using the biosketch environment.

\begin{biosketch}
The author was born, and then the author was ``educated,'' at least to
some degree.  After finishing high school in Florida, the author
completed a Bachelor of Arts degree at The Florida State University.
Following a decade in the work force in his discipline, the author
returned to FSU to pursue graduate work.
\end{biosketch}

%%% Here endeth the document.  LaTeX will ignore anything that follows
%%% the \end{document} command.

\end{document}
