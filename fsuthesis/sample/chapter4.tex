\def\BiBTeX{{\rmfamily B\kern-.05em%
   \textsc{i\kern-.025em b}\kern-.08em\TeX}}

\newenvironment{narrowref}{\par
  \frenchspacing
  \setlength{\leftskip}{4.5em}
  \setlength{\rightskip}{3em}
  \setlength{\parindent}{-1.5em}
  \setlength{\parskip}{.6\baselineskip}}%
 {\par\nonfrenchspacing\addvspace{.5\baselineskip}}

\chapter{Citations and References}
\section{Using the \texttt{references} Environment}
Let's start simple.  Assuming we don't have many citations, we'll
create a references section manually using the
\verb|\begin{references}| environment provided by the
\textsf{fsuthesis} class.  If your discipline has a style
guide for presenting references, you can use that to create your own
entries.  Here are some example entries with explicit formatting
specified:
\begin{verbatim}
\begin{references}
Picaut, J., F. Masia, and Y. du Penhoat, 1997: An advective-reflective
conceptual model for the oscillatory nature of the ENSO.
\textit{Science}, \textbf{277}, 663--666.

Yasunari, T., 1990: Impact of Indian monsoon on the coupled
atmosphere/ocean system in the tropical Pacific.
\textit{Meteor. Atmos. Phys.}, \textbf{44}, 19--41.
\end{references}
\end{verbatim}

Note that I had to specify italics and bold-facing myself, according
to the style guide that I'm using.  (If you don't have a
discipline-specific style guide, see \cite{Anonymous:1993:CMS}
or \cite{Anonymous:2009:PMA} for lots of bibliographic examples.)
The \lit{references} environment provides exdented entries, spacing,
and a heading, but the rest of the formatting is up to you.  Likewise,
citations of these references within your document must be formatted
manually.  Once processed, the reference entries above look like the
following:
\begin{narrowref}
Picaut, J., F. Masia, and Y. du Penhoat, 1997: An advective-reflective
conceptual model for the oscillatory nature of the ENSO.
\textit{Science}, \textbf{277}, 663--666.

Yasunari, T., 1990: Impact of Indian monsoon on the coupled
atmosphere/ocean system in the tropical Pacific.
\textit{Meteor. Atmos. Phys.}, \textbf{44}, 19--41.
\end{narrowref}

\section{Citations and \BiBTeX}
If your thesis or dissertation does not have many citations or
references, this may be all you need. However, if you have more than a
handful of citations, you owe it to yourself to invest a little more
energy into learning about the powerful, time-saving features of
\BiBTeX.  To use \BiBTeX, bibliographic entries are added to an
external file with tags identifying elements of the entry, such as
authors, titles, journals, etc.  You may then cite a reference in
your document using its unique key.  Many disciplines have developed
large \BiBTeX{} databases already, so if you're lucky, you only need
download a pre-built file ready to go.  You can always add a few more
references if those entries don't already exist in the file you
download.

For this sample document, I have created a small \BiBTeX{}
bibliography database in a file called \lit{myrefs.bib} which is
excerpted from a larger collection of pre-generated \TeX-related
entries I downloaded from the web.  If your discipline does not
already distribute \BiBTeX{} databases publicly, it's just a bit
more typing to create your own \BiBTeX{} file.  Another advantage of
the \BiBTeX{} approach is that you can continue to add to this
database throughout your professional career, creating entries as you
read books and journal articles you may want to reference in the
future.  I encourage you to refer to the standard \LaTeX{} references
(e.g., \cite{Lamport:1994:LDP} and \cite{Kopka:2004:GLT}) or to search
the web for further information on creating your own \BiBTeX{}
database.

Once you've created or downloaded a \BiBTeX{} database, you may cite a
document using its unique \textit{key} as an argument to
the \LaTeX{} \verb|\cite| macro.  For example, in the previous
paragraph, I used the commands \verb|\cite{Lamport:1994:LDP}| and
\verb|\cite{Kopka:2004:GLT}|, where \lit{Lamport:1994:LDP} and
\lit{Kopka:2004:GLT} are the keys for their respective documents.
By making these citations, the bibliographic entries for these
documents will be pulled from my \BiBTeX{} file and added to the
bibliography automatically.  I can also add entries to the
bibliography without having citations in my text by using
the \verb|\nocite{key}| command with the desired \textit{key}.
If I want every entry in my \BiBTeX{} file inserted into my
bibliography, then I can issue the command \verb|\nocite{*}|
as a shortcut.  (You don't have to worry about citations being
redundant: bibliographic entries will be included only once no matter
how many times they may be \verb|\cite|d or \verb|\nocite|d in your
document.)

This document uses the default citation and bibliography formatting
supplied by \LaTeX, but if your discipline publishes \BiBTeX{} files
for common use, it may also have created its own \BiBTeX{} formatting
style which you may want to use.  One common alternative is to use
the \pkg{natbib} package activated by
including \verb|\usepackage{natbib}| in the document preamble.  This
package will allow you to make author--year citations in your document
automatically (e.g., ``Fargunkle et al., (2001)'') using the
same \BiBTeX{} database file.  See \cite{Kopka:2004:GLT} (or search
the web) for more information on \pkg{natbib}.
