\documentclass[11pt,letterpaper]{article}

\title{The \textsf{fsuthesis} \LaTeX{} Class:\\
  A User's Guide}
\author{Bret D. Whissel\\Department of Earth, Ocean and Atmospheric
  Science\\Florida State University}
\date{August 23, 2011}

\newcommand*{\acro}[1]{{\small\textsc{#1}}}
\newcommand*{\booktitle}[1]{\textit{#1}}
\newcommand*{\complit}[1]{\texttt{#1}}
\newcommand*{\latexclass}[1]{\textsf{#1}}

\newcommand*{\fsuth}{\latexclass{fsuthesis}}

\renewcommand{\-}{\discretionary{}{}{}}

\def\BibTeX{{\rmfamily B\kern-.05em%
   \textsc{i\kern-.025em b}\kern-.08em\TeX}}

\setcounter{tocdepth}{2}


\begin{document}

\maketitle

\tableofcontents

\section{Introduction to \LaTeX}

If you are already a \TeX/\LaTeX{} convert, you may skip over this
introductory material and jump ahead to the description of the
\fsuth{} class macros in section~\ref{sec:Class}.  If you're new to
\LaTeX, you may want to learn a little bit more about what you may be
getting yourself into first.

If you have grown up only learning to use the word-processing tools
that are installed on a typical PC, \LaTeX{} may feel awkward at
first.  However, \LaTeX's ability to generate cross-references, lists
of tables and figures, and a table of contents---automatically---is
already worth the small amount of effort required to get started with
this very powerful typesetting system.  Further, if your document
contains mathematics, you'll be hard-pressed to find better software
for making equations look good in type.

Historically, \LaTeX{} is not a \acro{WYSIWYG}\footnote{pronounced
  ``wizzywig'': What You See Is What You Get} system.  Instead,
documents are created using any available plain text editor.  When
ready, the document is run through \LaTeX{} to produce viewable or
printable output.  This two-step process may be different from what
you're used to, but one advantage is that it allows authors to focus
more on the content of their documents, and to focus less on the
formatting (or at least to defer the attention to formatting until the
final stages of document preparation).

\section{Installation}

\subsection{The \LaTeX{} System}

The \LaTeX{} system (and the \TeX{} engine upon which it is built) may
require some time to get installed and running.  But it is completely
free software, and there are lots of resources for helping you to get
started.  These are much more comprehensive than this
\booktitle{User's Guide} can be.

If you are working in a Microsoft Windows environment, take a look at
the MiK\TeX{} project (see \complit{http://miktex.org}).  Mac users
will find the \textsf{Mac}\TeX{} resources useful (see
\complit{http://www.tug.org/mactex}).  Linux/\acro{UNIX} users should
investigate the \TeX{}\,Live distribution, if \TeX{} is not already a
part of your installation (see \complit{http://www.tug.org/texlive}).

\subsection{Plain Text Editor}

In addition to the \TeX/\LaTeX{} system, you will need a plain text
editor.  In a Windows environment, \complit{notepad} is all that's
required.  You can use more sophisticated editors, of course, as long
as the editor will save your document as a plain text file.  Text
editors are available on just about any modern computer platform, many
of them free and high-quality.  Linux/\acro{UNIX} enthusiasts will
probably already have access to and familiarity with \complit{emacs}
or \complit{vim}, both of which have been ported to Windows and Mac
environments.  There exist more advanced document development
environments for \LaTeX{} which include document previews and
\acro{WYSIWYG} functionality.

\subsection{Installing the \fsuth{} Class File}

The \fsuth{} class is packaged and distributed as a zip file.  When
the zip file is unpacked, a folder called \complit{fsuthesis} is
created.  In that folder will be found this \booktitle{User's Guide},
in both its \acro{PDF} and \LaTeX{} source form, as well as a few
other files, the \complit{thesis-template} folder, and a
\complit{sample} folder.

The \fsuth{} class file is called \complit{fsuthesis.cls}.  If you use
the directory \complit{thesis-template} as a starting point, the
\complit{fsuthesis.cls} file is already unpacked there and ready to
use.  You can copy the \complit{thesis-template} directory to a new
location and generate your document within that folder as a
self-contained entity.  No further installation is necessary.

Alternatively, you may install the \fsuth{} class file for system-wide
or permanent use.  To generate the class file, run the following
command:
\begin{verbatim}
latex fsuthesis.ins
\end{verbatim}
This operation extracts the class file from the file
\complit{fsuthesis.dtx} (which contains documentation of interest to
future maintainers of the \fsuth{} class).  Then copy the
\complit{fsuthesis.cls} file into the \LaTeX{} file search tree.  (The
proper location is operating system and installation-dependent.  For
\acro{UNIX}/Linux systems, this location might be something like
\complit{/usr/\-share/\-texmf-site/\-tex/\-latex/\-fsuthesis/}.)

If you're interested in modifying the \fsuth{} class, you may want to
read the source code's documentation.  To do this, run the following
sequence of commands:
\begin{verbatim}
latex fsuthesis.dtx 
makeindex -s gglo.ist -o fsuthesis.gls fsuthesis.glo
makeindex -s gind.ist -o fsuthesis.ind fsuthesis.idx
latex fsuthesis.dtx 
\end{verbatim}
Be sure to document your changes to the file by editing the
\complit{fsuthesis.dtx} file, not the \complit{fsuthesis.cls} file, as
changes to the latter file can be overwritten if the class file is
re-extracted.  (In the steps above, you may run \complit{pdflatex}
instead of \complit{latex} to generate a \acro{PDF} version of the class file
documentation directly.)

\section{Helpful \LaTeX{} References}

For simple texts, you might not need more from \LaTeX{} than what's
described in this \booktitle{User's Guide}.  For more complicated
texts, however, or for documents containing several tables, figures,
or mathematics, you will certainly want to supplement your \LaTeX{}
references.  You will find a wealth of information on-line using your
favorite web search engine, as well as several bound and printed
reference materials.  I have found the texts cited below to be of
particular value.
\begin{itemize}
\item
For first-timers, \booktitle{The Not So Short Introduction to \LaTeXe}
by Tobias Oetiker, Hubert Partl, Irene Hyna, and Elisabeth Schlegl
promises to have you off and running in a few hours' time.  It's a
document you may find readily on-line in \acro{PDF} form.

\item
The standard reference is the book \booktitle{\LaTeX: A Document
  Preparation System}, 2nd~Ed., by Leslie Lamport, the original author
of \LaTeX{}.  This text covers all the basics clearly and succinctly.

\item
A larger starting reference book is \booktitle{Guide to \LaTeX},
4th~Ed., by Helmut Kopka and Patrick W.~Daly.  At twice the length of
the Lamport book, \booktitle{Guide} covers all the basics, and it also
touches on a few of the more common add-on packages.  The book
comes with a CD-ROM with the \TeX\,Live distribution included, which
can save you a lot of downloading time.

\item
Once your working knowledge of \LaTeX{} is secure, \booktitle{The
  \LaTeX{} Companion}, 2nd~Ed., by Frank Mittelbach and Michel
Goossens covers a broad range of topics and \LaTeX{} add-on packages.
This text goes far beyond the basics, but it's an indispensable
reference if you're interested in customizing the appearance of
\LaTeX{} documents.
\end{itemize}

\section{Working with \LaTeX{}}

Files you create for processing by \LaTeX{} should have file
extensions of \complit{.tex}, e.g., \complit{mythesis.tex}.  For your
own convenience, you may split the document into pieces (for example,
one file per chapter), which may make the editing process a little
easier by keeping manageable the amount of text you must scroll
through at any one time.

While you're typing your document, you will insert macro commands
(or ``macros'') that mark up your document, indicating chapter and
section headings, equations, tables, figures, etc.  Markup languages
attempt to separate the content of the document from its appearance.
As an author, you need not be quite as concerned about how everything
looks, just what it says.  By marking up your document appropriately,
you can let \LaTeX{} worry about how everything looks.

A \LaTeX{} document begins with a section called ``the preamble''.  In
this section, you set up or change the document-wide processing
settings (like page margins, or selecting the font size, for example).
The rest of the document is called the document ``body''.  Some
\LaTeX{} commands are only allowed in the preamble, while others are
allowed only in the document body.

\section{The \fsuth{} Class}

\label{sec:Class}
\LaTeX{} comes with several pre-defined standard document types (or
classes), such as \latexclass{article}, \latexclass{book}, and
\latexclass{report}.  The \fsuth{} class is an extension of the
\LaTeX{} \latexclass{report} class.  In essence, the \fsuth{}
class provides all the features of \latexclass{report}, along with
customizations to meet the standards of FSU's \booktitle{Guidelines \&
  Requirements for Electronic Theses, Treatises and Dissertations},
revised August~2011.  The rest of this document describes how to use
the features of the \fsuth{} class.

\subsection{Document Files}

Packaged along with this \booktitle{User's Guide} and the \fsuth{}
class file, you will find a folder called \complit{thesis-template}.
Within the folder is a small collection of files, a skeleton upon
which you may build your own document.  I suggest that you copy and
rename this folder in a new location, giving your \emph{magnum opus}
its own workspace.

For now, we'll assume that you have renamed the folder
\complit{thesis}.  Inside the folder, you'll find a file called
\complit{mythesis.tex}.  This will be your document's principal file.
We will assume that you will create additional files in this folder to
add to your document, assuming at least one file per chapter.  You are
free to rename any of these files as you like, as long as they end
with the \complit{.tex} extension.

The document skeleton constitutes a complete document as it stands,
and you may run \LaTeX{} on \complit{mythesis.tex} immediately if you
need to test your installation.  (How you run \LaTeX{} is
platform-dependent, so you may need to refer to the section on
\textsc{Installation} above for references specific to your
environment.)  The rest of this \booktitle{User's Guide} follows the
contents of \complit{mythesis.tex}, demonstrating the features of
\fsuth.

\subsection{Macros and Comments}

\LaTeX{} macros (often used interchangeably with ``commands'') begin
with a `\verb+\+' (backslash) character, followed by text.  Macros
will often take arguments, and possibly optional arguments.  Optional
arguments are usually included in square brackets (e.g.,
\verb+[+\textit{option}\verb+]+) immediately following the macro
invocation.  Required arguments will usually be found in curly braces
(e.g., \verb+{+\textit{This is a required argument}\verb+}+) following
the optional argument (if present).

The percent sign (\verb+%+) is another special character in \LaTeX{}.
It introduces a document comment, which runs to the end of the line.
Commented text is ignored by \LaTeX{} entirely, and will not be typeset.
If you need to print a percent sign as part of your text, precede it
with the backslash character (`\verb+\+').  E.g.,
\begin{verbatim}
... total is 23\% of adjusted gross ...
\end{verbatim}
(See one of the references above for a complete list of \LaTeX{}
special characters.)

\subsection{The Document Preamble}

If you look at the file \complit{mythesis.tex}, you will see that it
consists primarily of \LaTeX{} macros and ``commented out'' lines
containing more \LaTeX{} macros.  As you add text and flesh out your
document, you may ``uncomment'' additional lines in this primary file
by removing the leading percent sign, thereby making the line active.

\subsubsection{Document Options}
The first line of every \LaTeX{} document declares the type of
document to be processed, along with a few processing options.  The
first line of the document skeleton file \complit{mythesis.tex}
contains the following line:
\begin{verbatim}
\documentclass[11pt]{fsuthesis}
\end{verbatim}
This line declares the document type to be \fsuth, and that the text
will be set in 11-point type.

Class \fsuth{} is derived from the \latexclass{report} class, so all
the standard document options supported by \latexclass{report} will be
supported by \fsuth.  (See one of the \LaTeX{} references above for
complete lists of document options.)  The \fsuth{} class supports
three additional document options: \complit{hardcopy},
\complit{chapterleaders}, and \complit{expanded}.

The \complit{hardcopy} option adds extra space along the binding edge
of a page.  This may be useful for printing hard copies for review by
your thesis committee, or if you want to have a professionally bound
copy of your thesis or dissertation.  If you also include the standard
\latexclass{report} option \complit{twoside}, then in addition to the
binding-edge offset, all the chapters of your document will be forced
to start on odd-numbered (right-hand) pages.

The \complit{chapterleaders} option adds leader dots on chapter
headings in the \textsl{Table of Contents}.  Normally, chapter
headings are displayed in bold type with a page number and
\emph{without} leader dots, while by default, sections and subsections
are displayed with leader dots connecting their page numbers.  If you
write a thesis without sections or subsections, or if you suppress
their display in the \textsl{Table of Contents}, then you might want
to specify the \complit{chapterleaders} option.

The \complit{expanded} option makes your document ``double-spaced''.
(In reality, the document is about 1.5-spaced.)  Some colleges, schools, or
departments will prefer expanded spacing to allow committee members to
pencil in comments.

\subsubsection{Extra Packages}

\LaTeX{} has many document feature add-ons.  If you wish to load
additional packages, these options should follow the document class
selection.  Be warned that some packages may not be compatible with
the \fsuth{} class.  Many optional packages may already come installed
with your \TeX/\LaTeX{} distribution, or you can download and install
them from the \acro{CTAN} website ({\tt www.ctan.org}).

In the skeleton document, several \verb|\usepackage| lines have
been commented out.  If you have title, chapter, or section headings
which include mathematics, you may want to uncomment the
\latexclass{textcase} package line, as this will prevent the
titling macros from upper-casing the math symbols inappropriately.

If you will be inserting figures into your document electronically,
you should uncomment the \latexclass{graphicx} line.  You can find
some simple examples of figure inclusion in the \complit{sample}
directory, but for the highest quality output, you owe it to yourself
to learn more about this topic.  Searching the web for ``latex figure
inclusion'' or other similar terms will turn up some useful links.

If you are generating an electronic version of your document
for which you'd like to have hyperlinks automatically connecting
cross-references and entries in the \textsl{Table of Contents}, you
should uncomment the \latexclass{hyperref} line.  If you find that you
are suddenly getting ``Overfull hbox'' errors while using
\latexclass{hyperref} (where there were none before), you could try
adding the line \verb+\hypersetup{breaklinks=true}+ to your document,
inserted just after the \verb+\usepackage{hyperref}+ line.  The
\latexclass{hyperref} package has lots of configuration options, and
you should refer to the package documentation for helpful information.

\subsubsection{Thesis/Dissertation Description Macros}

The next section in \complit{mythesis.tex} contains several macros
that customize the title page and committee page of your document.  As
a general rule, these macros require text arguments that should be
given in mixed case using title capitalization rules (i.e., each word
capitalized, except for articles, prepositions, and conjunctions;
refer to your discipline's preferred style guide if in doubt).  All
proper names should be capitalized normally.  If the FSU
\booktitle{Guidelines} require elements to be displayed differently
(all-caps, for example), the \fsuth{} class will make the adjustments
required for you.

The \verb|\title| macro declares the title of your thesis or
dissertation.  If the title is long, it will be
broken over several lines on the title page.  You have control over
how the title is broken into lines by using the \LaTeX{}
line-separator operator (`\verb+\\+') in the title.  (This command is
what the \LaTeX{} manual calls ``fragile'', and so you must say
`\verb+\protect\\+' when used in the argument of the \verb|\title|
command.)

The \verb|\author| macro gives your name.  Your name should be
given as specified in the FSU \booktitle{Guidelines}.

The \verb|\college| macro should contain the official name of your
school or college.

If your degree comes from a school or college with separate academic
departments which issue degrees, the \verb|\department| macro should
declare this name.  Otherwise, you should comment-out or delete the
\verb|\department| line from your document file.

The \verb|\manuscripttype| should be set to one of the following words, as
appropriate: \complit{Thesis}, \complit{Treatise}, or
\complit{Dissertation}.

The title of your degree (e.g., ``Master of Arts'' or ``Doctor of
Philosophy'') is given by the \verb|\degree| macro.

The macro \verb|\semester| should be set to one of \complit{Fall},
\complit{Spring}, or \complit{Summer}, according to the semester in
which your degree was awarded.

The year your degree is awarded should be given by
\verb|\degreeyear|.  This should be a full 4-digit year.

The date of your thesis, treatise, or dissertation defense should be
specified in the \verb|\defensedate| macro.  Refer to the FSU
\booktitle{Guidelines} for the appropriate format.

If you are generating a \acro{PDF} file, you can add a subject and
search keywords to the document's internal description.  The document
title and author's name will already be included in the document
metadata by default.  To add a subject to the metadata, use the
\verb|\subject{my subject}| macro.  To add search keywords, use the
\verb|\keywords| macro, separating each search term by commas or
semicolons.

\subsubsection{Committee Macros}

The \fsuth{} class provides macros for generating your committee
information page.  The \verb|\committeeperson| macro takes two
arguments.  The first argument is the name of the committee member,
given without titles.  The second argument is the committee membership
status, e.g., ``Professor Directing Dissertation'' or ``Committee
Member''.  (See the FSU \booktitle{Guidelines} about the appropriate
options.)  You should provide one \verb|\committeeperson| line for
each person, in the order in which they should appear on the committee
page.

\subsubsection{Changing Other Settings}

You may change other document defaults while still in the document
preamble.  For example, should you want to change the width of the
text column or the page margins, here's where you would do it.  (Note
that you must still adhere to FSU's \booktitle{Guidelines}, so be sure
you know what you're doing.)

\subsection{The Document Body}

If there are no more adjustments to be made, you begin the document
body with the \LaTeX{} command \verb|\begin{document}|.  You will
notice that whenever you \verb|\begin{|\textit{something}\verb|}|, you
should always supply a corresponding
\verb|\end{|\textit{something}\verb|}|, or \LaTeX{} will complain.  So
at the end of \complit{mythesis.tex}, you will find the
\verb|\end{document}| command.  Anything beyond this point in the file
is ignored by \LaTeX{}.

\subsubsection{Front Matter}

The first element after \verb|\begin{document}| should be the macro
command \verb|\frontmatter|, which sets up roman numeral page
numbering for the document elements that precede the first chapter of
your thesis or dissertation.  The document skeleton in
\complit{mythesis.tex} contains place-holders in the proper order for
all the optional elements of the front matter.  Uncomment those
elements that you will use, or you may leave commented or delete those
elements that you don't use.

Immediately following \verb|\frontmatter|, the macro commands
\verb|\maketitle| and \verb|\makecommitteepage| generate the document
title and committee pages, respectively.  Information for these pages
is gathered from the data you have already set in macro calls in the
preamble.

If you wish to include a dedication in your thesis or dissertation,
uncomment the \verb|\begin{dedication}| and \verb|\end{dedication}|
lines, and type your dedication between them.  The text that you
insert will appear about 1/3rd of the distance from the top of the
page.  The rest of the formatting is up to you.

Likewise, if you wish to include acknowledgments in your document,
uncomment the \verb|\begin{acknowledgments}| and
\verb|\end{acknowledgments}| lines, and insert the acknowledgment text
between these lines.  The resulting page will have the centered heading
\textbf{ACKNOWLEDGMENTS}, followed by your text.  If you wish to
rename the heading, add the following line to your document
preamble:
\begin{verbatim}
\renewcommand*{\acknowledgename}{My Acknowledgement Heading}
\end{verbatim}

The next item in the front matter is the \textsl{Table of Contents},
which is generated for you automatically by the macro
\verb|\tableofcontents|.  By default, the \textsl{Contents} page(s)
will contain entries for the remaining front matter material, and
entries for chapter headings, section headings, and subsection
headings.  If listing section or subsection headings provides too
much detail for your taste, you may remove these entries by resetting
the \LaTeX{} counter \complit{tocdepth}.  \LaTeX{} considers chapter
headings to be Level~0, section headings to be Level~1, and so on.
The default setting of \complit{tocdepth} is~2 (so subsection
headings are included).  To include only chapter and section headings
in the \textsl{Contents}, for example, you could reset
\complit{tocdepth} in the document preamble with the following line:
\begin{verbatim}
\setcounter{tocdepth}{1}
\end{verbatim}

The FSU \booktitle{Guidelines} state that if you have more than one
figure or table in your document, the figures and/or tables should be
contained in their own lists.  Turn each of these options on by
uncommenting the \verb|\listoftables| and/or \verb|\listoffigures|
lines in \complit{mythesis.tex}.  These tables will be generated for
you automatically when your document is processed.  For those
documents which contain multiple musical examples, a list of these may
also be generated by uncommenting \verb|\listofmusex|.

It is sometimes the case that a \textsl{List of Symbols} or a
\textsl{List of Abbreviations} might be helpful to your readers.
If you wish to include such document elements, uncomment the
appropriate \verb|\begin|--\verb|\end| pair, and add any text you may
require.  These entities would likely consist of tabular material, so
you'll want to dig into \LaTeX{} table-making using any of the basic
references mentioned earlier.

The last common element of the front matter is a document abstract.
Insert your text between the abstract \verb|\begin|--\verb|\end| pair.
If you wish to change the default heading of \textbf{ABSTRACT}, you
may do so by adding the following line to your document preamble:
\begin{verbatim}
\renewcommand*{\abstractname}{My Abstract}
\end{verbatim}

\subsection{The Main Text}

At last, with the preliminaries out of the way, you may now get to the
meat of your document.  Following the abstract, the command
\verb|\mainmatter| restarts page numbering at ``1'' in arabic
numerals, ready for your first chapter.

The skeleton file \complit{mythesis.tex} has been set up to include
the first chapter from an external file.  Note the command
\begin{verbatim}
\input chapter1
\end{verbatim}
This tells \LaTeX{} to insert the text of the file
\complit{chapter1.tex} at this position and continue processing.
There is nothing special about the file names, except that they should
end with the extension \complit{.tex}.  Otherwise, you may call the
external files whatever you like.  (However, avoid using filenames
with spaces or special symbols, as these may be difficult for either
\LaTeX{} or your operating system to handle properly.)  You can break
up large chapters into even smaller pieces if you like, and then
change \complit{mythesis.tex} accordingly, e.g.,
\begin{verbatim}
\input chapter1a
\input chapter1b
\end{verbatim}
Or you could just continue adding text to \complit{mythesis.tex}
directly, avoiding having to deal with any other external files
entirely.  This is all up to you.

\subsubsection{Sectioning/Heading Macros}

Several levels of headings are provided by the \fsuth{} class, in the
heading styles defined by FSU's \booktitle{Guidelines}.  By default,
entries down to the subsection level are listed in the \textsl{Table
  of Contents}.  (See the description of the \verb|\tableofcontents|
macro above for information on changing this default.) Listed
from the highest level down, these sectioning commands are:
\begin{itemize}\addtolength{\parskip}{-8pt}
\item \verb+\chapter+
\item \verb+\section+
\item \verb+\subsection+
\item \verb+\subsubsection+
\item \verb+\paragraph+
\item \verb+\subparagraph+
\end{itemize}
Each of these macros take a single argument, the text of the heading.
All headings should be capitalized as titles, i.e., mixed case text,
each word capitalized except articles, prepositions, and conjunctions.
Chapter headings will force the start of a new page.  The file
\complit{chapter1.tex} in the \complit{thesis} folder has some example
text to get you started.  If chapter titles include mathematics, you
may want to uncomment the \latexclass{textcase} \verb|\usepackage|
line near the beginning of the template document should you find your
symbols becoming inappropriately upper-cased.

By default, section and subsection headings are prefixed by section
and subsection numbers. Sub-subsections produce an unnumbered
in-line heading as the opening of a paragraph.  Paragraph and
sub-paragraph headings also produce in-line headings and start new
paragraphs, but with subtler font selections.

You may change the level at which the heading macros produce numbered
entries by setting \complit{secnumdepth}.  The default setting is
level~two, which means that subsections will be numbered automatically.
To stop numbering at the \complit{section} level (for example), reduce
the value of \complit{secnumdepth} to one by issuing the following
command in the document preamble:
\begin{verbatim}
\setcounter{secnumdepth}{1}
\end{verbatim}
By setting \complit{secnumdepth} to zero, you may disable all heading
numbering except at the chapter level.  Or you may increase the value
up to five to generate heading numbers all the way down to the
sub-paragraph heading level.

\subsubsection{Insertions: Figures, Tables, Musical Examples}

The \fsuth{} class provides the standard \LaTeX{} environments for
\complit{figure}s and \complit{table}s.  An additional environment
called \complit{musex} has been added for those authors who need to
provide musical examples.  The \complit{musex} environment behaves
similarly to the \complit{figure} environment, except that captions
include the heading ``Example'' instead of ``Figure'', and all the
musical examples can be listed in the front matter in the \textsl{List
  of Musical Examples}.

By setting material off in a \complit{figure}, \complit{table}, or
\complit{musex} environment, the material will be allowed to drift
from its position in the text to the closest available location as
follows: if there is space for the material at the bottom of the
current page, it will be placed there; otherwise, it will be placed at
the top of the next page, or perhaps on a page by itself.  (You have
some control over the placement of floating elements.  For more
detail, you'll need to consult one of the \LaTeX{} references.)

Each \complit{figure}, \complit{table}, or \complit{musex} should
contain a \LaTeX{} \verb+\caption+ macro whose single argument
contains the text of the caption.  For figures and musical examples,
the caption should be placed below the figure or musical example.  For
tables, the caption should be located above the tabular material.
Examples of the use of each of these environments can be found in the
in the \complit{sample} directory.

\LaTeX{} keeps track of the number of tables, figures, and musical
examples, and your caption will be labeled and numbered automatically.
The caption text will also be inserted into the appropriate
\mbox{\textsl{List of \ldots}} if you requested the list in the
front matter of your document.

\LaTeX{} has many features to assist you in producing tabular material
of arbitrary complexity.  Also, simple diagrams may be created using
the \LaTeX{} \complit{picture} environment.  If you want to include
graphics generated by external software, then you'll need to learn to
use the features of the \latexclass{graphicx} package, and you should
uncomment the appropriate \verb+\usepackage+ command in
\complit{mythesis.tex} preamble.  You are strongly advised to refer to
the \LaTeX{} references cited earlier to learn more about figures and
tables if you intend to use them in your manuscript.

\subsubsection{Cross References}

One of the advantages of working with \LaTeX{} is the ability to
automatically cross-reference equations, figures, tables, and musical
examples.  In writing and revising your manuscript, it is likely that
references to elements may shift as text is added or moved around.
\LaTeX{} addresses this problem by allowing you to assign a
\textit{label key} to each element.  Then you make a reference to an
element's label key in your text to retrieve its number or
page location.  When your document is processed, \LaTeX{} replaces all
the label key references with their numerical values.

As an example, let's take a look at how this might work if we wish to
refer to an equation in our text.  In the left column is the result of
what we've typed in the right column (unimportant text omitted for brevity).

\bigskip
\noindent
\begin{minipage}[t]{2.5in}%
  \small\sloppy
  Leonhard Euler was a prolific mathematician whose
  pioneering work in power series helped to develop the field
  of mathematical analysis.  Equation~\ref{eq:euler-id} on
  page~\pageref{eq:euler-id} is known as
  \textit{Euler's Identity}, what physicist Richard Feynman called
  ``the most remarkable formula in mathematics''.
  \begin{equation}
    e^{i\pi} + 1 = 0
    \label{eq:euler-id}
  \end{equation}
\end{minipage}
\hfill
\begin{minipage}[t]{2.1in}%
\small
\begin{verbatim}
Leonhard Euler was a prolific
...
Equation~\ref{eq:euler-id} on
page~\pageref{eq:euler-id} is
...
\begin{equation}
e^{i\pi} + 1 = 0
\label{eq:euler-id}
\end{equation}
\end{verbatim}
\end{minipage}

\bigskip
The \complit{equation} environment automatically numbers equations for
us.  The macro \verb+\label{eq:euler-id}+ creates the label key
``\complit{eq:euler-id}'', tied to the automatically-numbered
equation.  If we want to access the equation number, we may use the
\verb+\ref{eq:euler-id}+ macro, while the macro
\verb+\pageref{eq:euler-id}+ retrieves the page number.  For figures,
tables, or musical examples, the \verb+\label+ command should be
located within the \verb+\caption+ text.

Since your text may \verb+\ref+ label keys before the corresponding
\verb+\label+ has been encountered, you will need to run your document
through the \LaTeX{} processor \emph{at least twice}.  The first pass
will write all the label keys and page numbers out to an auxiliary
file, and the second pass will then be able to resolve all the
references properly.  (\LaTeX{} will complain about unresolved or
changed references, reminding you to run the processor a second time.)

As you're writing your document, you might want to keep a list of the
label keys you've created so that you don't have to surf through other
files to recall what a particular label key was.  Keep in mind that
figures, tables, musical examples, and equations all use the same
label system, and all label keys must be unique.  You may develop your
own label key standards (like using \complit{eq:} when referencing an
equation to avoid label ``collision'', for instance).  If you expect
to have lots of figures, tables, etc., you may find it helpful to use
descriptive label keys rather than generic ones, as they may be easier
to remember.  E.g., \complit{fig:map-Europe-pre1914} is probably more
mnemonic than \complit{fig:MapOne}.

\subsection{Back Matter}

Following the major chapters of your manuscript, you may have
additional material for one appendix or more.  To shift from chapter
headings to appendix headings, insert the macro \verb+\appendix+ at
the end of your last chapter, before the first appendix.  Then use the
\verb+\chapter+ macro just as you have for each of your chapters.
(Appendices will be lettered rather than numbered.)

\subsubsection{References/Bibliography}

The \fsuth{} class provides two options to produce a bibliography or
references section.  No matter which of the two options you choose,
the heading of the section may be set in the document preamble with
the following command:
\begin{verbatim}
\renewcommand*{\bibname}{Bibliography}
\end{verbatim}
According to FSU's \booktitle{Guidelines}, this section should be
called ``References'' if entries in the section contain only
references cited in the text of your manuscript.  The section should
be called ``Bibliography'' if the entries in this section cover a
broader scope of material than is actually cited in your manuscript.

The first (and simplest) option is to use the \complit{references}
environment.  Begin this section with \verb+\begin{references}+.
(Despite the environment's name, the section heading is still
determined by setting \verb+\bibname+.)  Then add each bibliographic
entry with a blank line between each reference.  Follow the last entry
with \verb+\end{references}+.  With this option, you will have to
format each entry according to the style guide you have chosen to
follow.

The second option is to set up a \BibTeX{} database.  To use
\BibTeX, you create or download a separate file of reference materials
in a particular format.  Then you may cite any of these references
within your manuscript using the \verb+\cite+ macro.  By running
\LaTeX{} in combination with \BibTeX, citations are resolved (similar
to how labels are resolved), and a list of the cited references are
pulled into your document automatically.  To use this feature, you
first select the bibliographic style, and then specify the \BibTeX{}
database file:
\begin{verbatim}
\bibliographystyle{plain}
\bibliography{myrefs}
\end{verbatim}
This selects the \complit{plain} bibliography style, and the
\BibTeX{} database is said to reside in \complit{myrefs.bib}.
Processing your document now requires a few extra steps as well:
\begin{itemize}\addtolength{\parskip}{-6pt}
\item Run \LaTeX{}
\item Run \BibTeX{}
\item Run \LaTeX{} \emph{twice more}
\end{itemize}

If you have a relatively small number of bibliographic entries or
citations, then choosing the \complit{references} environment is
probably the easiest solution.  However, if you are trying to manage a
large number of citations or work in a discipline that has already
established a large \BibTeX{} database, then it may save you
considerable effort to learn how to use \BibTeX, in which case you
will certainly need to use one of the \LaTeX{} references mentioned
earlier.

\subsubsection{Biographical Sketch}

At last, you've reached the final page of your \booktitle{magnum
  opus}.  It will contain your biographical sketch, starting with
\verb+\begin{biosketch}+, and ending with \verb+\end{biosketch}+ as
usual.  Insert what biographical material you wish to include here.

\section{Keeping Things in Order}

If you have lots of figures or musical example files in your document,
you may want to keep these files in the sub-folder already created
for you.  This helps to keep your thesis folder a little less
cluttered.  Then if you have a chart called \complit{pie.eps} stored
in the \complit{figures} folder, you just need to include the folder
name when issuing the \verb+\includegraphics+ command, e.g.,
\begin{verbatim}
\includegraphics{figures/pie.eps}
\end{verbatim}
You can create any number of folders and sub-folders to help keep
your files organized.

\section{More Examples}

The files in the \complit{thesis-template} directory are only a
bare-bones template to help you get started on your own manuscript.
You will find a more complete example of a thesis manuscript in the
\complit{sample} directory.  The \LaTeX{} source files in this
directory contain explanatory comments and many more examples of
useful code.  The file \complit{thesis.pdf} in the \complit{sample}
directory is the result of processing the source files, so you can
easily compare the source files to the output to see how everything
works.  You'll find some simple examples of equations, figures,
tables, and bibliographic citations to help you create your own
document.  You can find much more help from the web should you need
more sophisticated examples.

\section{Bugs, Corrections, Improvements}

Should you discover what you think is a bug in the way that \fsuth{}
formats your document, you may e-mail me at \texttt{bwhissel@fsu.edu}.
It would be helpful to send the portion of your document which you
believe is misbehaving.  Likewise, if you think that the appearance
of theses or dissertations may be improved in some way, or if you have
some macro definitions that you think may be generally useful and
could be added to \fsuth, I am happy to hear your ideas.

Also, if you think that any of this documentation is misleading or
unclear, \emph{please} let me know.  I wish to make this
\booktitle{User's Guide} and the \fsuth{} class as helpful as
possible.

Please note that I cannot help you to learn features of \LaTeX: there
are many resources and tutorials that are freely available, and I am
unable to support individual requests for help with anything that does
not pertain directly to the \fsuth{} class.

\bigskip
\noindent
Best wishes, and good luck!

\end{document}
